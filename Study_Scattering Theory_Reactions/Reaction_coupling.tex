\documentclass[11pt]{book}

\usepackage{kotex} % korean tex
\usepackage{float}
\usepackage{longtable}
\usepackage[bookmarks]{hyperref}
\usepackage[utf8]{inputenc} % set input encoding (not needed with XeLaTeX)
\usepackage{listings}
\lstset{language=[90]Fortran,
  basicstyle=\ttfamily,  
  %basicstyle=\small,
  keywordstyle=\color{red},
  commentstyle=\color{green},
  morecomment=[l]{!\ }% Comment only with space after !
}

\usepackage{geometry} % to change the page dimensions
\geometry{letterpaper} % or letterpaper (US) or a5paper or....
% \geometry{margins=2in} % for example, change the margins to 2 inches all round
% \geometry{landscape} % set up the page for landscape
%   read geometry.pdf for detailed page layout information

\usepackage{graphicx} % support the \includegraphics command and options

% \usepackage[parfill]{parskip} % Activate to begin paragraphs with an empty line rather than an indent

\usepackage{booktabs} % for much better looking tables
\usepackage{array} % for better arrays (eg matrices) in maths
\usepackage{paralist} % very flexible & customisable lists (eg. enumerate/itemize, etc.)
\usepackage{verbatim} % adds environment for commenting out blocks of text & for better verbatim
\usepackage{subfig} % make it possible to include more than one captioned figure/table in a single float
% These packages are all incorporated in the memoir class to one degree or another...

%%% HEADERS & FOOTERS
\usepackage{fancyhdr} % This should be set AFTER setting up the page geometry
\pagestyle{fancy} % options: empty , plain , fancy
\renewcommand{\headrulewidth}{0pt} % customise the layout...
\lhead{}\chead{}\rhead{}
\lfoot{}\cfoot{\thepage}\rfoot{}

%%% SECTION TITLE APPEARANCE
\usepackage{sectsty}
\allsectionsfont{\sffamily\mdseries\upshape} % (See the fntguide.pdf for font help)
% (This matches ConTeXt defaults)

%%% ToC (table of contents) APPEARANCE
\usepackage[nottoc,notlof,notlot]{tocbibind} % Put the bibliography in the ToC
\usepackage[titles,subfigure]{tocloft} % Alter the style of the Table of Contents
\renewcommand{\cftsecfont}{\rmfamily\mdseries\upshape}
\renewcommand{\cftsecpagefont}{\rmfamily\mdseries\upshape} % No bold!

\usepackage{amsmath}
\usepackage{amssymb}
\usepackage{epsfig}
\usepackage{color}
\parindent 10pt\textheight 9in\topmargin -0.4in\textwidth 6in
\oddsidemargin .25in\evensidemargin 0in
\def\bm{\boldsymbol}
\def\vh{{\bm h}}
\def\vp{{\bm p}}
\def\vq{{\bm q}}
\def\vk{{\bm k}}
\def\vl{{\bm l}}
\def\vx{{\bm x}}
\def\vy{{\bm y}}
\def\vv{{\bm v}}
\def\vr{{\bm r}}
\def\vR{{\bm R}}
\def\la{\langle}
\def\ra{\rangle}

\newcommand{\bea}{\begin{eqnarray}}
\newcommand{\eea}{\end{eqnarray}}
\newcommand{\be}{\begin{eqnarray}}
\newcommand{\ee}{\end{eqnarray}}
\newcommand{\no}{\nonumber \\}
\newcommand{\nnb}{\nonumber}
\newcommand{\etal}{{\it et al.}~}
\newcommand{\eg}{{\it e.g.}}
\newcommand{\ie}{{\it i.e.}}
\newcommand{\sll}[1]{#1\hspace{-0.5em}/}
\newcommand{\del}{\partial}
\newcommand{\threejsymbol}[6]{\left(\begin{tabular}{ccc} {$#1$}&{$#2$}&{$#3$}\\
                             {$#4$}&{$#5$}&{$#6$}\end{tabular}\right)}
\newcommand{\sixjsymbol}[6]{\left\{\begin{tabular}{ccc} {$#1$}&{$#2$}&{$#3$}\\
                             {$#4$}&{$#5$}&{$#6$} \end{tabular}\right\}}
\newcommand{\ninejsymbol}[9]{\left\{\begin{tabular}{ccc}
                             {$#1$}&{$#2$}&{$#3$}\\
                             {$#4$}&{$#5$}&{$#6$}\\
                             {$#7$}&{$#8$}&{$#9$}\end{tabular}\right\}}

%%% The "real" document content comes below...

\title{Couplings in coupled channel calculation(FRESCO)}
\author{Young-Ho Song}
\date{\today}
%\date{} % Activate to display a given date or no date (if empty),
         % otherwise the current date is printed 

\begin{document}
\maketitle
\tableofcontents
\newpage

\chapter{Coupled Channel equations} 

\section{Model space wave function}
Let us denote the model space wave function of scattering system as
\bea 
|\Psi\ra =\sum_\alpha |\phi_\alpha\ra \psi_\alpha(R_\alpha)
\eea 
where $\alpha$ includes quantum numbers of partitions,
$|\alpha\ra\equiv |x p t; (L I_p)J_p ,I_t; J_{tot} M_{tot}\ra $
where $(x p t)$ represents partition of projectile and target, $L$ is a orbital angular momentum 
between projectile and target, $I_p$ is a spin of projectile, ${\bm J}_p={\bm L}+{\bm I}_p$,
$I_t$ is a spin of target, $J_{tot} M_{tot}$ is a total angular momentum of system and its projection
and $R_\alpha$ is a coordinates between projectile and target.

In other words, more explicit form of $|\phi_\alpha\ra$ is
\footnote{This expression follows the convention in Thompson's textbook.}    
\bea 
|\phi_\alpha\ra &=&  \left[ \left[i^{L} Y_{L}(\hat{R}_x)\otimes \phi^{xp}_{I_p}(\xi_p)\right]_{J_p} 
\otimes \phi^{xt}_{I_t}(\xi_t)\right]_{J_{tot}M_{tot}},
\eea 
where $\phi^{xp}_{I_p}(\xi_p)$ and $\phi^{xt}_{I_t}(\xi_t)$ are bound state wave functions of 
projectile and target with $\xi_p$ and $\xi_t$ as internal coordinates of projectile and target.


\begin{itemize} 
\item For a system with specific total angular momentum, wave function becomes
\bea 
\Psi_{x,J_{tot}}^{M_{tot}}(R_x,\xi_p,\xi_t)
&=&\sum_{\alpha} \left[ 
    \left[i^{L} Y_{L}(\hat{R}_x)\otimes \phi^{xp}_{I_p}(\xi_p)\right]_{J_p} 
    \otimes \phi^{xt}_{I_t}(\xi_t)\right]_{J_{tot}M_{tot}}
    \frac{1}{R_x}f_{\alpha}^{J_{tot}}(R_x) \no 
&=&\sum_{\alpha} |xpt:(L I_p) J_p,I_t;J_{tot} M_{tot}\ra 
    \frac{f_{\alpha}^{J_{tot}}(R_x) }{R_x} \no 
&=&\sum_{\alpha} |\alpha; J_{tot} M_{tot}\ra \frac{f_{\alpha}^{J_{tot}}(R_x) }{R_x}
\eea 
where $\alpha=(x,p,t,L,I_p,J_p,I_t)$.
 
\item For a wave function from initial plane wave with momentum $\vk_i$ and spin projections of
    $\mu_{p_i}$ and $\mu_{t_i}$ can be written as 
\bea 
\Psi_{x_ip_it_i}^{\mu_{p_i}\mu_{t_i}}(R_x,\xi_p,\xi_t; \vk_i)
=\sum_{J_{tot}M_{tot}}\sum_{\alpha \alpha_i}
  |\alpha;J_{tot}M_{tot}\ra 
   \frac{f^{J_{tot}}_{\alpha \alpha_i}(R_x)  }{R_x} 
   A^{J_{tot} M_{tot}}_{\mu_{p_i}\mu_{t_i}}(\alpha_i,\vk_i)
\eea  
where, "incoming coefficient" is defined as
\bea 
A^{J_{tot} M_{tot}}_{\mu_{p_i}\mu_{t_i}}(\alpha_i,\vk_i)
=\frac{4\pi}{k_i}\sum_{M_i m_i} Y^*_{L_iM_i}(\vk_i)
  \la L_iM_i,I_{p_i} \mu_{p_i}| J_{p_i} m_i\ra 
  \la J_{p_i} m_i,I_{t_i}\mu_{t_i}|J_{tot} M_{tot}\ra. 
\eea  
with asymptotic normalization is 
\bea 
f_{\alpha\alpha_i}(R_x)\to 
\frac{i}{2}[H_\alpha^{(-)}(\eta_\alpha,k_\alpha R_\alpha)\delta_{\alpha\alpha_i}
           -S_{\alpha\alpha_i}(k_\alpha) H_\alpha^{(+)}(\eta_\alpha,k_\alpha R_\alpha)]
\eea 

\end{itemize}

where $H^{\pm}$ are Coulomb Hankel functions.

\section{Coupled channel equations}

One can group full Hamiltonian as
\bea
H= H_{xt}({\bm \xi}_t)+H_{xp}({\bm \xi}_p)+\hat{T}_x({\bm R}_x)+{\cal V}_{xtp}({\bm R}_x,{\bm \xi}_t,{\bm \xi}_p).
\eea 
where 
\bea 
H_{xp}({\bm \xi}_p) \phi^{xp}_{I_p}({\bm \xi}_p)&=&\epsilon_{xp}\phi_{I_p}^{xp}({\bm \xi}_p),\no 
H_{xt}({\bm \xi}_t) \phi^{xt}_{I_t}({\bm \xi}_t)&=&\epsilon_{xt}\phi_{I_t}^{xt}({\bm \xi}_t),
\eea 
relative kinetic term and interaction between target and projectile becomes 
\bea 
\hat{T}_x({\bm R}_x)&=&-\frac{\hbar^2}{2\mu_x}\nabla_{R_x}^2,\no 
{\cal V}_{xtp}({\bm R}_x,{\bm \xi}_t,{\bm \xi}_p)&=&\sum_{i\in p, j\in t} V_{ij}(\vr_i-\vr_j).
\eea 
with reduced mass $\mu_x=\frac{m_{xp}m_{xt}}{m_{xp}+m_{xt}}$.

Assume the Model space Wave function $\Psi$ satisfies,
\bea 
0&=&[{\cal H}-E]|\Psi\ra 
  =[{\cal H}-E]|\phi_i\ra \psi_i +
    [{\cal H}-E]\sum_{j\neq i}|\phi_j\ra \psi_j
\eea 
If we multiply $\la \phi_i|$ on the left hand side, for each channel $i$, 
\bea 
\la \phi_i|E-{\cal H}|\phi_i\ra \psi_i
= -\sum_{j\neq i}\la \phi_i|E-{\cal H}|\phi_j\ra \psi_j.
\eea 
The left hand side can be simplified by
\bea 
\la \phi_i|E-{\cal H}|\phi_i\ra \psi_i
&=&\la \phi_i|[E-H_i^{bd}-T_i-V_i]|\phi_i\ra \psi_i
=\la \phi_i|[E_i-T_i-V_i]|\phi_i\ra \psi_i \no 
&=&(E_i-T_i(R_i)-\la\phi_i|V_i|\phi_i\ra(R_i))\psi_i(R_i)
\eea 
where $V_i={\cal V}_{xtp}({\bm R}_x,{\bm \xi}_t,{\bm \xi}_p)$ ,
$\la\phi_i|V_i|\phi_i\ra(R_i)$ implies integration over projectile and target bound state internal coordinates $\xi_p,\xi_t$.

For the right-hand side, we may use two different ways, in post form
\bea
-\la \phi_i|E-{\cal H}|\phi_j\ra \psi_j
&=& \la \phi_i|T_i-E_i+V_i|\phi_j\ra \psi_j 
   = (T_i-E_i) \la \phi_i|\phi_j\ra\psi_j
     +\la \phi_i|V_i|\phi_j\ra \psi_j  
\eea 
or in prior form
\bea 
-\la \phi_i|E-{\cal H}|\phi_j\ra \psi_j
&=& \la \phi_i|T_j-E_j+V_j|\phi_j\ra \psi_j 
   =  \la \phi_i|\phi_j\ra (T_j-E_j)\psi_j
     +\la \phi_i|V_j|\phi_j\ra \psi_j  
\eea 

For the same partition, $x'=x$, we may normalize 
$\hat{N}_{\alpha'\alpha}=\la \phi_{\alpha'}|\phi_{\alpha}\ra=\delta_{\alpha'\alpha}$.
However, in general $\la \phi_i|\phi_j\ra$ are non-orthogonal and may not commute with  $(T_i-E_i)$.

Thus, we get 
\bea 
(E_i-T_i(R_i)-\la\phi_i|V_i|\phi_i\ra(R_i))\psi_i(R_i)
&=&\sum_{j\neq i}\left[ (T_i-E_i) \la \phi_i|\phi_j\ra\psi_j
     +\la \phi_i|V_i|\phi_j\ra \psi_j \right] ,\no 
&=&\sum_{j\neq i} \left[ \la \phi_i|\phi_j\ra (T_j-E_j)\psi_j
     +\la \phi_i|V_j|\phi_j\ra \psi_j\right]       
\eea 

If we introduce auxiliary optical potential $U_i$, we may rearrange in "post form"
\bea 
(E_i-T_i-U_i)\psi_i
&=& \left( \sum_{j\neq i} \la \phi_i|V_i|\phi_j\ra \psi_j \right)
   +\left( [\la\phi_i|V_i|\phi_i\ra-U_i]\psi_i \right) \no & &
   +\left( \sum_{j\neq i}  (T_j-E_j)\la \phi_i|\phi_j\ra \psi_j \right)
\eea 
Or, in "prior form",
\bea     
(E_i-T_i-U_i)\psi_i
&=& \left( \sum_{j\neq i} \la \phi_i|V_j|\phi_j\ra \psi_j \right)
   +\left( [\la\phi_i|V_i|\phi_i\ra(R_i)-U_i]\psi_i \right) 
   \no & & 
   +\left( \sum_{j\neq i}  \la \phi_i|\phi_j\ra  (T_j-E_j)\psi_j \right) ,\no   
\eea 
Let us denote 
\bea 
V^{prior}(R_i,R_j)&=&\la \phi_i|V_j|\phi_j\ra,\quad 
V^{post}(R_i,R_j)=\la \phi_i|V_i|\phi_j\ra,
\eea 
If we define $U_i(R_i)=\la \phi_i|V_i|\phi_i\ra$, 
diagonal term in the right hand side would vanish. 

For radial wave function $\psi_\alpha=f_\alpha/R$ in a prior form for rearrangement, 
while separating inelastic scattering and rearrangements counplings,
\footnote{Compared to the coupled channel equation given in FRESCO document,
(1) we have additional non-orthogonal terms (2) diagonal term in the right side can be
ignored (3) We have not expanded in multipole of couplings yet (4) $i^{L-L'}$ factor 
are hidden in the definition of $|\phi_\alpha\ra$. 	

}   
\bea 
& &(E_{xpt}-T_{xL}(R_x)-U_x(R_x))f_\alpha(R_x) \no 
& &\quad =\la \phi_\alpha| V_x-U_x|\phi_\alpha\ra(R_x) f_\alpha(R_x) 
+\left( \sum_{\alpha'\neq \alpha} V_{\alpha \alpha'}(R_x)f_{\alpha'}(R_x)\right) 
\no 
& &\quad  +\left(\sum_{x\neq x',\alpha' }\int dR_{x'}  V^{prior}_{\alpha\alpha'}(R_x,R_{x'})
               f_{\alpha'}(R_{x'})\right) 
\no & &\quad               
          +\left( \sum_{\alpha\neq \alpha'}\int dR_{x'} N_{\alpha\alpha'}(T_{x'L'}-E_{x'p't'})f_{\alpha'}(R_{x'})\right)    
\eea 
where
\bea 
T_{xL}(R)&=&-\frac{\hbar^2}{2\mu_x}\left( \frac{d^2}{dR^2}-\frac{L(L+1)}{R^2}\right),\no 
V_{\alpha\alpha'}(R_x)&=&\la \phi_\alpha|V_x(R_x,\xi)|\phi_{\alpha'}\ra ,\no 
V^{prior}_{\alpha\alpha'}(R_x,R_{x'})&=& R_{x}
                     \la \phi_\alpha|V^{prior}(R_{x'},\xi)|\phi_{\alpha'}\ra\frac{1}{R_{x'}},\no 
N_{\alpha\alpha'}(R_x,R_{x'})&=& R_x\la \phi_\alpha|\phi_{\alpha'}\ra \frac{1}{R_{x'}}.                     
\eea 

The actual form of couplings requires explicit form of potentials in a certain model.
For example, inelastic channel coupling which have the same partition, 
\bea 
V_{\alpha\alpha'}(R_x)=\int d\xi_p d\xi_t \la i^{L} [Y_L(\hat{R}_x)\otimes \phi_p(\xi_p)]\phi_t(\xi_t)|
    V_x(R_x,\xi_p,\xi_t)|i^{L'} 
    [Y_{L'}(\hat{R}_{x})\otimes \phi_{p'}(\xi_{p'})]\phi_{t'}(\xi_{t'})\ra 
\eea 
can be described as an collective model or single particle excitation models
for the integration over internal coordinates. 

Also, one may further multipole expand the couplings in terms of transferred angular momentums,
$\Delta L,\Delta S,\Delta J$. 


\chapter{Couplings, Interactions}

\section{Multipole expansion of interaction}
In general, the reaction occurs through the interaction between target and projectile,
\bea 
{\cal V}_{xtp}({\bm R}_x,{\bm \xi}_t,{\bm \xi}_p)&=&\sum_{i\in p, j\in t} V_{ij}(\vr_i-\vr_j).
\eea 
In coupled reaction channel calculation, we need to compute (if there is no rearrangement), 
\bea 
V_{\alpha,\alpha'}(R_x)=\la \alpha| {\cal V}_{xtp}({\bm R}_x,{\bm \xi}_t,{\bm \xi}_p)|\alpha'\ra 
\eea 
where integration over internal coordinates and angular integration of ${\bm R}_x$
is implied. 
By factoring out factor $i^{L-L'}$, we can write 
\bea 
V_{\alpha\alpha'}(R_x)=\int d\Omega_{R} d\xi_p d\xi_t 
\left \la [Y_L(\hat{R}_x)\otimes \phi_p(\xi_p)]_{J_p}\otimes \phi_t(\xi_t)\right|
V_x({\bm R}_x,\xi_p,\xi_t)\left| 
[Y_{L'}(\hat{R}_{x})\otimes \phi_{p'}(\xi_{p'})]\otimes \phi_{t'}(\xi_{t'})\right\ra 
\eea 

By tensor decompose of the potential, we may write\footnote{I suppose
one would have $T_{\lambda -\mu}(R,{\bm \xi}_t,{\bm \xi}_p)$ in general,
instead of  $V_\lambda(R)T_{\lambda -\mu}({\bm \xi}_t,{\bm \xi}_p)$.
 }
\bea 
{\cal V}_{xtp}({\bm R}_x,{\bm \xi}_t,{\bm \xi}_p)
=\sqrt{4\pi} \sum_{\lambda\mu} V_\lambda(R) Y_{\lambda\mu}(\hat{R}) T_{\lambda -\mu}({\bm \xi}_t,{\bm \xi}_p)
\eea 
Then, the matrix elements for states $\la \alpha'|$ and $|\alpha\ra$
can be expressed in terms of the product of form factor and reduced matrix elements
, $V_\lambda(R)\la \alpha||T_{\lambda}({\bm \xi}_t,{\bm \xi}_p)||\alpha'\ra $, and 
other all coefficients related with shuffling states and tensor decomposition of  
potential. 

In FRESCO, 
the potential operators are decomposed into multipoles and the matrix elements related with
orbital angular momentums are computed by FRESCO so that one only have to give inputs for 
reduced matrix elements and form factors of the coupling $V_\lambda(R)\la \alpha||T_{\lambda}({\bm \xi}_t,{\bm \xi}_p)||\alpha'\ra $. In some special model, even the reduced matrix elements or
form factors will be computed automatically.   

This is explained in the manual of FRESCO and the textbook of I.J. Thompson.
However, there is some convention difference between two. 



\section{Example of double folding model}
{\bf From now on, I am going to describe how to obtain coupling interaction 
but exact factors should be checked carefully later. (Thus, this is just for rough idea).}

Suppose we have a scalar folding potential $v(\vr_i-\vr_j)$, and full interaction between
projectile and target becomes\footnote{We assumed simple separation of wave function 
	into projectile, target and orbital while ignoring anti-symmetrization of full wave function
of the system.
} 
\bea 
{\cal V}_{xtp}({\bm R}_x,{\bm \xi}_t,{\bm \xi}_p)&=&\sum_{i\in p, j\in t} v({\bm R}+\vr_i-\vr_j) \no 
 &=& \int d{\bm r}_t d{\bm r}_p   v({\bm R}+\vr_p-\vr_t) 
      \sum_{i\in p, j\in t} \delta^{(3)}(\vr_p-\vr_i)\delta^{(3)}(\vr_t-\vr_j)     .
\eea 
We can decompose the channel state as 
\bea 
\left| [Y_{L'}(\hat{R}_{x})\otimes \phi_{p'}(\xi_{p'})]\otimes \phi_{t'}(\xi_{t'})\right\ra
= \sum_{M} C_{\alpha,M} Y_{L'}(\hat{R}) \phi_{I_p}(\xi_p)\phi_{I_t}(\xi_t)
\eea 
where $C_{\alpha,M}$ is a collection of Wigner coefficients with projections $M=\{L_z,I_{pz},I_{tz}, J_z\}$.
Then,
\bea 
V_{\alpha,\alpha'}(R)&=& \la \alpha| {\cal V}_{xtp}({\bm R}_x,{\bm \xi}_t,{\bm \xi}_p)|\alpha'\ra \no 
 &=& \sum_{M,M'} C_{\alpha,M}C_{\alpha',M'} 
  \int d{\bm r}_t d{\bm r}_p   \la Y_{L'}(\hat{R})| v({\bm R}+\vr_p-\vr_t) |Y_L(\hat{R})\ra
  \no & &  \quad \times  
   \la \phi_{I_p}|\sum_{i}  \delta^{(3)}(\vr_p-\vr_i) |\phi_{I'_p}\ra 
   \la \phi_{I_t}|\sum_{j}  \delta^{(3)}(\vr_t-\vr_j) |\phi_{I'_t}\ra    
\eea 
Suppose we can obtain the density (or transition density) from 
\bea 
\rho_{I I'}(\vr) =\la \phi_{I}|\sum_{i}  \delta^{(3)}(\vr-\vr_i) |\phi_{I'}\ra.
\eea  
It will corresponds to density for elastic process if $I=I'$, or  
to transition density for inelastic process if $I\neq I'$.

We may use Fourier transformation for $v(\vr)$,
\bea 
v(\vr)=\frac{1}{(2\pi)^{3}}\int d^3 k e^{i\vk\cdot \vr} \tilde{v}(\vk),
\eea 
then,
\bea 
V_{\alpha,\alpha'}(R)&=&\sum_{M,M'} C_{\alpha,M}C_{\alpha',M'} 
   \int d{\bm r}_t d{\bm r}_p \la Y_{L'}(\hat{R})| v({\bm R}+\vr_p-\vr_t) |Y_L(\hat{R})\ra
   \rho_{I_p I'_p}(\vr_p)\rho_{I_t I'_t}(\vr_t) \no 
   &=& \sum_{M,M'} C_{\alpha,M}C_{\alpha',M'} \int d{\bm r}_t d{\bm r}_p
    \frac{1}{(2\pi)^{3}}\int d^3 k \la Y_{L'}(\hat{R})| \tilde{v}(\vk) e^{i\vk\cdot(\vR+\vr_p-\vr_t) } |Y_L(\hat{R})\ra \rho_{I_p I'_p}(\vr_p)\rho_{I_t I'_t}(\vr_t) \no 
   &=&  \sum_{M,M'} C_{\alpha,M}C_{\alpha',M'}\frac{1}{(2\pi)^{3}}
   \int d^3 k \la Y_{L'}(\hat{R})|  e^{i\vk\cdot \vR } |Y_L(\hat{R})\ra \times 
   \tilde{v}(\vk) \tilde{\rho}_{I_p I'_p}(-\vk)\tilde{\rho}_{I_t I'_t}(\vk)
\eea 
where the Fourier transformation of density are introduced. 
Then, by the partial wave expansion of plane wave, we get 
\bea 
V_{\alpha,\alpha'}(R)&=& \sum_{\Delta L} \sum_{M,M'}\sum_{L_z}
  C_{\alpha,M}C_{\alpha',M'}\frac{\sqrt{4\pi}}{(2\pi)^{3}} i^{\Delta L}
  \int d k k^2 j_{\Delta L}(k R) \times 
  \la Y_{L'}|  Y_{\Delta L}(\hat{R}) |Y_L\ra 
  \no & &  \times \int d\Omega_{k} \left( \tilde{v}(\vk) \tilde{\rho}_{I_p I'_p}(-\vk)\tilde{\rho}_{I_t I'_t}(\vk) Y_{\Delta L}^*(\hat{k}) \right) \no 
  &=& \sum_{\Delta L} \sum_{M,M'}\sum_{L_z} 
    C_{\alpha,M}C_{\alpha',M'} V_{\alpha,\alpha'}^{\Delta L}(R) \la Y_{L'}|  Y_{\Delta L}(\hat{R}) |Y_L\ra
\eea 
In the last line, we defined a coupling Form factor  $V_{\alpha,\alpha'}^{\Delta L}(R)$
which includes all integration over internal degrees of freedom. 

By comparing this equation with the expression for the general spin transfer coupling in FRESCO manual
will gives us the exact expression for the input to the FRESCO. 

{\bf More careful analysis of the equation would be necessary.}

\subsection{Spin/Isospin dependent folding potential}
If the folding potential have additional dependence on spin or isospin,
\bea 
v_{12}=v_0(r_{12})+v_1(r_{12})\tau_1\cdot\tau_2+v_2(r_{12})\sigma_1\cdot\sigma_2+\cdots 
\eea  
If the isospin does not change during the process we can simply replace, 
$\tau_1\cdot\tau_2\to \tau_1^z \tau_2^z$ and 
\bea 
\la \phi_{I}|\sum_{i}  \delta^{(3)}(\vr-\vr_i)\tau_i^z |\phi_{I'}\ra
= \rho^p_{II'}(\vr)-\rho^n_{II'}(\vr).
\eea  
In a very special case, $\rho_n=\frac{N}{Z}\rho_p=\frac{N}{A}\rho$, this corresponds to 
\bea 
\la \phi_{I}|\sum_{i}  \delta^{(3)}(\vr-\vr_i)\tau_i^z |\phi_{I'}\ra =\frac{Z-N}{A}\rho(\vr)
\eea 


For spin, we would get vector spin density 
\bea 
\la \phi_{I}|\sum_{i}  \delta^{(3)}(\vr-\vr_i){\bm \sigma}_i|\phi_{I'}\ra
={\vec \rho}_{II'}(\vr)
\eea 

\end{document}


