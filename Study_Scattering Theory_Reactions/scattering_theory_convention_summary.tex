\documentclass[10pt]{article}

\usepackage[utf8]{inputenc} % set input encoding (not needed with XeLaTeX)

\usepackage{geometry} % to change the page dimensions
\geometry{letterpaper} % or letterpaper (US) or a5paper or....
% \geometry{margins=2in} % for example, change the margins 
%to 2 inches all round
% \geometry{landscape} % set up the page for landscape
%   read geometry.pdf for detailed page layout information

\usepackage{graphicx} % support the \includegraphics command and options

% \usepackage[parfill]{parskip} % Activate to begin paragraphs 
%with an empty line rather than an indent

\usepackage{booktabs} % for much better looking tables
\usepackage{array} % for better arrays (eg matrices) in maths
\usepackage{paralist} % very flexible & customisable lists 
%(eg. enumerate/itemize, etc.)
\usepackage{verbatim} % adds environment for commenting 
% out blocks of text & for better verbatim
\usepackage{subfig} % make it possible to include 
%more than one captioned figure/table in a single float
% These packages are all incorporated in the memoir class 
%to one degree or another...

%%% HEADERS & FOOTERS
\usepackage{fancyhdr} % This should be set 
% AFTER setting up the page geometry
\pagestyle{fancy} % options: empty , plain , fancy
\renewcommand{\headrulewidth}{0pt} % customise the layout...
\lhead{}\chead{}\rhead{}
\lfoot{}\cfoot{\thepage}\rfoot{}

%%% SECTION TITLE APPEARANCE
\usepackage{sectsty}
\allsectionsfont{\sffamily\mdseries\upshape} 
% (See the fntguide.pdf for font help)
% (This matches ConTeXt defaults)

%%% ToC (table of contents) APPEARANCE
\usepackage[nottoc,notlof,notlot]{tocbibind} 
% Put the bibliography in the ToC
\usepackage[titles,subfigure]{tocloft} 
% Alter the style of the Table of Contents
\renewcommand{\cftsecfont}{\rmfamily\mdseries\upshape}
\renewcommand{\cftsecpagefont}{\rmfamily\mdseries\upshape} % No bold!

\usepackage{amsmath}
\usepackage{amssymb}
\usepackage{epsfig}
\usepackage{color}

\usepackage{empheq}
% make possible to box equations, For example
%\begin{empheq}[box=\fbox]{align*}
%a&=b \tag{test}\\
%E&=mc^2 + \int_a^a x\, dx
%\end{empheq}

\parindent 10pt\textheight 9in\topmargin -0.4in\textwidth 6in
\oddsidemargin .25in\evensidemargin 0in
\def\bm{\boldsymbol}
\newcommand{\bea}{\begin{eqnarray}}
\newcommand{\eea}{\end{eqnarray}}
\newcommand{\be}{\begin{eqnarray}}
\newcommand{\ee}{\end{eqnarray}}
\newcommand{\no}{\nonumber \\}
\newcommand{\nnb}{\nonumber}
\newcommand{\etal}{{\it et al.}~}
\newcommand{\eg}{{\it e.g.}}
\newcommand{\ie}{{\it i.e.}}
\newcommand{\sll}[1]{#1\hspace{-0.5em}/}
\newcommand{\del}{\partial}
\def\vh{{\bm h}}
\def\vp{{\bm p}}
\def\vq{{\bm q}}
\def\vk{{\bm k}}
\def\vl{{\bm l}}
\def\vx{{\bm x}}
\def\vy{{\bm y}}
\def\vv{{\bm v}}
\def\vr{{\bm r}}
\def\vR{{\bm R}}
\def\la{\langle}
\def\ra{\rangle}
\newcommand{\threejsymbol}[6]
{\left(\begin{tabular}{ccc} {$#1$}&{$#2$}&{$#3$}\\
                             {$#4$}&{$#5$}&{$#6$}\end{tabular}\right)}
\newcommand{\sixjsymbol}[6]
{\left\{\begin{tabular}{ccc} {$#1$}&{$#2$}&{$#3$}\\
                             {$#4$}&{$#5$}&{$#6$} \end{tabular}\right\}}


%%% The "real" document content comes below...

\title{Conventions and equations for scattering}
\author{Young-Ho Song}
\date{\today}
%\date{} % Activate to display a given date or no date (if empty),
         % otherwise the current date is printed 

\begin{document}
\maketitle

\section{S-matrix}
\begin{itemize}

\item units: Let us always use only $\mbox{fm}$ as a fundamental units. Other quantities are always converted by using
$\hbar=c=1$ and $\hbar c=1=197 \mbox{MeV}.\mbox{fm}$.
\item normalization of $|\vk\ra$ and $|\vx\ra$.

\bea
\la \vx|\vk\ra&\equiv&\frac{1}{(2\pi)^\frac{3}{2}}e^{i\vk\cdot\vx}
\no
\int d\vx |\vx\ra \la \vx|&=&1, \quad \la \vx|\vx'\ra
=\delta^{3}(\vx-\vx'),\no 
\int d\vk |\vk\ra \la \vk|&=&1, \quad \la \vk'|\vk\ra=\delta^{3}(\vk-\vk')
\eea

$|\vk\ra$ is a solution of $H_0|\vk\ra=E_k|\vk\ra$.
$|\vk\ra^{(\pm)}$ are solution of $(H_0+V)|\vk\ra^{(\pm)}=E_k|\vk\ra^{(\pm)}$.

In c.m. frame $E_k=\frac{\vk^2}{2\mu}$
with $\vk=\mu {\bm v}_{rel}$ relative momentum
between two cluster, $\mu$ reduced mass.


\item S-matrix
\bea
\la \vk|\hat{S}|\vk'\ra
&=&{}^{(-)}\la\vk|\vk'\ra^{(+)}
=\la \vk|\vk'\ra^{(+)}
 +\lim_{\epsilon\to 0} \la \vk|V \frac{1}{E_k+ i\epsilon-H}|\vk'\ra^{(+)}\no
&=&\la \vk|\vk'\ra
 +\lim_{\epsilon\to 0} \la \vk|V \frac{1}{E_k+ i\epsilon-H}|\vk'\ra^{(+)}
 +\lim_{\epsilon\to 0} \la \vk|\frac{1}{E_k'+ i\epsilon-H_0}V|\vk'\ra^{(+)}\no
&=&\la \vk|\vk'\ra
 +\lim_{\epsilon\to 0}\frac{\la \vk|V |\vk'\ra^{(+)}}
                           {E_k+ i\epsilon-E_{k'}} 
 +\lim_{\epsilon\to 0}\frac{\la \vk|V|\vk'\ra^{(+)}}
                           {E_k'+ i\epsilon-E_k}   
\no
&=&\la \vk|\vk'\ra-2\pi i\delta(E_k-E_{k'})\la \vk|V|\vk'\ra^{(+)}
\eea
i.e.
\bea
\boxed{
\la \vk|\hat{S}|\vk'\ra=\la \vk|\vk'\ra-2\pi i\delta(E_k-E_{k'})\la \vk|V|\vk'\ra^{(+)}
}
\eea

This expression is convention independent

\item scattering amplitude
From LS equation,
\bea
\la\vr|\vk\ra^{+}
&=&\la \vr|\vk\ra+\int d^3\vr' \la \vr|G_0(E+i\epsilon)|\vr' \ra
 \la\vr'|V|\vk\ra^{+} \no 
&=&\la \vr|\vk\ra- \int d^3\vr'\frac{\mu}{2\pi\hbar^2}
   \frac{e^{ik|\vr'-\vr|}}{|\vr'-\vr|}\la \vr'|V|\vk\ra^+
\eea
By using
\bea
\la \vr'|G_0(E\pm i\epsilon)|\vr \ra&=&-\frac{\mu}{2\pi\hbar^2}
              \frac{e^{\pm i k|\vr'-\vr|}}{|\vr-\vr'|},\no 
|\vr-\vr'|&=& r-r' {\hat \vr}\cdot{\hat \vr'}+\dots,\no 
\lim_{\vr\to \infty}\la \vr'|G_0(E\pm i\epsilon)|\vr \ra
   &\to& -\frac{\mu}{2\pi\hbar^2}
   \frac{e^{ikr-ikr'\hat{\vr}\cdot{\hat\vr}'}}{r}
   = -\frac{e^{ikr}}{r}\frac{\mu}{2\pi\hbar^2}e^{-i\vk' \cdot{\vr}'}
\eea
where we defined $\vk'=k\hat{\vr}$.
We can define scattering amplitude asymptotically
\bea
\psi_{out}(\vr)\to_{\vr\to \infty} \frac{1}{(2\pi)^{\frac{3}{2}}}
     \left(e^{i\vk\cdot\vr}+f(\vk',\vk)\frac{e^{ikr}}{r}\right).
\eea
where
\bea
\boxed{
f(\vk',\vk)=-\frac{(2\pi)^{\frac{3}{2}}\mu}{2\pi\hbar^2} 
                    \int d^3\vr' e^{-i\vk'\cdot\vr'}
                    \la \vr'|V|\vk\ra^{(+)}
            =-\frac{(2\pi)^3\mu}{2\pi\hbar^2}\la \vk'|V|\vk\ra^{(+)}
            }
\eea

If convention, $\la \vk|\vk'\ra=(2\pi)^3\delta^{(3)}(\vk-\vk')$
was used, $f(\vk,\vk')=-\frac{\mu}{2\pi\hbar^2}\la \vk'|V|\vk\ra^{(+)}$
 
\item cross section
\bea
\frac{d\sigma}{d\Omega_k}&=&|f(\vk,\vk')|^2 \no
\sigma&=&\int d\Omega_k |f(\vk,\vk')|^2 
       =\frac{4\pi}{k}{\rm Im}f(\vk,\vk')
\eea

\item relation between S-, T-, R- matrix and scattering amplitude
and potential.
\bea
\hat{R}& \equiv& \hat{1}-\hat{S} \no
\la \vk|\hat{T}|\vk'\ra &\equiv& \la \vk|V|\vk'\ra^{(+)} \no
\la \vk|\hat{S}|\vk'\ra
&=&\la \vk|\vk'\ra-2\pi i\delta(E_k-E_{k'})\la \vk|V|\vk'\ra^{(+)}
\no
&=&\la \vk|\vk'\ra-2\pi i\delta(E_k-E_{k'})\la \vk|T|\vk'\ra \no
&=&\la \vk|\vk'\ra
   +\frac{1}{2\pi\mu} i\delta(E_k-E_{k'}) f(\vk',\vk)
\eea

If $\la \vk|\vk'\ra=(2\pi)^3\delta^{(3)}(\vk-\vk')$
convention used,
$$\la \vk|\hat{S}|\vk'\ra
=\la \vk|\vk'\ra-2\pi i\delta(E_k-E_{k'})\la \vk|V|\vk'\ra^{(+)}
=\la \vk|\vk'\ra+\frac{(2\pi)^2}{\mu} i\delta(E_k-E_{k'})
  f(\vk',\vk).$$

\item Energy conservation
\bea
& &\la \vk|\vk'\ra=\delta^{(3)}(\vk-\vk')
  =\frac{\delta(k-k')}{k k'}\delta^{(2)}(\hat{\vk}-\hat{\vk}')\no
& &\delta(E_k-E_{k'})=\frac{\mu}{k}\delta(k-k') 
\eea
\bea
\la \vk|\hat{S}|\vk'\ra
&\equiv&  \frac{\delta(k-k')}{k k'}\la\hat{\vk}|\hat{S}(k)|\hat{\vk}'\ra \no
&=&\frac{\delta(k-k')}{k k'}\left(\delta^{(2)}(\hat{\vk}-\hat{\vk}')
   -2\pi i \mu k \la \vk| T|\vk'\ra \right) \no
&=&\frac{\delta(k-k')}{k k'}\left(\delta^{(2)}(\hat{\vk}-\hat{\vk}')
   +\frac{i k}{2\pi}  f(\vk',\vk)\right) 
\eea

If $\la \vk|\vk'\ra=(2\pi)^3\delta^{(3)}(\vk-\vk')$
convention used,
\bea
\la \vk|\hat{S}|\vk\ra
&=&(2\pi)^3\frac{\delta(k-k')}{k k'}\left(\delta^{(2)}(\hat{\vk}-\hat{\vk}')
   -i \frac{\mu k}{(2\pi)^2}  \la \vk| V|\vk'\ra \right) \no
&=&(2\pi)^3\frac{\delta(k-k')}{k k'}
    \left(\delta^{(2)}(\hat{\vk}-\hat{\vk}')
   +\frac{i k}{2\pi}  f(\vk',\vk)\right)  
\eea
\end{itemize}

\section{Partial wave decomposition}
\begin{itemize}
\item partial wave expansion of plane wave
\bea
\la \vr|\vp\ra=\frac{1}{(2\pi)^{\frac{3}{2}}}e^{i\vp\cdot\vr}
&=& \sum_{lm} \la r lm| p lm\ra \la \hat{r}|r lm\ra 
                                \la p lm |\hat{p}\ra    
\no
&=& \sum_{lm} \sqrt{\frac{2}{\pi}} j_l(pr) i^l Y_{lm}(\hat{r})Y^*_{lm}(\hat{p}) \no
&=& \sum_{l}\sqrt{\frac{2}{\pi}} j_l(pr) i^l Y_{l0}(\hat{r})\sqrt{\frac{2l+1}{4\pi}} 
\mbox{ if $\hat{p}=\hat{z}$}\no
&=& \frac{1}{(2\pi)^{\frac{3}{2}}}
\sum_{l}j_l(pr) i^l(2l+1) P_{l}(\cos\theta) 
\mbox{ if $\hat{p}=\hat{z}$}\no
\eea

If $\la \vk|\vk'\ra=(2\pi)^3\delta^{(3)}(\vk-\vk')$
convention is used, it is convenient to normalize,
\bea
\la \vr|\vp\ra=e^{i\vp\cdot\vr}
=\sqrt{4\pi}\sum_{lm} j_l(pr) i^l Y_{lm}(\hat{r})Y^*_{lm}(\hat{p})
=\sum_l j_l(pr)i^l (2l+1) P_l(\cos\theta)
\mbox{ if } \hat{p}=\hat{z}
\eea

\item Conventions: From partial wave expansion of
plane wave, we have several possible conventions to 
define $\la r lm|plm\ra$, $\la \hat{r}| r lm\ra$,
and $\la \hat{p}|p lm\ra$. Possible choices are

\begin{itemize}
\item Convention 1:
\bea
\la r lm|p lm\ra \equiv\sqrt{\frac{2}{\pi}}j_l(pr) i^l,\quad
\la \hat{r}| r lm\ra \equiv Y_{lm}(\hat{x}),\quad
\la \hat{p}|p lm\ra\equiv Y_{lm}(\hat{p})
\eea
\item Convention 2:
\bea
\la r lm|p lm\ra \equiv\sqrt{\frac{2}{\pi}}j_l(pr),\quad
\la \hat{r}| r lm\ra \equiv i^l Y_{lm}(\hat{x}),\quad
\la \hat{p}|p lm\ra \equiv Y_{lm}(\hat{p})
\eea
\item Convention 3:
\bea
\la r lm|p lm\ra \equiv \sqrt{\frac{2}{\pi}}j_l(pr),\quad
\la \hat{r}| r lm\ra \equiv Y_{lm}(\hat{x}),\quad 
\la \hat{p}|p lm\ra \equiv i^{-l} Y_{lm}(\hat{p})
\eea
\item Convention 4:
\bea
\la r lm|p lm\ra \equiv\sqrt{\frac{2}{\pi}}j_l(pr),\quad
\la \hat{r}| r lm\ra \equiv Y_{lm}(\hat{x}),\quad 
\la \hat{p}|p lm\ra \equiv Y_{lm}(\hat{p})
\eea
\end{itemize}
Actually, we may also introduce different normalization
such that $\la r lm|plm\ra =j_{l}(pr)$. Let us call this 
conventions as $1',2',3',4'$. Among them, let us compare
conventions 2 which will be called as modified spherical 
harmonics convention(ms) and 4 which will be denoted
by spherical harmonics convention(sp).

{\bf Note}: In convention 4, because there is no $i^l$
appears in state vectors, we have to explicitly include
$i^l$ in the partial wave expansion.

\item Momentum space expansion: completeness relation becomes
\bea
\la \vp'|p l m\ra\equiv \frac{\delta(p'-p)}{p p'}Y_{lm}(\hat{\vp}'),
\quad \la p lm |p'l'm'\ra=\frac{\delta(p'-p)}{p p'}\delta_{ll'}\delta_{mm'}
\eea
Thus, in spherical Harmonics convention, we 
can write
\bea
|\vp\ra&=&\sum_{lm} |p lm\ra_{sp} i^l \la plm|\hat{p}\ra_{sp}
\eea
and in modified spherical harmonics convention
\bea
|\vp\ra&=&\sum_{lm} |p lm\ra_{ms} \la plm|\hat{p}\ra_{ms}
\eea
We have completeness relation in (ms) convention 
\bea
\sum_{lm}\int dp p^2 |p lm\ra \la p lm|
=1
\eea

\item Configuration space
In spherical harmonics convention,
\bea
\la \vx| r l m\ra_{sp}\equiv 
\frac{\delta(x-r)}{xr} Y_{lm}(\hat{x})
\eea
In modified spherical harmonics convention,
\bea
\la \vx| r l m\ra_{ms}\equiv 
\frac{\delta(x-r)}{xr} i^l Y_{lm}(\hat{x})
\eea

\bea
& &\sum_{lm}\int dx x^2 | xlm\ra \la x lm|=1 \no
& &\la x' l' m'| x lm\ra=\frac{\delta(x'-x)}{xx'}\delta_{ll'}\delta_{mm'}
\eea
In both spherical harmonics convention and 
modified spherical harmonics convention
\bea
|\vx\ra=\sum_{lm} |x lm\ra_{sp} \la x lm|\hat{x}\ra_{sp}
=\sum_{lm} |x lm\ra_{ms} \la x lm|\hat{x}\ra_{ms}         
\eea

\item general partial wave expansion: 
For general case of many body scattering and 
non-conservation of angular momentum, let us
denote $\alpha$ as general
'angular-spin' quantum numbers. 
Let us distinguish
$|k,\alpha\ra$ with $|\alpha\ra$, where the first includes
radial wave function while the second only contains 
'angular' quantum numbers, thus is dimensionless. 

Define
\bea
Y_{\alpha}(\hat{k})&\equiv&\la \hat{k}|k \alpha\ra,\no
\la \hat{k}|\hat{k}'\ra&=&\delta^{(2)}(\hat{k}-\hat{k}')
 =\sum_{\alpha\beta}\delta_{\alpha\beta} 
  Y_{\alpha}(\hat{k})Y_{\beta}^*(\hat{k}')
\eea
For each convention
\bea
|\vk\ra&=&
\sum_\alpha |k, \alpha\ra_{sp}  
                i^\alpha \la \alpha|\hat{k}\ra_{sp}\no
&=&
\sum_\alpha |k, \alpha\ra_{ms}  
            \la \alpha|\hat{k}\ra_{ms}.
\eea
\bea
|\vx\ra&=&\sum_{\alpha} |x,\alpha\ra_{sp} 
           \la x,\alpha|\hat{x}\ra_{sp},\no
       &=&\sum_{\alpha} |x,\alpha\ra_{ms} 
           \la x,\alpha|\hat{x}\ra_{ms}
\eea
with $\la \hat{x}|x,\alpha \ra_{sp}=Y_{\alpha}(\hat{x})$
and $\la \hat{x}|x,\alpha \ra_{ms}=i^\alpha Y_{\alpha}(\hat{x})$

\item partial wave expansion of scattering amplitude
\bea
f(\vk,\vk')&=&-(2\pi)^2\mu \la \vk|V|\vk'\ra^{(+)}\no
           &=&-(2\pi)^2\mu \sum_{\alpha\beta} 
        \la k,\alpha|V| k,\beta\ra^{(+)}_{sp}
        i^{-\alpha+\beta}
        Y_{\alpha}(\hat{k})Y_{\beta}^*(\hat{k}') \no
        &=&-(2\pi)^2\mu \sum_{\alpha\beta} 
        \la k,\alpha|V| k,\beta\ra^{(+)}_{ms}
        Y_{\alpha}(\hat{k})Y_{\beta}^*(\hat{k}')  
\eea

How can we define $S_{\alpha,\beta}(k)$( $f_{\alpha\beta}(k)$
and $T_{\alpha,\beta}(k)$) from $\la \vk| S|\vk' \ra$ ?


Thus, if we define $S_{\alpha\beta}$, $f_{\alpha\beta}$ and
$T_{\alpha\beta}$ as, we have
\bea
\la \hat{k}|\hat{S}(k)|\hat{k}'\ra&\equiv&
      \sum_{\alpha\beta} i^{-\alpha+\beta}
      Y_{\alpha}(\hat{k})Y_{\beta}^*(\hat{k}') S_{\alpha\beta}(k)
      \mbox{ in sp convention}
      ,\no
      &\equiv&
      \sum_{\alpha\beta}
      Y_{\alpha}(\hat{k})Y_{\beta}^*(\hat{k}') S_{\alpha\beta}(k)
      \mbox{ in ms convention}
\eea
Similar relation holds for $T$ and $f$.

Regardless of convention,  we can write 
\begin{equation}
\boxed{
\begin{array}{ccl}
S_{\alpha\beta}(k)
&=&\delta_{\alpha\beta}-(2\pi)i \mu k 
        \la k,\alpha| V| k,\beta\ra^{(+)},\\
&=&\delta_{\alpha\beta} -(2\pi)i \mu k T_{\alpha\beta}(k),\\
&=&\delta_{\alpha\beta}+ i\frac{k}{2\pi} f_{\alpha\beta}(k)
\end{array}
}
\end{equation}
However, the meaning state vector depends on which
convention we are using and $S_{\alpha\beta}^{(ms)}$
and $S_{\alpha\beta}^{(sp)}$ is different.

the relation between potential matrix
element with $f_{\alpha\beta}$ depends on the convention,
\bea
\boxed{
\begin{array}{cl}
f_{\alpha\beta}^{(ms)}(k)
&=-(2\pi)^2\mu \la k,\alpha|V|k,\beta\ra^{(+)}_{ms} \\
f_{\alpha\beta}^{(sp)}(k)
&=-(2\pi)^2\mu \la k,\alpha|V|k,\beta\ra^{(+)}_{sp}
\end{array}
}
\eea
Thus,
\bea
S_{\alpha\beta}^{(ms)}(k)
&=&\delta_{\alpha\beta}
  -i2\pi \mu k \la k,\alpha|V|k,\beta\ra^{(+)}_{ms}\no
&=&\delta_{\alpha\beta}
  -i4 \mu k \la k,\alpha|V|k,\beta\ra^{(+)}_{ms'}\no
S_{\alpha\beta}^{(sp)}(k)&=&\delta_{\alpha\beta}
  -4 i \mu k \la k,\alpha|V|k,\beta\ra^{(+)}_{sp'} 
\eea



\item In general scattered state does not have the same
quantum numbers with asymptotic state.
Thus, let us further expands scattered state as
\bea
\la \vx |\Psi_\alpha(k)\ra^{(+)}_{sp}
&=&\sum_{\alpha'} 
\la x,\alpha'|\Psi^{(+)}_{\alpha',\alpha}(k)\ra_{sp}
Y_{\alpha'}(\hat{x}),\no
\la \vx |\Psi_\alpha(k)\ra^{(+)}_{ms}
&=&\sum_{\alpha'} 
\la x,\alpha'|\Psi^{(+)}_{\alpha',\alpha}(k)\ra_{ms}
i^{\alpha'}Y_{\alpha'}(\hat{x})
\eea
where,
\bea
\la x,\alpha'|\Psi^{(+)}_{\alpha',\alpha}(k)\ra_{sp}
\equiv \Psi^{(+)}_{\alpha',\alpha}(k,x)
\eea
is radial wave function solution 
of the scattering with 
boundary condition,
\bea
\la x \alpha'|\Psi_{\alpha',\alpha}(k)\ra^{(+)}_{sp}
&=&\sqrt{\frac{2}{\pi}}\frac{i}{2}
                  [h^{(2)}_{l'}(pr)\delta_{\alpha'\alpha}
                      -S^{J,(sp)}_{\alpha'\alpha} 
                      h^{(1)}_{l'}(pr)] 
                      \mbox{ at large r}
\eea

In modified spherical harmonics convention,
\bea
\la x \alpha'|\Psi_{\alpha'\alpha}\ra^{(+)}_{ms}
&=&\sqrt{\frac{2}{\pi}}
    \frac{i}{2}[h^{(2)}_{l'}(pr)\delta_{\alpha'\alpha}
                      -S^{J,(ms)}_{\alpha'\alpha} 
                      h^{(1)}_{l'}(pr)] 
                      \mbox{ at large r}\no
&=&\sqrt{\frac{2}{\pi}}
   i^{-\alpha'+\alpha} \frac{i}{2}
                  [h^{(2)}_{l'}(pr)\delta_{\alpha'\alpha}
                      -S^{J,(sp)}_{\alpha'\alpha} 
                      h^{(1)}_{l'}(pr)] 
                      \mbox{ at large r}\no
&=&i^{-\alpha'+\alpha}\la x \alpha'|\Psi_{\alpha'\alpha}\ra^{(+)}_{sp}
\eea
Thus, we can write, in both primed conventions,
\bea
S_{\alpha\beta}(k)=\delta_{\alpha\beta}-4i\mu k
 \sum_{\beta'}\la k, \alpha|V|\Psi^{(+)}_{\beta',\beta}(k),\beta'\ra
\eea

We may consider it as, because wave function 
$\la\vx|\vp\ra^{(+)}$ is written before any convention,
\bea
\la \vx|\vp\ra^{(+)}&=&\sum_{l}\la \hat{x}|x,\alpha\ra_{sp}
                     \la x,\alpha|p,\alpha'\ra^{(+)}_{sp}
                     \la p,\alpha'|\hat{p}\ra_{sp} i^{\alpha'}\no
                    &=&\sum_{l}\la \hat{x}|x,\alpha\ra_{ms}
                     \la x,\alpha|p,\alpha'\ra^{(+)}_{ms}
                     \la p,\alpha'|\hat{p}\ra_{ms}  
\eea
Thus, we should have a relation,
\bea
\la x,\alpha|p,\alpha'\ra^{(+)}_{sp}i^{\alpha'}
&=& i^{\alpha}\la x,\alpha|p,\alpha'\ra^{(+)}_{ms}.
\eea

\item Special case : angular momentum conserved case.

Usual definition of $f_l(E)$ is
\bea
f(E,\theta)\equiv\sum_{l}(2l+1)f_l(E)P_l(\cos\theta)
\eea
Comparing this with previous expansion,
\bea
f(\vk,\vk'=k \hat{z}) &=&\sum_{lm, l' m'} 
         Y_{lm}(\hat{k})Y^*_{l'm'}(\hat{k}')
         f_{l l'}(k)\no
    &=&\sum_{ll'} \delta_{ll'} f_{ll}(k)
           \frac{2l+1}{4\pi}P_l(\cos\theta)
\eea
Thus, $f_l(E)=\frac{f_{l,l}(k)}{4\pi}$ and $S_{ll'}(k)=\delta_{ll'}s_l(k)$,
\bea
s_l(k)=1+2ik f_l(k)
\eea

\end{itemize}

\section{spin rotation}
\bea
|m_n,m_d\ra
=\sum_{l_y,j_y,J}
  |[1\otimes(l_y \frac{1}{2})_{j_y}]_J\ra
  C_{l_y 0,\frac{1}{2}m_n}^{j_y m_n}
  C_{1m_d, j_y m_n}^{J m_n+m_d}Y_{l_y 0}(0)   
  \mbox{with }Y_{l_y 0}(0)=\sqrt{\frac{2l_y+1}{4\pi}}
\eea

\begin{itemize}
\item Equation from Rimas's note:  this is written in modified
spherical harmonics convention or its state vectors
contains $i^l$ factors in itself.
\begin{align*}
f_{m_{n}m_{d},m_{n}m_{d}}(0)  & =%
%TCIMACRO{\dsum \limits_{j_{y}j_{y}^{\prime}l_{y}l_{y}^{\prime}J}}%
%BeginExpansion
{\displaystyle\sum\limits_{j_{y}j_{y}^{\prime}l_{y}l_{y}^{\prime}J}}
%EndExpansion
\frac{\sqrt{\pi\widehat{l_{y}}}}{ip}C_{l_{y}0,\frac{1}{2}m_{n}}^{j_{y}m_{n}%
}C_{1m_{d},j_{y}m_{n}}^{Jm_{n}+m_{d}}C_{l_{y}^{\prime}0,\frac{1}{2}m_{n}%
}^{j_{y}^{\prime}m_{n}}C_{1m_{d},j_{y}^{\prime}m_{n}}^{Jm_{n}+m_{d}%
}\left\langle l_{y}^{\prime}j_{y}^{\prime}\right\vert S^{J}-1\left\vert
l_{y}j_{y}\right\rangle Y_{l_{y}^{\prime}0}(0)\\
& =%
%TCIMACRO{\dsum \limits_{j_{y}j_{y}^{\prime}l_{y}l_{y}^{\prime}J}}%
%BeginExpansion
{\displaystyle\sum\limits_{j_{y}j_{y}^{\prime}l_{y}l_{y}^{\prime}J}}
%EndExpansion
\frac{\sqrt{\widehat{l_{y}^{\prime}}\widehat{l_{y}}}}{2ip}C_{l_{y}0,\frac
{1}{2}m_{n}}^{j_{y}m_{n}}C_{1m_{d},j_{y}m_{n}}^{Jm_{n}+m_{d}}C_{l_{y}^{\prime
}0,\frac{1}{2}m_{n}}^{j_{y}^{\prime}m_{n}}C_{1m_{d},j_{y}^{\prime}m_{n}%
}^{Jm_{n}+m_{d}}\left\langle l_{y}^{\prime}j_{y}^{\prime}\right\vert
S^{J}-1\left\vert l_{y}j_{y}\right\rangle
\end{align*}

Then,
\bea
{\rm Re}\sum_{m_d}[f_{\frac{1}{2}m_d,\frac{1}{2}m_d}
            -f_{-\frac{1}{2}m_d,-\frac{1}{2}m_d}]
&=&{\rm Re}\frac{2i}{p}\left[\la 1\frac{1}{2}|S^{\frac{1}{2}}-1|0\frac{1}{2}\ra+2\la 1\frac{1}{2}|S^{\frac{3}{2}}-1|0\frac{1}{2}\ra\right]
\eea
Thus,
\bea
\frac{1}{N}\frac{d\phi}{dz}
&=&-\frac{2\pi}{p}\frac{1}{3}{\rm Re}\sum_{m_d}[f_{\frac{1}{2}m_d,\frac{1}{2}m_d}
            -f_{-\frac{1}{2}m_d,-\frac{1}{2}m_d}]\no
&=&\frac{4\pi}{3p^2}{\rm Im}\left[\la 1\frac{1}{2}|S^{\frac{1}{2}}-1|0\frac{1}{2}\ra+2\la 1\frac{1}{2}|S^{\frac{3}{2}}-1|0\frac{1}{2}\ra\right]
\eea
where basis are $|l_y j_y\ra$.

If we convert it to $|l_y S\ra$ basis, we get
\bea
\frac{1}{N}\frac{d\phi}{dz}
&=&-\frac{4\pi}{9p^2}{\rm Im}\left[
\la 1\frac{1}{2}|S^{\frac{1}{2}}-1|0\frac{1}{2}\ra
-2\sqrt{2}\la 1\frac{3}{2}|S^{\frac{1}{2}}-1|0\frac{1}{2}\ra
 \right.\no & &\left.
+4\la 1\frac{1}{2}|S^{\frac{3}{2}}-1|0\frac{3}{2}\ra
-2\sqrt{5}\la 1\frac{3}{2}|S^{\frac{3}{2}}-1|0\frac{3}{2}
\ra\right]
\eea
where basis are $|l_y S\ra$.

\item 
First the relation with S-matrix with scattering amplitude.
\bea
\la \alpha|S-1|\beta\ra=\frac{ ip}{2\pi}f_{\alpha\beta}
\eea
Then,
\bea
\la \alpha|S^J-1|\beta\ra=\frac{ip}{2\pi} f_{\alpha\beta}(p)
=-\frac{ip}{2\pi}(4\pi^2\mu)\la p,\alpha|V|p,\beta\ra^{(+)}
=-i(2\pi \mu p)\la p,\alpha|V|p,\beta\ra^{(+)}
\eea
Then,
\bea
\frac{1}{N}\frac{d\phi}{dz}
=-\frac{8\pi^2\mu}{3p}{\rm Re}\left[
 \la p,(1\frac{1}{2})\frac{1}{2}|V|p,(0\frac{1}{2})\frac{1}{2}\ra^{(+)}
 +2 \la p,(1\frac{1}{2})\frac{3}{2}|V|p,(0\frac{1}{2})\frac{3}{2}\ra^{(+)}\right]
\eea
in $|p,(l_y j_y) J\ra$ basis.
\item in $|p,(l_y {\cal S}) J\ra$ basis :
\bea
\frac{1}{N}\frac{d\phi}{dz}
&=&\frac{8\pi^2\mu}{9p}{\rm Re}\left[
 \la p,(1\frac{1}{2})\frac{1}{2}|V|p,(0\frac{1}{2})\frac{1}{2}\ra^{(+)}
 -2\sqrt{2}\la p,(1\frac{3}{2})\frac{1}{2}|V|p,(0\frac{1}{2})\frac{1}{2}\ra^{(+)}
 \right.\no & &\left. 
 +4\la p,(1\frac{1}{2})\frac{3}{2}|V|p,(0\frac{1}{2})\frac{3}{2}\ra^{(+)}
 -2\sqrt{5}\la p,(1\frac{3}{2})\frac{3}{2}|V|p,(0\frac{3}{2})\frac{3}{2}\ra^{(+)}
 \right]
\eea
in $|p,(l_y {\cal S}) J\ra$ basis.

\item 
Until now I used $|p,\alpha\ra$ to be normalized such as
$\sqrt{\frac{2}{\pi}}j_\alpha(pr)$. However, if matrix elements
were calculate in basis $|\alpha\ra\to j_\alpha(pr)$, then
the relation between S-matrix and potential becomes
(You can find the same equation at Glockle's book.)
\bea
S_{l'S,lS}
&=&\delta_{l'S,lS}-4i \mu p\sum_{l''}
                   \la {\cal Y}^{JM}_{l'S} j_{l'}|V|
                   {\cal Y}^{JM}_{l'' S} \Psi_{l''S,lS}^J\ra
\eea
where, 
\bea
\Psi_{l'S,lS}^J(r)\to -\frac{1}{2i}[h^{(2)}_{l'}(pr)\delta_{ll'}
                      -S^J_{l'l} h^{(1)}_{l'}(pr)] 
                      \mbox{ at large r}.
\eea
Note that the radial integration will involve 
$\int dr r^2 j_{l'}(pr)*V_{l'l'}*\Psi_{l'',l}(pr)$.

Thus, expression for K-matrix becomes                   
\bea
K_{l' l}\simeq -2\mu p \sum_{l''} 
               \la {\cal Y}^{JM}_{l'S} j_{l'}|V|
                   {\cal Y}^{JM}_{l'' S} \Psi_{l''S,lS}^J\ra     
\eea

If we replace $\mu\to \frac{m_N}{2}$ for 
2-nucleon scattering, you have
\bea
K_{l'l}\simeq -m_N p \sum_{l''} 
               \la {\cal Y}^{JM}_{l'S} j_{l'}|V|
                   {\cal Y}^{JM}_{l'' S} \Psi_{l''S,lS}^J\ra     
\eea 

For n-d scattering, we have $\mu=2/3 m_N$ and $q=\sqrt{3/4} p$.
\bea
K_{l'l}\simeq -2\mu p \sum_{l''} 
               \la {\cal Y}^{JM}_{l'S} j_{l'}|V|
                   {\cal Y}^{JM}_{l'' S} \Psi_{l''S,lS}^J\ra     
           =-\frac{4}{3}\sqrt{\frac{4}{3}} m_N q
            \sum_{l''} 
               \la {\cal Y}^{JM}_{l'S} j_{l'}|V|
                   {\cal Y}^{JM}_{l'' S} \Psi_{l''S,lS}^J\ra         
\eea 
Thus, 
However, if the wave function and integration was done as
$\int dy y^2 j_{l'}(qy)*V_{l'l'}*\Psi_{l'',l}(qy)$,
\bea
K_{l'l}&\simeq& -2\mu p \sum_{l''} 
               \int dr r^2 {\cal Y}^{*JM}_{l'S} j_{l'}(pr) V(r)
                   {\cal Y}^{JM}_{l'' S} \Psi_{l''S,lS}^J(pr)
              \no
              &\simeq& -m_N q \sum_{l''} \int dy y^2 
               {\cal Y}^{* JM}_{l'S} j_{l'}(qy) V(r)
               {\cal Y}^{JM}_{l'' S} \Psi_{l''S,lS}^J(qy)   
\eea

\item Matrix elements from the code calculates,
\bea
\mbox{(code result)}=\int dx x^2 dy y^2 \frac{f(x,y)}{xy} V_{pv}
                                        \frac{f(x,y)}{xy}   
\eea
with normalization,
\bea
\frac{f(x,y)}{xy}\to \frac{1}{q^{l_y}}\frac{F(x,y)}{xy} 
\to \frac{1}{q^{l_y}} j_{l_y}(q y)\phi(x)
\eea  
And because $q=\sqrt{\frac{3}{4}}p$, $y=\sqrt{\frac{4}{3}}r$,
\bea
\mbox{(code result)}&\to& 
\frac{1}{q}(\sqrt{\frac{4}{3}})^3
\int dx x^2 dr r^2 j_1( pr)\phi(x) V_{pv} j_0(pr)\phi(x)\no
&\to& (\sqrt{\frac{4}{3}})^4 \frac{1}{p}
    \la j_1(pr)|V|j_0(pr)\ra
\eea 
Thus, finally we have a relation
between the potential matrix element appears in 
Schiavilla's paper and code results,
\bea
\frac{1}{p}\la \mbox{P-wave}|V_{pv}|\mbox{S-wave}\ra
=(\frac{3}{4})^2 \mbox{(my code result)}
=(-i)(\frac{3}{4})^2 \mbox{(Rimas code result)}
\eea

\item total cross section.
According to Rimas's code We have ${\rm Re}[S]$ matrix as
$0.999747$ for $ly=0,j_y=\frac{1}{2},J=\frac{1}{2}$,
$0.974371$ for $ly=0,j_y=\frac{1}{2},J=\frac{3}{2}$.
Other matrix elements are negligible.
Total cross section can be written in terms of S-matrix as
\bea
\sigma_{tot}=\frac{2\pi}{k^2}{\rm Im}\sum_{\alpha\beta}i^{-l_\alpha+l_\beta-1}\sqrt{(2l_\alpha+1)(2l_\beta+1)}[S_{\alpha\beta}-\delta_{\alpha\beta}]
\eea
For $E_{lab}=15$ keV, $k_{lab}=0.0179 fm^{-1}$, experimental result
corresponds to $3.35$ b. {\bf Note} when we use above equation,
we have to be careful because $\alpha,\beta$ represent
all possible quantum numbers including polarizations.
Thus, unpolarized total cross section requires average 
over polarization. 
For example, 
$\frac{1}{(2j_1+1)(2j_2+1)}\sum_{m_1 m_2} M_{m_1m_2,m_1m_2}$.
If two particle were spin $\frac{1}{2}$, it will be
$\frac{1}{4}[M^{J=0}+3 M^{J=1}]$ and if two particle were
spin $1$ and $\frac{1}{2}$,
$\frac{1}{6}[2 M^{J=\frac{1}{2}}+4 M^{J=\frac{3}{2}}]$.
Thus, n-d scattering case,
\bea
\sigma_{tot}=\frac{2\pi}{k^2}(\frac{1}{3}[S^{J=1/2}_{00}-1]
+\frac{2}{3}[S^{J=3/2}_{00}-1]).
\eea 
This gives $335.548 fm^2=3.35b$. 

In other words, 
\bea
\sigma_{tot}
&=&\frac{1}{2}(\sigma_{+}+\sigma_{-})
=\frac{1}{2}\frac{4\pi}{p}{\rm Im}[f_{+}+f_{-}]\no
&=&\frac{1}{2}\frac{1}{3}\frac{1}{4\pi}2
   \frac{4\pi}{p}{\rm Im}  
   \left[
   \la 0\frac{1}{2}|f^{1/2}|0\frac{1}{2}\ra
   +\la 1\frac{1}{2}|f^{1/2}|1\frac{1}{2}\ra
    +\la 1\frac{3}{2}|f^{1/2}|1\frac{3}{2}\ra
\right. \no 
& &\left.+2\la 0\frac{3}{2}|f^{3/2}|0\frac{3}{2}\ra
     +2\la 1\frac{1}{2}|f^{3/2}|1\frac{1}{2}\ra
     +2\la 1\frac{3}{2}|f^{3/2}|1\frac{3}{2}\ra   \right]\no
&\simeq&\frac{1}{3 p}{\rm Im}  
   \left[
   \la 0\frac{1}{2}|f^{1/2}|0\frac{1}{2}\ra
   +2\la 0\frac{3}{2}|f^{3/2}|0\frac{3}{2}\ra\right]\no
   &=&\frac{2\pi}{p^2}{\rm Re}  
   \left[
   \frac{1}{3}\la 0\frac{1}{2}|R^{1/2}|0\frac{1}{2}\ra
   +\frac{2}{3}\la 0\frac{3}{2}|R^{3/2}|0\frac{3}{2}\ra\right]
\eea


\end{itemize}
\end{document}