\documentclass[10pt]{book}
\usepackage[bookmarks]{hyperref}
\usepackage{kotex} % korean tex
\usepackage[utf8]{inputenc} % set input encoding (not needed with XeLaTeX)

\usepackage{geometry} % to change the page dimensions
\geometry{letterpaper} % or letterpaper (US) or a5paper or....
% \geometry{margins=2in} % for example, change the margins 
%to 2 inches all round
% \geometry{landscape} % set up the page for landscape
%   read geometry.pdf for detailed page layout information

\usepackage{graphicx} % support the \includegraphics command and options

% \usepackage[parfill]{parskip} % Activate to begin paragraphs 
%with an empty line rather than an indent

\usepackage{booktabs} % for much better looking tables
\usepackage{array} % for better arrays (eg matrices) in maths
\usepackage{paralist} % very flexible & customisable lists 
%(eg. enumerate/itemize, etc.)
\usepackage{verbatim} % adds environment for commenting 
% out blocks of text & for better verbatim
\usepackage{subfig} % make it possible to include 
%more than one captioned figure/table in a single float
% These packages are all incorporated in the memoir class 
%to one degree or another...

%%% HEADERS & FOOTERS
\usepackage{fancyhdr} % This should be set 
% AFTER setting up the page geometry
\pagestyle{fancy} % options: empty , plain , fancy
\renewcommand{\headrulewidth}{0pt} % customise the layout...
\lhead{}\chead{}\rhead{}
\lfoot{}\cfoot{\thepage}\rfoot{}

%%% SECTION TITLE APPEARANCE
\usepackage{sectsty}
\allsectionsfont{\sffamily\mdseries\upshape} 
% (See the fntguide.pdf for font help)
% (This matches ConTeXt defaults)

%%% ToC (table of contents) APPEARANCE
\usepackage[nottoc,notlof,notlot]{tocbibind} 
% Put the bibliography in the ToC
\usepackage[titles,subfigure]{tocloft} 
% Alter the style of the Table of Contents
\renewcommand{\cftsecfont}{\rmfamily\mdseries\upshape}
\renewcommand{\cftsecpagefont}{\rmfamily\mdseries\upshape} % No bold!

\usepackage{amsmath}
\usepackage{amssymb}
\usepackage{epsfig}
\usepackage{color}
\usepackage{framed}
\usepackage{empheq}
% make possible to box equations, For example
%\begin{empheq}[box=\fbox]{align*}
%a&=b \tag{test}\\
%E&=mc^2 + \int_a^a x\, dx
%\end{empheq}


\parindent 10pt\textheight 9in\topmargin -0.4in\textwidth 6in
\oddsidemargin .25in\evensidemargin 0in
\def\bm{\boldsymbol}
\newcommand{\bea}{\begin{eqnarray}}
\newcommand{\eea}{\end{eqnarray}}
\newcommand{\be}{\begin{eqnarray}}
\newcommand{\ee}{\end{eqnarray}}
\newcommand{\no}{\nonumber \\}
\newcommand{\nnb}{\nonumber}
\newcommand{\etal}{{\it et al.}~}
\newcommand{\eg}{{\it e.g.}}
\newcommand{\ie}{{\it i.e.}}
\newcommand{\sll}[1]{#1\hspace{-0.5em}/}
\newcommand{\del}{\partial}
\def\vs{{\bm \sigma}}
\def\vh{{\bm h}}
\def\vp{{\bm p}}
\def\vq{{\bm q}}
\def\vk{{\bm k}}
\def\vl{{\bm l}}
\def\vx{{\bm x}}
\def\vy{{\bm y}}
\def\vv{{\bm v}}
\def\vr{{\bm r}}
\def\vR{{\bm R}}
\def\la{\langle}
\def\ra{\rangle}
\newcommand{\threejsymbol}[6]
{\left(\begin{tabular}{ccc} {$#1$}&{$#2$}&{$#3$}\\
                             {$#4$}&{$#5$}&{$#6$}\end{tabular}\right)}
\newcommand{\sixjsymbol}[6]
{\left\{\begin{tabular}{ccc} {$#1$}&{$#2$}&{$#3$}\\
                             {$#4$}&{$#5$}&{$#6$} \end{tabular}\right\}}


%%% The "real" document content comes below...

\title{Scattering Theory Note: Two body}
\author{Young-Ho Song}
\date{\today}
%\date{} % Activate to display a given date or no date (if empty),
         % otherwise the current date is printed 

\begin{document}
\maketitle
\tableofcontents
\newpage

\chapter{Basic Conventions}
%===============KINEMATICS==================================

\section{Kinematics}
\label{sec:1}
\begin{itemize}
\item 일반적인 two-body problem can be decomposed as 
center of mass kinetic energy and two-body relative 
motion.
\bea
\la \vr_1 \vr_2|H|\vr'_1\vr'_2\ra
=\la \vR \vr|H|\vR' \vr'\ra
=\delta^{(3)}(\vR-\vR')
 \left[
 \delta^{(3)}(\vr-\vr')\left(-\frac{\nabla^2_R}{2M_{tot}}
                       -\frac{\nabla_r^2}{2\mu}\right)
 +V(\vr,\vr')\right]                      
\eea

C.M. frame 에서,
scattering 에너지는 two-body 의 relative kinetic energy 이고,
$E=\frac{\vk^2}{2\mu}$ with reduced mass 
$\mu=\frac{m_1m_2}{m_1+m_2}$로 나타내진다. 
\footnote{
주의: $\vk_{rel}\equiv \mu v_{rel} \neq \vp_1-\vp_2$.}
\footnote{
{\bf 주의: state나 matrix element를 나타낼 때,
$|\vk\ra$, $|k, \alpha\ra$, $|\alpha\ra$ 는 각각
다른 의미를 가진다. $|\vk\ra$는 plane wave, $|k,\alpha\ra$ 는
partial wave with radial wave function, $|\alpha\ra$는 
angular part without radial wave function.}
}

In c.m. frame $E_k=\frac{\vk^2}{2\mu}$
with $\vk=\mu {\bm v}_{rel}$ relative momentum
between two cluster, $\mu$ reduced mass.

\item Non-relativistic kinematics: 

$m_1$ and $m_2$ particles in c.m. frame and
$m_2$ is rest in lab frame.
In relativistic expression,
\bea
(m_2+\sqrt{m_1^2+\vp_1^2})^2-\vp_1^2 
       =(\sqrt{m_1^2+\vp_c^2}+\sqrt{m_2^2+\vp_c^2})^2,
\eea
where $\vp_{1}$ is momentum in Lab. frame, $\vp_c$ is momentum in
C.M. frame. Solving this equation gives a relation between,
$\vp_1$ and $\vp_c$. In terms of kinetic energy,
\bea
(m_1+m_2+T^L)^2-(T^L+m_1)^2+m_1^2
=(m_1+m_2+T^C)^2,
\eea
gives the relation between $T^L$ and $T^C$.
\footnote{In case $m_1=m_2$, $T^L=2 T^C+\frac{(T^{C})^2}{2m}$}

Relative momentum $\vk_{rel}=\mu{\bf v}_{rel}$ 
is given in non-relativistic expression,
$\frac{\vk_{rel}}{\mu}=\frac{\vp_1}{m_1}$ in lab frame
and 
$\vk_{rel}=\mu(\frac{\vp_c}{m_1}-\frac{-\vp_c}{m_2})=\vp_c$
in c.m. frame, 
In non-relativistic approximation,
\bea
& &T^L=\frac{\hbar^2}{2m_1}\vp^2,\quad T^C=\frac{\hbar^2}{2\mu}\vk^2,
\quad \mu=\frac{m_1 m_2}{m_1+m_2},\no
& & \hbar\vk=\mu{\bm v}_{rel}=m_1{\bm v}_{1c}=-m_2{\bm v}_{2c},
\quad {\bm v}_{rel}={\bm v}_1-{\bm v}_2,\no 
& &\hbar{\vp}_1=m_1{\bm v}_{rel}=\frac{m_1}{\mu}\vk_{rel} ,
\no
& &
 T^L=\frac{m_1}{2}{\bm v}_{rel}^2=\frac{m_1}{\mu}T^c
\eea

Let us always use only $\mbox{fm}$ as a fundamental units.
Other quantities are always converted by using
$\hbar=c=1$ and $\hbar c=1=197 \mbox{MeV}.\mbox{fm}$.

\item 
{\bf In non-relativistic} inelastic scattering, $A(projectile)+B(target)\to C+D$,
the angle of particle C from the lab. frame and C.M. frame is related by
\bea
\tan\theta_{lab}&=&\frac{\sin\theta_{cm}}{\rho+\cos\theta_{cm}},\no
\rho&=&\left(\frac{m_A m_C}{m_B m_D}\frac{E}{Q+E}\right)^{\frac{1}{2}}
\eea
where, $Q+E_A+E_B=E_C+E_D$ in C.M. frame
with $E$ is the relative energy in the incident channel.
Q is the internal energy released in the reaction.
Q is $Q>0$
(means final states have more kinetic energy than initial) 
for exothermic(발열) reaction and
$Q<0$ for endorthermic reaction.
 
For non-relativistic and elastic scattering case, 
$Q=0$,$A=C$ and $B=D$, thus
$\rho=m_A/m_B$.
\end{itemize}

\subsection{Differential cross section between c.m. frame and lab. frame}

Basic principle is that the number of particles 
into certain solid angle does not change from reference frame.
Total cross sections, as ratios of fluxes, are not changed by Lorentz transformation.
However, differential cross section can change by Lorentz transformation.

Thus,
\bea
\left(\frac{d\sigma}{d\Omega}\right)' d\Omega'
=\left(\frac{d\sigma}{d\Omega}\right) d\Omega,\quad
\left(\frac{d\sigma}{d\Omega}\right)'=\left(\frac{d\sigma}{d\Omega}\right)
 \left(\frac{d\Omega}{d\Omega'}\right)
\eea
In non-relativistic kinematics,
\bea
\left(\frac{d\sigma}{d\Omega_1}\right)_{Lab}
&=&\left(\frac{d\sigma}{d\Omega}\right)_{cm}
    \frac{d\cos\theta_{C}}{d\cos\theta_{1L}}
  \simeq \left(\frac{d\sigma}{d\Omega}\right)_{cm} 
  \frac{(1+2\lambda \cos\theta_{cm}+\lambda^2)^{\frac{3}{2}}}
   {|1+\lambda \cos\theta_{cm}|}
\eea
where, $\lambda=m_1/m_2$.
\bea
\left(\frac{d\sigma}{d\Omega_2}\right)_{Lab}
&\simeq&4\sin(\frac{\theta_C}{2})\left(\frac{d\sigma}{d\Omega_2}\right)_{cm}
\eea

For inelastic reaction, we replace $\lambda$ by
$\rho$. 


\section{Normalization convention}
앞으로 bra-ket notation을 도입할 때는 언제나 
normalization에 주의해야한다.
\bea 
\la \vk|\vk'\ra ={\cal N}\delta^{(3)}(\vk-\vk'),\quad
\la \vx|\vx'\ra =\delta^{(3)}(\vx-\vx')
\eea 
Sakurai book, Rimas 논문, Glockle 의 책은 다음 convention을 사용한다.
\bea
\la \vx|\vk\ra&\equiv&\frac{1}{(2\pi)^\frac{3}{2}}e^{i\vk\cdot\vx}
\no
\int d\vk |\vk\ra \la \vk|&=&1, \quad \la \vk'|\vk\ra=\delta^{(3)}(\vk-\vk')
\eea

반면에, Another popular convention 
$\la \vk|\vk'\ra=(2\pi)^3\delta^3(\vk-\vk')$ 으로 정의할 경우엔 
\bea
\la \vx|\vk\ra&=&e^{i\vk\cdot\vx},\no 
\int \frac{d\vk}{(2\pi)^3} |\vk\ra \la \vk|&=&1, 
\quad \la \vk'|\vk\ra=(2\pi)^3 \delta^{3}(\vk-\vk')
\eea
이 된다. 이 노트에서는 앞의 normalization 즉, $\la \vk'|\vk\ra=\delta^3(\vk-\vk')$을
사용하기로 하자.
\footnote{ high energy 에서는 state를 4-vector로
생각하고,
\bea
\la k^\mu | k^{'\mu}\ra=(2\pi)^4(2E_k)\delta^{(4)}(k-k')
\eea
로 정하는 것이 편리하고, 이 때는 energy conservation을
$(2\pi)\delta(k^0-k^{'0})$로 뽑아내는 것이 편리하다. }

%===============FORMAL SCATTERING============================
\chapter{Formal scattering theory: In momentum space  }

\section{Formal scattering theory: In-, Out- state and Moller operator}
Scattering 은 흔히 free incoming state 로부터 
free outgoing state 로의 변화를 나타낸다.
따라서, 이 둘을 연결시켜 주는 S operator(scattering operator)를 
\bea
|\psi_{out}\ra=S|\psi_{in}\ra
\eea
을 생각할 수 있다. 

We can consider $|\psi_{in}\ra$ state such that 
time evolution of $|\psi\ra$ at $t=0$ state becomes

\bea
& &U(t)|\psi\ra \to_{t\to -\infty}  U^0(t)|\psi_{in}\ra ,\no
& &U(t)|\psi\ra \to_{t\to +\infty}  U^0(t)|\psi_{out}\ra.
\eea
where $U(t)=e^{-iH t}$ and $U_0(t)=e^{-iH_0 t}$. Thus,

M\"{o}ller operator $\Omega_{\pm}$ 를 다음과 같이 정의할 수 있다.
\bea
\Omega_{\pm}&\equiv&\lim_{t\to \mp\infty} U(t)^\dagger U_0(t),\no
|\psi\ra&=&\Omega_{+}|\psi_{in}\ra=\Omega_{-}|\psi_{out}\ra
\eea

  
\begin{equation}
\boxed{
\begin{array}{ccccc}
\mbox{in asymptote} & \rightarrow^{\Omega_+} & \mbox{actual state at t=0}
                    & \leftarrow^{\Omega_-}  & \mbox{ out asymptote}\\
|\psi_{in}\ra       & \rightarrow            & |\psi\ra 
                    &\leftarrow              & |\psi_{out}\ra\\
|\phi\ra            &\rightarrow             &|\phi+\ra
                    &                        &  \\
                    &                        & |\chi-\ra
                    & \leftarrow             & |\chi\ra\\       
\end{array}
}
\end{equation}
where, $|\phi+\ra$ means the actual state of the system at $t=0$
{\it if the in asymptote was $|\phi_{in}\ra=|\phi\ra$}.
(It does not mean state at $t\to +\infty$.)
\footnote{$\Omega_{+}|\phi\ra\equiv |\phi+\ra$,
$\Omega_{-}|\chi\ra\equiv |\chi-\ra$.}


S-matrix는  
\bea
S=\Omega^\dagger_{-}\Omega_{+}=1+R \mbox{ or } 1-R
\eea
로 정의 될 수 있고, Unitary 임을 알 수 있다. 
Probability is related with
\bea
w(\chi\leftarrow \phi)=|\la\chi|S|\phi\ra|^2.
\eea

앞으로 
$|\vk\ra$와 $|\vk\ra^{(\pm)}$는 free LS equation과 
full LS equation의 solution이다. 즉,
\bea
H_0|\vk\ra&=&E_k|\vk\ra\no
H|\vk\ra^{(\pm)}&=&(H_0+V)|\vk\ra^{(\pm)}=E_k|\vk\ra^{(\pm)}.
\eea
여기서, $\pm$는 서로 다른 boundary condition을 만족하는 해 임을 
나타낸다.

\section{LS equation for scattering}
\begin{itemize}
\item
From the Shrodinger equation for scattering wave function,
\bea
& &H_0|\vk\ra=E_k|\vk\ra,\no
& &H|\vk\ra^{(\pm)}=(H_0+V)|\vk\ra^{(\pm)} = E_k |\vk\ra^{(\pm)}
\eea
위 식은 LS eq.으로 바꿔 쓸 수 있다. 단, $\frac{1}{E-H_0}$는 
real axis 에서 잘 정의 되는 함수가 아니므로, $\pm i\epsilon$
을 도입하여,
\bea
\boxed{
|\vk\ra^{(\pm)}=|\vk\ra+\frac{1}{E-H_0\pm i\epsilon}V|\vk\ra^{(\pm)},}
\eea
이고, $\pm$는 각각 서로 다른 boundary consition을 준다.
이 식에서  
$H_0$ 는 free particle state $|\vk\ra$에 작용할 때,  $E_k=\frac{\vk^2}{2\mu}$ 이고, 
$E$ is an energy of $|\vk\ra^+$ which 
need not necessarily  be $H_0=\frac{\vk^2}{2\mu}$ 
for relative two-body. However, elastic scattering case, $E_k$ will be the same as $H_0$.

\item Green's function or resolvant : Green function( free resolvant or free propagator) is defined as
\bea
G(z)\equiv\frac{1}{z-H},\quad
G_0(z)\equiv\frac{1}{z-H_0}
\eea
This satisfies (LS equation for Green function) 
\bea
G(z)=G_0(z)+G_0(z)VG(z)
\eea
for $z=E\pm i\epsilon$. Or we can define,
\bea
G^{\pm}(E)=G(E\pm i\epsilon)
\eea
\item {\bf LS equation for scattering state} From above Green's functions,  we obtain $LS$ equation,
\bea
|\vk\ra^{(\pm)}&=&\lim_{\epsilon\to 0} 
       \pm i\epsilon[G_0(E\pm i\epsilon)
                             +G_0(E\pm i\epsilon)V G(E\pm i\epsilon)]|\vk\ra\no 
       &=&|\vk\ra+G_0(E\pm i\epsilon)V|\vk\ra^{(\pm)}
\eea
where, 
\bea
|\vk\ra&=&\lim_{\epsilon\to 0}\frac{\pm i\epsilon}{E_k\pm i\epsilon-H_0}|\vk\ra ,\quad \mbox{Because $H_0|\vk\ra=E_k|\vk\ra$,}\no 
|\vk\ra^{(\pm)}&=&\lim_{\epsilon\to 0}\frac{\pm i\epsilon}{E_k\pm i\epsilon-H}|\vk\ra=(1+\lim_{\epsilon\to 0}\frac{1}{E_k\pm i\epsilon-H} V)|\vk\ra\no
   &=&\lim_{\epsilon\to 0} (\pm i\epsilon) G(E+i\epsilon)|\vk\ra
\eea
are used. 여기서, 
$H_0|\vk\ra=E_k|\vk\ra$, $H|\vk'\ra^{(\pm)}=E_{k'}|\vk'\ra^{(\pm)}$, but 
$H|\vk\ra$ need not be $E_k|\vk\ra $임에 주의.


\item 
위 식에서는 Schrodinger eq으로 부터 시작하였지만, 
Moller operator를 이용하여, 같은 결과를 얻을 수 있다.
\bea
|\vk\ra^{(\pm)}&=&\Omega_{\pm}|\vk\ra
       =\lim_{t\to \mp\infty} U(t)^\dagger U_0(t)|\vk\ra\no 
       &=&\lim_{\epsilon\to 0}\pm\frac{\epsilon}{\hbar}
       \int^0_{\mp\infty} dt
       e^{\pm\frac{\epsilon}{\hbar}t}U(t)^\dagger U_0(t)|\vk\ra\no 
      &=&\lim_{\epsilon\to 0}\frac{\pm i\epsilon}{E_k\pm i\epsilon-H}|\vk\ra
\eea
where $E_k$ is an energy of $|\vk\ra^{(\pm)}$ state.

\item {\bf S-matrix:}  
\bea
\la \vk|\hat{S}|\vk'\ra
&=&{}^{(-)}\la\vk|\vk'\ra^{(+)}
=\la \vk|\vk'\ra^{(+)}
 +\lim_{\epsilon\to 0} \la \vk|V \frac{1}{E_k+ i\epsilon-H}|\vk'\ra^{(+)}\no
&=&\la \vk|\vk'\ra
 +\lim_{\epsilon\to 0} \la \vk|V \frac{1}{E_k+ i\epsilon-H}|\vk'\ra^{(+)}
 +\lim_{\epsilon\to 0} \la \vk|\frac{1}{E_k'+ i\epsilon-H_0}V|\vk'\ra^{(+)}\no
&=&\la \vk|\vk'\ra
 +\lim_{\epsilon\to 0}\frac{\la \vk|V |\vk'\ra^{(+)}}
                           {E_k+ i\epsilon-E_{k'}} 
 +\lim_{\epsilon\to 0}\frac{\la \vk|V|\vk'\ra^{(+)}}
                           {E_k'+ i\epsilon-E_k}   
\no
&=&\la \vk|\vk'\ra-2\pi i\delta(E_k-E_{k'})\la \vk|V|\vk'\ra^{(+)}
\eea
The final expression
\bea
\boxed{
\la \vk|\hat{S}|\vk'\ra=\la \vk|\vk'\ra-2\pi i\delta(E_k-E_{k'})\la \vk|V|\vk'\ra^{(+)}
}
\eea
is independent of convention. 

\item {\bf T-matrix:} 위 식으로 부터 on-shell T-matrix 를 
\bea
t(\vk\leftarrow\vk')\equiv \la \vk|V|\vk'\ra^{(+)}
\eea
으로 정의 할 수 있다. 하지만, convention 에 따라서
t-matrix의 정의는 달라질 수 있다. 또한 partial wave convention에 따라서도 달라질 수 있음에 주의.

보통은 S-matrix 식에서 에너지 보존 delta function을 빼내고,
\bea
\hat{S}=1-i(factor)\hat{T}
\eea
와 같은 식으로 나타낼 수도 있다. 
그러나, factor는  convention에 따라서 
달라짐에 주의. 
예를 들어, 
\bea
\la\vk|\hat{S}|\vk'\ra=\delta(E_k-E_{k'})s_{\vk,\vk'},
\quad
\la\vk|\hat{T}|\vk'\ra=\delta(E_k-E_{k'})t_{\vk,\vk'}
                 =\frac{\mu}{k}\delta(k-k')t_{\vk,\vk'}
\eea
와 같이 정의하는 것이 하나의 convention 이다.

위의 T-matrix의 정의, 
\bea
\boxed{
\la \vk'|V|\vk\ra^{(+)} \equiv \la \vk'|T|\vk\ra,\mbox{ or } 
   V|\vk\ra^{(+)}= T|\vk\ra}
\eea
와 LS equation을 이용하면\footnote{
{\bf 주의} 할 것은 $T=V+VG^{+}_0T$ 나 $V|\vk\ra^{+}=T|\vk\ra$
와 같이 쓸 때의 T는 사실 $\hat{T}$가 아니라, 
위 식에서의 $t_{\vk,\vk'}=\la \vk|t|\vk'\ra$와 같이 쓸 때의
t-matrix에 해당한다는 것이다.
}
\bea
T|\vp\ra&=&V|\psi_p\ra^{(+)}=V|\vp\ra+VG_0^+ V|\psi_p\ra^{(+)}\no
              &=&V|\vp\ra+V G_0^{(+)} T|\vp\ra
\eea 
이므로, formal하게 T-matrix의 LS eq.이 얻어진다. 
\bea
T(E)&=&V+VG_0(E) T(E) \no 
 &=&V+VG_0V+VG_0VG_0V+\dots 
\eea

또는 momentum space에서 by inserting
$\int d^3\vk |\vk\ra \la \vk|=1$,
다음과 같이 쓸 수 있다.
\footnote{
This expression depends on the convention. 

In $\la \vk|\vk'\ra=(2\pi)^3\delta^{(3)}(\vk-\vk')$ normalization,
\bea
\la\vk'|V|\vk\ra^{(+)}
=\la \vk'|V|\vk\ra
+\int \frac{d\tilde{\vk}}{(2\pi)^3}
\la \vk'|V|\tilde{\vk}\ra
\frac{1}{E_\vk-E_{\tilde{\vk}}+i\epsilon}
\la\tilde{\vk}|V|\vk\ra^{(+)}. 
\eea

이 때,
만약, $T(\vk',\vk)=\frac{1}{(2\pi)^3}\la \vk'|V|\vk\ra^{(+)}$ 로 정의할 경우,
\bea
T(\vk',\vk)=\frac{V(\vk',\vk)}{(2\pi)^3}
           +\int d^3{\tilde\vk}
           \frac{V(\vk',{\tilde\vk})}{(2\pi)^3}
           \frac{2\mu}{k^2-\tilde{k}^2+i\epsilon}
           T(\vk',\vk)
\eea
가 된다. 

}
\bea
\boxed{
\la\vk'|V|\vk\ra^{(+)}
=\la \vk'|V|\vk\ra
+\int d\tilde{\vk}
\la \vk'|V|\tilde{\vk}\ra
\frac{1}{E_\vk-E_{\tilde{\vk}}+i\epsilon}
\la\tilde{\vk}|V|\vk\ra^{(+)} 
}
\eea

\item {\bf Free Green's function:} in configuration space
\bea
\la \vx|G_0(E+i\epsilon)|\vx'\ra
&=&\int d^3 p \la \vx|\vp\ra\frac{1}{E+i\epsilon-p^2/2\mu}\la \vp|\vx'\ra\no 
&=&\frac{1}{(2\pi)^3}\int d^3 p e^{i\vp\cdot(\vx-\vx')}
\frac{1}{E+i\epsilon-p^2/{2\mu}}\no
&=&\frac{1}{(2\pi)^3}(4\pi)\int_0^\infty d p p^2 j_0(p|\vx-\vx'|)
     \frac{1}{E+i\epsilon-p^2/2\mu}\no
&=&-\frac{\mu}{2\pi\hbar^2}
     \frac{e^{i k|\vx-\vx'|}}{|\vx-\vx'|}
\eea
여기서, $\mu=\frac{m}{2}$로 two nucleon reduced mass
and $E=k^2/(2\mu)$. Final expression is independent of normalization
convention. Note another common definition of Green's function is
\bea
(\nabla^2+k^2)G^0_{\pm}(\vx,\vx')&=&\delta^{(3)}(\vx-\vx'),\no
G^0_{\pm}(\vx,\vx')&=&
\frac{\hbar^2}{2\mu}\la \vx|\frac{1}{E-H_0\pm i\epsilon}|\vx\ra 
                  =-\frac{1}{4\pi}\frac{e^{\pm ik|\vx-\vx'|}}{|\vx-\vx'|}
\eea

\item {\bf scattering amplitude:} In configuration space, LS equation become
\bea
\la\vr|\vk\ra^{(+)}
&=&\la \vr|\vk\ra+\int d^3\vr' \la \vr|G_0(E+i\epsilon)|\vr' \ra
 \la\vr'|V|\vk\ra^{(+)} \no 
&=&\la \vr|\vk\ra- \int d^3\vr'\frac{\mu}{2\pi\hbar^2}
   \frac{e^{ik|\vr'-\vr|}}{|\vr'-\vr|}\la \vr'|V|\vk\ra^{(+)}
\eea
By using
\bea
\la \vr'|G_0(E\pm i\epsilon)|\vr \ra&=&-\frac{\mu}{2\pi\hbar^2}
              \frac{e^{\pm i k|\vr'-\vr|}}{|\vr-\vr'|},\no 
|\vr-\vr'|&=& r-r' {\hat \vr}\cdot{\hat \vr'}+\dots,\no 
\lim_{\vr\to \infty}\la \vr'|G_0(E\pm i\epsilon)|\vr \ra
   &\to& -\frac{\mu}{2\pi\hbar^2}
   \frac{e^{\pm i(kr-ikr'\hat{\vr}\cdot{\hat\vr}')}}{r}
   = -\frac{e^{\pm ikr}}{r}\frac{\mu}{2\pi\hbar^2}e^{\mp i\vk' \cdot{\vr}'}
\eea
where we defined $\vk'=k\hat{\vr}$. 여기에서 Greens function에 있는 
$i\epsilon$ 의 의미가 분명해진다. $\pm i\epsilon $ gives each outgoing and
ingoing waves or gives boundary conditions.

따라서, 
\bea
\la\vr|\vk\ra^{(+)}
&\to&\la \vr|\vk\ra- \frac{\mu}{2\pi\hbar^2}\frac{e^{ikr}}{r}\la \vk'|V|\vk\ra^{(+)}
\eea

we can define scattering amplitude asymptotically
\bea
\boxed{
\psi_{out}(\vr)\to_{\vr\to \infty} \frac{1}{(2\pi)^{\frac{3}{2}}}
     \left(e^{i\vk\cdot\vr}+f(\vk',\vk)\frac{e^{ikr}}{r}\right).}
\eea
where
\bea
f(\vk',\vk)&=&-\frac{(2\pi)^{\frac{3}{2}}\mu}{2\pi\hbar^2} 
                    \int d^3\vr' e^{-i\vk'\cdot\vr'}
                    \la \vr'|V|\vk\ra^{(+)}
\eea
scattering matrix $\hat{f}$를 scattering amplitude로 부터 정의하면,
\bea
\boxed{
\la \vk'|\hat{f}|\vk\ra\equiv f(\vk',\vk)
=-\frac{(2\pi)^3\mu}{2\pi\hbar^2}\la \vk'|V|\vk\ra^{(+)}
}
\eea
으로 formal 하게 쓸 수 있다.
\footnote{ Normalization convention에 따라서,
\bea 
\hat{f}=-\frac{(2\pi)^3}{\cal N}\frac{\mu}{2\pi\hbar}\hat{V}.
\eea 
 
In  $\la \vk|\vk'\ra=(2\pi)^3\delta^{(3)}(\vk-\vk')$
normalization, $f(\vk,\vk')=-\frac{\mu}{2\pi\hbar^2}\la \vk'|V|\vk\ra^{(+)}$.
}
From the definition T-matrix, 
$\la \vk'|V|\vk\ra^{(+)}=\la \vk' |T|\vk\ra $,
즉 $V|\vk\ra^{(+)}=T|\vk\ra$, 
then we get
\bea
f(\vk',\vk) &=&-\frac{(2\pi)^3\mu}{2\pi\hbar^2}\la\vk'|V|\vk\ra^{(+)}
            =-\frac{(2\pi)^2\mu}{\hbar^2}\la\vk'|T|\vk\ra
\eea


만약, Coulomb interaction과 같은 long range interaction이 있는 경우에는 
asymptotic wave 로 plane wave 를 사용할 수 없기 때문에,
scattering amplitude를 다르게 정의 해야한다. 단, 이 때, differential cross section은 잘 정의되더라도, 
total cross ssection은 diverge 한다.


\item relation between S-, T-, R- matrix and scattering amplitude
and potential.
\bea
\hat{R}& \equiv& \hat{1}-\hat{S} ,\quad \mbox{ or sometimes } \hat{S}-1\no
\la \vk|\hat{T}|\vk'\ra &\equiv& \la \vk|V|\vk'\ra^{(+)} \no
\la \vk|\hat{S}|\vk'\ra
&=&\la \vk|\vk'\ra-2\pi i\delta(E_k-E_{k'})\la \vk|V|\vk'\ra^{(+)}
\no
&=&\la \vk|\vk'\ra-2\pi i\delta(E_k-E_{k'})\la \vk|T|\vk'\ra \no
&=&\la \vk|\vk'\ra
   +\frac{1}{2\pi\mu} i\delta(E_k-E_{k'}) f(\vk',\vk)
\eea

In $\la \vk|\vk'\ra=(2\pi)^3\delta^{(3)}(\vk-\vk')$
convention,
$$\la \vk|\hat{S}|\vk'\ra
=\la \vk|\vk'\ra-2\pi i\delta(E_k-E_{k'})\la \vk|V|\vk'\ra^{(+)}
=\la \vk|\vk'\ra+\frac{(2\pi)^2}{\mu} i\delta(E_k-E_{k'})
  f(\vk',\vk).$$

\item factoring Energy conservation : In case of elastic scattering,
     we can factor out energy conservation part and only consider 
     matrix elements for angles,
\bea
& &\la \vk|\vk'\ra={\cal N}\delta^{(3)}(\vk-\vk')
  ={\cal N}\frac{\delta(k-k')}{k k'}\delta^{(2)}(\hat{\vk}-\hat{\vk}')\no
& &\delta(E_k-E_{k'})=\frac{\mu}{k}\delta(k-k'),
  \mbox{ in case $E_k=\frac{k^2}{2\mu}$.} 
\eea
\bea
\la \vk|\hat{S}|\vk'\ra
&\equiv&  {\cal N}\frac{\delta(k-k')}{k k'}\la\hat{\vk}|\hat{S}(k)|\hat{\vk}'\ra \no
&=&{\cal N}\frac{\delta(k-k')}{k k'}\left(\delta^{(2)}(\hat{\vk}-\hat{\vk}')
   -\frac{2\pi i \mu k}{\cal N} 
   \la \vk| T|\vk'\ra \right) \no
&=&{\cal N}\frac{\delta(k-k')}{k k'}\left(\delta^{(2)}(\hat{\vk}-\hat{\vk}')
   +\frac{i k}{2\pi}  f(\vk',\vk)\right) 
\eea
따라서,
\bea
\boxed{ 
\hat{S}(k)=1-\frac{1}{\cal N}2\pi i \mu k \hat{T}(k)=1+\frac{ik}{2\pi}f(\vk',\vk)
}
\eea
와 같이 나타낼 수 있다. 단, 이 경우, $k=k'$ is already implied and
only consider angles for S-matrix.
\footnote{ 
If $\la \vk|\vk'\ra=(2\pi)^3\delta^{(3)}(\vk-\vk')$
convention used,
\bea
\hat{S}=1- i \frac{\mu k}{(2\pi)^2} \hat{T}
  =1+\frac{ik}{2\pi}f(\vk',\vk)
\eea
}
\end{itemize}

\subsection{Examples}
\subsubsection{Example: Born approximation}
Let us consider the case of scattering by weak local potential $V(\vx)$.
If the potential is weak enough we may approximate $|\psi\ra^{(+)}\simeq |\phi\ra +\dots$.
Then, by the first-order Born approximation,
\bea
f^{(1)}(\vk,\vk')=-\frac{1}{4\pi}\frac{2\mu}{\hbar^2}\int d^3 x' e^{i(\vk-\vk')\cdot\vx'} V(\vx')
\eea 
For the case of Yukawa potential,
\bea
V(r)&=&V_0\frac{e^{-m r}}{m r},\no 
f^{(1)}(\theta)&=& -\frac{2\mu V_0}{m\hbar^2 }\frac{1}{q^2+m^2},\quad
  q=|\vk-\vk'|=4k^2 \sin^2\frac{\theta}{2},\no   
|f^{(1)}(\theta)|^2&=& \left(\frac{2\mu V_0}{m\hbar^2 }\right)^2 \frac{1}{(q^2+m^2)^2}
\eea 
$m\to 0$ limit gives Rutherford scattering.

\subsubsection{Example: Eikonal approximation}
If potential $V(\vx)$ varies very slowly compared to the wave length $\lambda$(which is small),
we may approximate the wave function as semi-classical wave function, $E=\frac{\hbar^2 k^2}{2\mu}$
\bea 
\psi^{(+)}&\sim& e^{i S(\vx)/\hbar}, \no 
\frac{\hbar^2 k^2}{2\mu}&=&\frac{(\nabla S)^2}{2\mu}+V, \quad \mbox{Hamilton-Jacobi equation}. 
\eea 
With further assumption that the classical trajectory is just a straight line path, we have
\bea 
\frac{S(\vx={\bm b}+z\hat{z})}{\hbar}&=& kz+\int_{-\infty}^z d z' \left[\sqrt{k^2-\frac{2\mu}{\hbar^2} V(\sqrt{b^2+z'^2})}-k\right]\no  
      &\simeq & kz -\frac{\mu}{\hbar^2 k}\int_{-\infty}^z d z' V(\sqrt{b^2+z'^2})
\eea 
This means that the particle wave function only changes
the phase during inside the scattering region.
We may obtain scattering amplitude by putting $\psi^{(+)}\sim e^{i S(\vx)/\hbar}$
to $\la \vk'|V|\vk\ra^{(+)}\simeq \la \vk'|V|\vk\ra_{eik}$.
By using 
\bea 
\int_0^{2\pi} d\phi_b e^{-ikb\theta \cos\phi_b }=2\pi J_0(kb\theta),
\eea  
the final result
\bea 
f(\vk',\vk)&=&-ik\int_0^\infty db b J_0(kb\theta)[e^{2i\Delta(b)}-1],\no 
\Delta(b)&=& -\frac{\mu}{2k\hbar^2}\int_{-\infty}^{\infty} V(\sqrt{b^2+z^2}) dz
\eea 


\subsection{LS equation for bound state}
Bound state 의 경우는 
\bea
(H_0+V)|\psi_b\ra&=&E_b|\psi_b\ra, \quad E=E_b<0,\no
(H_0-E_b)|\psi_b\ra&=&-V|\psi_b\ra  
\eea
여기서, $H_0>0$이고, $E_b<0$ 이므로, 
homogeneous LS equation
\bea
|\psi_b\ra=\frac{1}{E_b-H_0}V|\psi_b\ra
\eea
가 얻어진다. boundstate 의 경우는 $E_b-H_0$ cannot give pole 이므로
$i\epsilon$ 을 Green function에 포함하지 않아도 되고,
free solution $|\psi\ra$,(such that $(H_0-E_b)|\phi\ra=0$), 를 
포함하지 않아도 된다는 점에 주의. 

위 식을 configuration space 에서 쓰면, $|E_b|=\frac{\kappa^2}{2\mu}$ or 
$k=i\kappa $ 일 때, 
\bea
\la \vx|\psi_b\ra=\psi_b(\vx)=-\frac{\mu}{2\pi\hbar^2}\int d^3x'
                  \frac{e^{-\kappa|\vx-\vx'|}}{|\vx-\vx'|}V(\vx')\psi_b(\vx') 
\eea
이 되어 large r 에서 exponential fall-off 형태가 됨을 알 수 있다.

\subsection{Bound-state 와 scattering state 의 관계.}
이미 보았듯이 bound state 와 scattering state가 만족하는 식은 약간 다르다.
\bea
|\Psi_b\ra&=&G_0(E_b)V|\Psi_b\ra, \quad E=E_b<0,\no
t(E)&=&V+V G_0(E+i\epsilon) t(E),\quad E>0
\eea
여기서, transition operator 의 E를 $E_b$로 continuation 시키면 어떻게 될까?
Formally, we can express t-matrix as sum of series
\bea
t(E)=(1-V G_0(E))^{-1} V
\eea
Expanding in $VG_0(E)$
\bea
t(E)&=&(1+VG_0+VG_0VG_0+\dots)V=V(1+G_0V+G_0V G_0V+\dots)
\eea
then,
\bea
t(E_b)|\Psi_b\ra=V(1+1+1+\dots)|\Psi_b\ra
\eea
bound state 에 대한  관계식으로부터 $t(E_b<0)$는 diverging 하는 것을
알 수 있다. 즉 bound state에 대해 $T(E=-E_b)$ 는 pole을 가진다.

또는 약간 다르게,
\bea
t(E)&=&[(G_0^{-1}-V)G_0]^{-1}V=G_0^{-1}\frac{1}{E-H_0-V}V\no 
     &=&G_0^{-1}\frac{1}{E-H}V=G_0^{-1} G^{-1} V
\eea
로 바꿔 쓴다음 complete set을 넣으면,
\bea
|\Psi_b\ra\la\Psi_b|+\int d^3 p |\Psi_p^{+}\ra\la\Psi_p^{+}|=1 
\eea
\bea
t(E)&=&(E-H_0)|\Psi_b\ra\frac{1}{E-E_b}\la \Psi_b|V
        +\int d^3p (E-H_0)|\vp\ra^+ \frac{1}{E-\vp^2/m}\la \vp^{+}| V \no
      &=&V|\Psi_b\ra\frac{1}{E-E_b}\la\Psi_b|V
         +\int d^3p V|\vp\ra^+ \frac{1}{E-\vp^2/m}\la \vp^{+}| V 
\eea
가 되고, $E\to E_b$ 일 때, pole을 가지는 것을 볼 수 있다.
\bea
t(E)\to V|\Psi_b\ra\frac{1}{E-E_b}\la\Psi_b|V, \quad E\to E_b
\eea

또는 , scattering wave의 asymptotic form으로 부터 $E_b=-\kappa^2/(2\mu)$,
\bea
\bar{u}(r,k)\simeq 
  [s(k) e^{ik r}- e^{-ik r}]
\to [s(\pm i\kappa) e^{\mp \kappa r}- e^{\pm \kappa r}]
\eea
인데, 여기서 물리적으로 중요한 것은 ratio between incoming and outgoing wave, 즉 S 이다. 
Bound state는 outgoing wave without incoming wave라고 생각할 수 있고,
따라서 ratio가 infinite여야한다. 이경우, $k=i\kappa$ 에서 $S(k=i\kappa)$ 
가 pole을 가지는 것이 
correct bound state solution을 줌을 알 수 있다. 단지, $S(E)$ 의 negative 
energy axis 에서의 pole은 $\pm i\kappa$ 를 모두 의미하므로, 적합하지 않고,
$S(k)$ 가 positive imaginary axis 에서 pole을 가져야 함을 알 수 있다.


%===============Partial Wave decomposition

\chapter{Partial wave decomposition}

\section{Basic relations}
In general, any function of angle between two vector $\cos\theta=\hat{\vr}_1\cdot\hat{\vr}_2$
can be expanded in terms of Legendre Polynomial and
more generally by spherical harmonics.
\begin{itemize}
\item delta function expansion is
\bea
\boxed{
\delta^{(3)}(\vr_1-\vr_2)=
 \frac{\delta(r_1-r_2)}{r_1 r_2}\delta^{(2)}(\hat{r}_1-\hat{r}_2)
 = 
 \frac{\delta(r_1-r_2)}{r_1 r_2}
 \sum_{lm} Y_{lm}(\hat{r}_1)Y^*_{lm}(\hat{r}_2) }
\eea
Or,
\bea
\la \hat{r}_1|\hat{r}_2\ra=\delta^{(2)}(\hat{r}_1-\hat{r}_2)
 =\delta(\phi_1-\phi_2)\delta(\cos\theta_1-\cos\theta_2)
=\sum_{lm} Y_{lm}(\hat{r}_1)Y^*_{lm}(\hat{r}_2)
\eea 
\item Useful relation for spherical harmonics:
\bea
\sum_{m} Y_{lm}^*(\hat{r}_1) Y_{lm}(\hat{r}_2)&=&
\frac{(2l+1)}{4\pi }P_l(\hat{r}_1\cdot\hat{r}_2),
 \quad \mbox{ addition theorem} \no   
Y_{lm}(\hat{z})&=&\sqrt{\frac{2l+1}{4\pi}}\delta_{m0}
       \quad \mbox{ if $\hat{p}=\hat{z}$},\no 
Y_{l0}(\hat{r})&=&\sqrt{\frac{2l+1}{4\pi}} P_l(\cos\theta)
       \quad \mbox{ if $m=0$}
\eea

\item  Completeness relation from spherical bessel functions,
\bea
\boxed{
\int_0^\infty x^2 j_n(a x) j_n(b x)dx
=\frac{\pi}{2}\frac{\delta(a-b)}{ab}}
\eea

\bea 
j_l(kr)=\frac{1}{2i^l}\int_{-1}^{+1} e^{ikr\cos\theta} P_l(\cos\theta)d\cos\theta 
\eea 

\end{itemize}

\section{Partial wave expansion of plane wave}
Plane wave $e^{i\vk\cdot\vr}$ in partial waves as
\bea
\boxed{e^{i\vp\cdot\vr}
=4\pi \sum_{lm} j_l(pr) i^l Y_{lm}(\hat{r})Y^*_{lm}(\hat{p})
=\sum_l j_l(pr)i^l (2l+1) P_l(\cos\theta)
\mbox{ if } \hat{p}=\hat{z} }
\eea
In $\la \vk|\vk'\ra=(2\pi)^3\delta^{(3)}(\vk-\vk')$
convention, $\la \vr|\vp\ra=e^{i\vp\cdot\vr}$.

In normalization $\la \vk|\vk'\ra=\delta^{(3)}(\vk-\vk') $,
\bea
\la \vr|\vp\ra=\frac{1}{(2\pi)^{\frac{3}{2}}}e^{i\vp\cdot\vr}
&=& \sum_{lm} \sqrt{\frac{2}{\pi}} j_l(pr) i^l Y_{lm}(\hat{r})Y^*_{lm}(\hat{p}) \no
&=& \frac{1}{(2\pi)^{\frac{3}{2}}}
\sum_{l}j_l(pr) i^l(2l+1) P_{l}(\cos\theta) 
\mbox{ if $\hat{p}=\hat{z}$}
\eea

\subsection{Bra-Ket representation in partial wave expansion}
\begin{itemize}
\item Formally previous expansion can be easily expressed by introducing 
      partial wave expansion of Bra $\la \vx|$ and Ket $|\vp\ra $states.
      However, normalization and convention of this expansion is not unique.
      
      Usual normalization convention of states are
      \bea 
      \la \vx'|\vx\ra &=&\delta^{(3)}(\vx-\vx'),\quad 
      \int d^3x |\vx\ra \la \vx| =1,\no 
      \la \vp|\vp'\ra &=&{\cal N} \delta^{(3)}(\vp-\vp'),\quad 
      \int \frac{d^3p}{{\cal N}}|\vp\ra \la \vp| =1,\no 
      \la \vx|\vp\ra &=& \sqrt{\frac{{\cal N}}{(2\pi)^3}} e^{i\vp\cdot\vx}
      \eea 
\item Position state expansion: $\la \vx|\vx'\ra=\delta^{(3)}(\vx-\vx')$ can be rewritten as
partial expansion of $|\vx\ra$, by introducing $|xlm\ra$ and $\la \hat{x}|xlm\ra $
\bea
\boxed{ 
|\vx\ra=\sum_{lm}|x lm\ra\la x lm|\hat{x}\ra}
\eea
such that 
\bea
\la x lm|x' l' m'\ra=\frac{\delta(x-x')}{xx'}\delta_{ll'}\delta_{mm'},
\quad
\sum_{lm}\int dx x^2 |x lm\ra\la x lm|=1.
\eea
We have two choices for the definition of $\la \hat{x}|xlm\ra $:
\bea 
\la\hat{x}| x lm\ra&\equiv& Y_{lm}(\hat{x}),\quad 
   \mbox{{\bf spherical harmonics convention} },\no 
\la\hat{x}| x lm\ra&\equiv& i^l Y_{lm}(\hat{x}),\quad
  \mbox{{\bf modified spherical harmonics convention} }.
\eea 

\item Momentum state expansion: Though
 angular part $\la \hat{p}|p lm\ra=Y_{lm}(\hat{p})$ is almost universal,
 definition of radial component varies according to convention. 
 Let us define expansion as 
 \bea 
 \boxed{ 
 |\vp\ra =\sum_{lm} C_{pl} |plm\ra \la plm|\hat{p}\ra  }
 \eea 
 such that,
 \bea 
  |C_{pl}|^2 \la plm|p' lm\ra ={\cal N}\frac{\delta(p-p')}{pp'}\delta_{ll'}\delta_{mm'},\quad 
  \sum_{lm} \int \frac{dp p^2}{{\cal N}}|C_{pl}|^2 |p lm\ra \la p' lm|=1.
 \eea 
 By using the partial wave expansion of plane wave,
 \bea 
 \la \vx|\vp\ra &=&\sqrt{\frac{\cal N}{(2\pi)^{3}}} e^{i\vp\cdot\vx } \no 
                &=&\sqrt{\frac{\cal N}{(2\pi)^{3}}} 4\pi \sum_{lm} j_l(pr) i^l Y_{lm}(\hat{r})Y^*_{lm}(\hat{p}) \no 
                &=&\sum_{lm,l'm'} C_{pl} \la xlm|p l'm'\ra \la \hat{x}|xlm\ra \la pl'm'| \hat{p}\ra 
 \eea 
 If we use spherical harmonics convention, $\la \hat{x}|xlm\ra=Y_{lm}(\hat{x})$,
 We have relation 
 \bea 
 C_{pl} \la xlm|p l'm'\ra=\sqrt{\frac{\cal N}{(2\pi)^{3}}} 4\pi j_{l}(pr) i^l \delta_{ll'}\delta_{mm'}
 \eea 
 On the other hand, if we use modified spherical harmonics convention, 
 $\la \hat{x}|xlm\ra=i^l Y_{lm}(\hat{x})$, 
 \bea 
  C_{pl} \la xlm|p l'm'\ra=\sqrt{\frac{\cal N}{(2\pi)^{3}}} 4\pi j_{l}(pr)\delta_{ll'}\delta_{mm'}
 \eea 
 
  We have many choices: $C_{pl}=1$, $C_{pl}=i^l$, $C_{pl}=4\pi$
  and so on. This fixes the normalization of $\la xlm|p l'm'\ra$ too.

\item Whether to include $i^l$ factor to 
the definition of $\la \hat{x}|xlm\ra$ or $\la xlm|plm\ra $ or treat them explicitly
for the expansion are convention dependent. 
Personally I prefer to use modified spherical harmonics convention with 
$\la \hat{x}|xlm\ra =i^l Y_{lm}(\hat{x})$, 
$C_{pl}=1$, 
${\cal N}=1$ , $\la xlm|plm\ra =\sqrt{\frac{2}{\pi} }j_l(px)$.

\item If we fix the normalization of $|p lm\ra $ as
      $\la x lm|p j_{lm}\ra=j_l(px)$, then, up to $i^{l}$ factors,
      \bea 
      C_{pl}=\sqrt{\frac{\cal N}{(2\pi)^3}}4\pi&\to& \sqrt{\frac{2}{\pi}}
                                     \mbox{ or } 4\pi 
      \eea 
      But, be careful that
      \bea 
      \la p j_{lm}|p' j_{lm}\ra =\frac{\pi}{2}\frac{\delta(p-p')}{pp'}
                        =\int dx x^2 j_l(px) j_l(p'x) 
      \eea 
\item expansion of momentum function: Any function depends on momentum chould be partial wave expanded.
      Common definition of partial wave component $f_l(p)$ and $f_{\alpha\beta}(p,p')$ are
      \bea 
      f(\vp)&=&\sum_{\alpha} f_{\alpha}(p) (2l+1) P_l(\cos\theta) ,\no
      f(\vp,\vp')&=&\sum_{\alpha,\beta} f_{\alpha,\beta}(p,p') Y_\alpha(\hat{p}) Y^*_\beta(\hat{p}')   
      \eea 
      However, if we represent them as matrix elements of operators $\hat{f}$, 
      the relation between $f_{\alpha}$, $f_{\alpha,\beta}$ and $\la p,\alpha|\hat{f}|p,\beta\ra $
      depends on conventions.
      \bea 
      \la \vp'|\hat{f}|\vp\ra 
      &=&\sum_{\alpha,\beta} C^*_{p'\alpha} C_{p\beta}\la \hat{p}'|p',\alpha\ra\la p',\alpha|
                           \hat{f}|p,\beta\ra \la p,\beta|\hat{p}\ra \no 
      &=&\sum_{\alpha,\beta} \la p',\alpha|
                                 \hat{f}|p,\beta\ra 
                                C^*_{p'\alpha} C_{p\beta} 
                          Y_\alpha(\hat{p}') Y^*_{\beta}(\hat{p})\no
      &=&\sum_{\alpha,\beta} \la p',\alpha|
                                 \hat{f}|p,\beta\ra 
                                C^*_{p'\alpha} C_{p\beta} 
                          \sqrt{\frac{(2l_\alpha+1)(2l_\beta+1)}{(4\pi)^2}}
                          P_\alpha(\cos\theta),\quad \mbox{if }\hat{p}=\hat{z}
      \eea 
      Uncoupled case can be considered for the case $\vp=\vp'$, $\alpha=\beta$.
      Thus, if we had $|C_{pl}|^2=1$, 
      $\frac{f_{\alpha\alpha}(p,p)}{4\pi}\leftrightarrow f_\alpha(p)$.
      
      \item For S-matrix and scattering amplitudes, usual convention is to relate them by
      \bea 
      f_{\alpha,\beta}(p)=\left(\frac{S-1}{2ip}\right)_{\alpha\beta}
      \eea 
      To be consistent with the normalization of S-matrix $S_{\alpha\beta}=\delta_{\alpha\beta}+\dots$,
      \bea 
      f(\vp, \vp')=\sum_{\alpha\beta} 4\pi
       \left(\frac{S-1}{2ip}\right)_{\alpha\beta}
      i^{-\alpha+\beta}
      Y_{\alpha}(\hat{p})Y_{\beta}^*(\hat{p}')
      \eea 
      we get,
      \bea 
      \frac{\cal N}{(2\pi)^3} 4\pi \la p \alpha |\hat{f}|p \beta \ra 
      =f_{\alpha,\beta}(p)=\left(\frac{S-1}{2ip}\right)_{\alpha\beta}
      \eea 
      and treating
      \bea 
      \hat{f}=-\frac{(2\pi)^3}{\cal N}\frac{\mu}{2\pi\hbar}\hat{V}
      \quad\to\quad 
      \left(\frac{S-1}{2ip}\right)_{\alpha\beta}= -2\mu \la j_\alpha|\hat{V}|j_\beta\ra^{(+)} 
      \eea 
      
      \item 때로는 $|p,lm\ra$ 이외에 $|E,lm\ra$ state를 정의하여 사용하기도 한다.
      $E=p^2/(2\mu)$ 인 경우에는
      \bea
      |E,lm\ra&=&\sqrt{\mu p}|p lm\ra,\quad  
      \la E' l' m'|E l m\ra=\delta(E-E')\delta_{l'l}\delta_{m'm}\no
      \la \vx| E lm\ra&=& i^l\sqrt{\frac{2}{\pi}}\sqrt{\mu p} j_l(pr)Y_{lm}(\hat{x}),\quad
      \la \vp| E lm\ra=
      \frac{1}{\sqrt{\mu p}}\delta(E_p-E) Y_{lm}(\hat{p})
      \eea
\end{itemize}
\subsubsection{Time reversal invariance and the choice of convention}
\begin{itemize} 
\item Under time reversal $\Theta$,
\bea 
\Theta\vx\Theta^{-1}=\vx, \quad \Theta\vp\Theta^{-1}=-\vp 
\eea 
From the fundamental commutation relation, $[x_i,p_j]=i\hbar\delta_{ij}$, 
\bea 
\Theta [x_i,p_j]\Theta^{-1}=[x_i,(-)p_j]
=\Theta i\hbar \delta_{ij} \Theta^{-1}=-i\hbar\delta_{ij}
\eea 
We can know that $\Theta$ is an anti-unitary operator.
And,
\bea 
\Theta {\bm J}\Theta^{-1}=-{\bm J},
\quad \mbox{from } [J_i,J_j]=i\hbar\epsilon_{ijk}J_k
\eea 
For spatial wave,
\bea 
\Theta Y_{lm}(\theta,\phi)\Theta^{-1}
&=&Y_{lm}^*(\theta,\phi)
    =(-1)^m Y_{l-m}(\theta,\phi),\no  
\Theta|l,m\ra &=& (-1)^m|l,-m\ra     
\eea 
For spinor, $\Theta=\eta U K $, with $\eta$ is an intrinsic
phase factor, $U$ is unitary, $K$ is complex conjugation operator,
\bea 
\Theta= \eta e^{-i\frac{\pi}{\hbar} J_y} K
\eea  
We may choose convention of $\eta$ such that
\bea 
\Theta |S,m\ra =(-1)^{S-m}|j,-m\ra 
\eea 

\item time-reversal 을 생각할 때는 
$\la \hat{x}| lm\ra=i^l Y_{lm}(\hat{x})$, modified spherical harmonics convention
로 정의하는 것이 편리하다. 
Then,  모든 $|j m\ra$ 상태에 대해
\bea
{\cal T}|j m\ra^{\pm}&=&(-1)^{j-m}|j -m\ra^{(\mp)},\no
{\cal T}|(j_1 j_2) j m\ra^{\pm}&=&(-1)^{j-m}|(j_1 j_2) j -m\ra^{(\mp)}
\eea
가 성립한다. $i^l$ factor가 없으면, 
\bea
{\cal T}|(l s) j m\ra^{\pm}=
(-1)^{l+j-m}|(l s) j -m\ra^{\mp}
\eea
가 되어, orbital angular momentum 에 대해서는 특별히 다르게 취급해야한다. 하지만,
어느 경우에나,
\bea
|\vx\ra&=&\sum_{lm} |x lm\ra \la x lm|\hat{x}\ra\no
|\vp\ra&=&\sum_{lm} |p lm\ra \la p lm|\hat{p}\ra
\eea
로 나타내어 지는 것은 같다. \footnote{
특별히, $\la \vp|\vp'\ra=(2\pi)^3\delta^{3}(\vp-\vp')$
and $\la x l| p l\ra=j_l(px)$ 의 convention을 사용할 경우는
expansion을
\bea
|\vx\ra=\sum_{lm}|x lm\ra \la x lm|\hat{x}\ra,\quad 
|\vp\ra=\sum_{lm} (4\pi) i^l |p lm\ra \la plm|\hat{p}\ra
\eea
임에 유의.
}
\end{itemize}
-------------------------------------------------------------------------------------------
Following pages require rewritings... 

\subsection{Partial wave decomposition convention }
\begin{itemize}

\item general partial wave expansion: 
For general case of many body scattering and 
non-conservation of angular momentum, let us
denote $\alpha$ as general
'angular-spin' quantum numbers. 
Let us distinguish
$|k,\alpha\ra$ with $|\alpha\ra$, where the first includes
radial wave function while the second only contains 
'angular' quantum numbers, thus is dimensionless. 

Define
\bea
Y_{\alpha}(\hat{k})&\equiv&\la \hat{k}|k \alpha\ra,\no
\la \hat{k}|\hat{k}'\ra&=&\delta^{(2)}(\hat{k}-\hat{k}')
 =\sum_{\alpha\beta}\delta_{\alpha\beta} 
  Y_{\alpha}(\hat{k})Y_{\beta}^*(\hat{k}')
\eea
For each convention
\bea
|\vk\ra&=&
\sum_\alpha |k, \alpha\ra_{sp}  
                i^\alpha \la \alpha|\hat{k}\ra_{sp}
                =
\sum_\alpha |k, \alpha\ra_{ms}  
            \la \alpha|\hat{k}\ra_{ms}.
\eea
\bea
|\vx\ra&=&\sum_{\alpha} |x,\alpha\ra_{sp} 
           \la x,\alpha|\hat{x}\ra_{sp}
       =\sum_{\alpha} |x,\alpha\ra_{ms} 
           \la x,\alpha|\hat{x}\ra_{ms}
\eea
with $\la \hat{x}|x,\alpha \ra_{sp}=Y_{\alpha}(\hat{x})$
and $\la \hat{x}|x,\alpha \ra_{ms}=i^\alpha Y_{\alpha}(\hat{x})$

\item partial wave expansion of scattering amplitude
\bea
f(\vk,\vk')&=&-(2\pi)^2\mu \la \vk|V|\vk'\ra^{(+)}\no
           &=&-(2\pi)^2\mu \sum_{\alpha\beta} 
        \la k,\alpha|V| k,\beta\ra^{(+)}_{sp}
        i^{-\alpha+\beta}
        Y_{\alpha}(\hat{k})Y_{\beta}^*(\hat{k}') \no
        &=&-(2\pi)^2\mu \sum_{\alpha\beta} 
        \la k,\alpha|V| k,\beta\ra^{(+)}_{ms}
        Y_{\alpha}(\hat{k})Y_{\beta}^*(\hat{k}')  
\eea
where, $i^{-\alpha+\beta}$ factor can be absorbed to the defnition of $|k,\alpha\ra $
and $|k,\beta\ra $

Thus, we can define $S_{\alpha\beta}$, $f_{\alpha\beta}$ and
$T_{\alpha\beta}$ as,
\bea
\la \hat{k}|\hat{S}(k)|\hat{k}'\ra&\equiv&
      \sum_{\alpha\beta} i^{-\alpha+\beta}
      Y_{\alpha}(\hat{k})Y_{\beta}^*(\hat{k}') S_{\alpha\beta}(k)
      \mbox{ in sp convention}
      ,\no
      &\equiv&
      \sum_{\alpha\beta}
      Y_{\alpha}(\hat{k})Y_{\beta}^*(\hat{k}') S_{\alpha\beta}(k)
      \mbox{ in ms convention}
\eea
Similar relation holds for $T$ and $f$.

Regardless of convention,  we can write 
\begin{equation}
\boxed{
\begin{array}{ccl}
S_{\alpha\beta}(k)
&=&\delta_{\alpha\beta}-(2\pi)i \mu k 
        \la k,\alpha| V| k,\beta\ra^{(+)},\\
&=&\delta_{\alpha\beta} -(2\pi)i \mu k T_{\alpha\beta}(k),\\
&=&\delta_{\alpha\beta}+ i\frac{k}{2\pi} f_{\alpha\beta}(k)
\end{array}
}
\end{equation}
However, $S_{\alpha\beta}^{(ms)}$
and $S_{\alpha\beta}^{(sp)}$ is different.
And the definition of T-matrix can be different by factor $(2\pi)^3$.

the relation between potential matrix
element with $f_{\alpha\beta}$ depends on the convention,
\bea
\boxed{
\begin{array}{cl}
f_{\alpha\beta}^{(ms)}(k)
&=-(2\pi)^2\mu \la k,\alpha|V|k,\beta\ra^{(+)}_{ms} \\
f_{\alpha\beta}^{(sp)}(k)
&=-(2\pi)^2\mu \la k,\alpha|V|k,\beta\ra^{(+)}_{sp}
\end{array}
}
\eea
Thus,
\bea
S_{\alpha\beta}^{(ms)}(k)
&=&\delta_{\alpha\beta}
  -i2\pi \mu k \la k,\alpha|V|k,\beta\ra^{(+)}_{ms}\no
&=&\delta_{\alpha\beta}
  -i4 \mu k \la k,\alpha|V|k,\beta\ra^{(+)}_{ms'}\no
S_{\alpha\beta}^{(sp)}(k)&=&\delta_{\alpha\beta}
  -4 i \mu k \la k,\alpha|V|k,\beta\ra^{(+)}_{sp'} 
\eea



\item In general scattered state does not have the same
quantum numbers with asymptotic state.
Thus, let us further expands scattered state as
\bea
\la \vx |\Psi_\alpha(k)\ra^{(+)}_{sp}
&=&\sum_{\alpha'} 
\la x,\alpha'|\Psi^{(+)}_{\alpha',\alpha}(k)\ra_{sp}
Y_{\alpha'}(\hat{x}),\no
\la \vx |\Psi_\alpha(k)\ra^{(+)}_{ms}
&=&\sum_{\alpha'} 
\la x,\alpha'|\Psi^{(+)}_{\alpha',\alpha}(k)\ra_{ms}
i^{\alpha'}Y_{\alpha'}(\hat{x})
\eea
where,
\bea
\la x,\alpha'|\Psi^{(+)}_{\alpha',\alpha}(k)\ra_{sp}
\equiv \Psi^{(+)}_{\alpha',\alpha}(k,x)
\eea
is radial wave function solution 
of the scattering with 
boundary condition,
\bea
\la x \alpha'|\Psi_{\alpha',\alpha}(k)\ra^{(+)}_{sp}
&=&\sqrt{\frac{2}{\pi}}\frac{i}{2}
                  [h^{(2)}_{l'}(pr)\delta_{\alpha'\alpha}
                      -S^{J,(sp)}_{\alpha'\alpha} 
                      h^{(1)}_{l'}(pr)] 
                      \mbox{ at large r}
\eea

In modified spherical harmonics convention,
\bea
\la x \alpha'|\Psi_{\alpha'\alpha}\ra^{(+)}_{ms}
&=&\sqrt{\frac{2}{\pi}}
    \frac{i}{2}[h^{(2)}_{l'}(pr)\delta_{\alpha'\alpha}
                      -S^{J,(ms)}_{\alpha'\alpha} 
                      h^{(1)}_{l'}(pr)] 
                      \mbox{ at large r}\no
&=&\sqrt{\frac{2}{\pi}}
   i^{-\alpha'+\alpha} \frac{i}{2}
                  [h^{(2)}_{l'}(pr)\delta_{\alpha'\alpha}
                      -S^{J,(sp)}_{\alpha'\alpha} 
                      h^{(1)}_{l'}(pr)] 
                      \mbox{ at large r}\no
&=&i^{-\alpha'+\alpha}\la x \alpha'|\Psi_{\alpha'\alpha}\ra^{(+)}_{sp}
\eea
Thus, we can write, in both primed conventions,
\bea
S_{\alpha\beta}(k)=\delta_{\alpha\beta}-4i\mu k
 \sum_{\beta'}\la k, \alpha|V|\Psi^{(+)}_{\beta',\beta}(k),\beta'\ra
\eea

We may consider it as, because wave function 
$\la\vx|\vp\ra^{(+)}$ is written before any convention,
\bea
\la \vx|\vp\ra^{(+)}&=&\sum_{l}\la \hat{x}|x,\alpha\ra_{sp}
                     \la x,\alpha|p,\alpha'\ra^{(+)}_{sp}
                     \la p,\alpha'|\hat{p}\ra_{sp} i^{\alpha'}\no
                    &=&\sum_{l}\la \hat{x}|x,\alpha\ra_{ms}
                     \la x,\alpha|p,\alpha'\ra^{(+)}_{ms}
                     \la p,\alpha'|\hat{p}\ra_{ms}  
\eea
Thus, we should have a relation,
\bea
\la x,\alpha|p,\alpha'\ra^{(+)}_{sp}i^{\alpha'}
&=& i^{\alpha}\la x,\alpha|p,\alpha'\ra^{(+)}_{ms}.
\eea



\subsection{Partial wave expressions}
We can define partial wave matrix elements 
of potential such that
\bea 
V_{\alpha\beta}(x',x )
=\la x',\alpha|\hat{V}|x, \beta \ra
=\int d\Omega'\int d\Omega 
 {\cal Y}^*_{\alpha}(\hat{x}')V(\vx',\vx)
 {\cal Y}_{\beta}(\hat{x}) 
\eea 
In momentum space,
\bea 
V_{\alpha\beta}(p',p)
=\la p',\alpha|\hat{V}|p,\beta\ra 
=\frac{2}{\pi}\int d x' dx x^{'2} \int dx x^2
 j_{l'}(p' x') V_{\alpha,\beta }(x',x)
 j_{l}(px) 
\eea 

t-matrix 에 대한 LS equation을 partial wave basis 에서 나타내면, 
\bea
t_{\alpha ',\alpha}(E,p' p)
&=&V_{\alpha',\alpha}(p'p)
+\sum_{\tilde{\alpha}}\int d\tilde{p}\tilde{p}^2
 V_{\alpha',\tilde{\alpha}}(p'\tilde{p})
\frac{1}{E-\tilde{p}^2/(2\mu)+i\epsilon}
 t_{\tilde{\alpha},\alpha}(\tilde{p},p)
\eea
where,
\bea 
[t\mbox{ or } V]_{\alpha ',\alpha}(E,p' p)\equiv
\la p',\alpha'|(T\mbox{ or } V)|p,\alpha\ra
\eea
여기서 V가 rotationally invariant 라면, 
partial wave component $t_l(p'p)$와 $V_l(p'p)$를 다음과 같이 정의할 수 있다.,
\bea
\la p' l' m'|t(E)\mbox{, or }V |p l m\ra
&=&[t_l(p'p)\mbox{, or }V_l(p'p)]\delta_{l'l}\delta_{m'm}
\eea
\bea
t_l(p'p)&=&V_l(p'p)+\int_0^\infty dp'' p^{''2} V_l(p'p'')\frac{1}{E+i\epsilon-p''^2/2\mu}t_l(p''p)
\eea

따라서 , local potential 의 partial wave representation 은 
\bea
V_l(p'p)&=&\la p' lm|V|p lm\ra
         =\int d\hat{p}' Y^*_{lm}(\hat{p}')
          \int d\hat{p} Y_{lm}(\hat{p})
          \la p' \hat{p}'|V|p\hat{p}\ra\no
           &=&\frac{2}{\pi}
           \int_0^\infty dr r^2\int_0^\infty dr' r^{'2} j_l(p'r')\frac{\delta(r-r')}{rr'}V(r) j_l(pr)\no
           &=&\frac{2}{\pi}\int_0^\infty dr r^2 j_l(p'r)V(r) j_l(pr)
           \no
\eea
이 된다. 또는 local potential 의 경우
\bea
\la \vq|V|\vq'\ra&=&\tilde{V}(|\vq-\vq'|)
                 =\frac{4\pi}{(2\pi)^3}
                 \int dr r^2 j_0(|\vq-\vq'|r) V(r)
\no
V_{l}(p'p)&=&2\pi\int_{-1}^1 dx P_l(x)\tilde{V}(\sqrt{p^2+p'^2-2pp' x})
\eea

주의할 것은 $\la plm|V|plm\ra$ 이 radial function
$\sqrt{\frac{2}{\pi}}j_l(pr)$을 포함한다는 것이다. 만약 partial
wave representation을 정의할 때, $\la j_l lm|V|j_l lm\ra$
과 같이 정의하면, 이 둘 사이엔
\bea
&\la p'lm|V|plm\ra &=\frac{2}{\pi}\la j_l(p'r) lm|V|j_l(pr) lm\ra,\mbox{ OR}\no
&\la p'lm|V|plm\ra^{(+)} &=\frac{2}{\pi}\la j_l(p'r) lm|V|\psi_l(pr) lm\ra
\eea 
인 관계가 있게 된다.\footnote{
따라서, 다음과 같이 쓸 수도 있다.
\bea
\la j_{l'}(p') l'| T| j_{l}(p) l\ra
&=&\la j_{l'}(p') l'| V| j_{l}(p) l\ra
+\frac{2}{\pi}\sum_{\tilde{l}}\int d\tilde{p}\tilde{p}^2
 \la j_{l'}(p') l'| V| j_{\tilde{l}}(\tilde{p}) \tilde{l}\ra
\frac{1}{E-\tilde{p}^2/(2\mu)+i\epsilon}
 \la j_{\tilde{l}}(\tilde{p}) \tilde{l}| T|j_{l}(p) l \ra
\eea
}

어느 operator 의 partial wave matrix element를 나타 낼 때, position space에서의 representation
이 필요한 경우에는 언제나 $i^l$ factor에 유의해야하지만, momentum space나 spin, isospin space에 작용하는  operator의 경우에는 어느 경우나 같다. 

S-matrix, T-matrix의 on-energy-shell element 는 
\bea
\la \vk|S|\vk'\ra& \equiv& 
\frac{\delta(k-k')}{k k'} \la k,\hat{k}|S| k,\hat{k}'\ra
=\frac{\delta(k-k')}{k k'} \hat{S}_{\hat{k}\hat{k}'}(k)\ra,\no
\hat{S}_{\hat{k}\hat{k}'}(k)&=&\delta_{\hat{k}\hat{k}'}
               -2\pi i \mu k \hat{t}_{\hat{k}\hat{k}'}(k)\no
            &=&\delta_{\hat{k}\hat{k}'}
               +i\frac{k}{2\pi}\hat{f}_{\hat{k}\hat{k}'}(k)   
\eea
마찬가지로 
\bea
\hat{S}_{\alpha\beta}^J(p)=
\delta_{\alpha\beta}-2\pi i \mu k t^J_{\alpha\beta}(p)
=\delta_{\alpha\beta}+i\frac{p}{2\pi} f^J_{\alpha\beta}(p)
\eea
로 나타낼 수 있다.
\footnote{
$R$-matrix is defined as $\hat{R}=\hat{1}-\hat{S}$.

In different normalization convention, we have
\bea
\la \vk|S|\vk'\ra
=(2\pi)^3\frac{\delta(k-k')}{k k'}\delta_{\hat{k}\hat{k}'}
-2\pi i\frac{\mu}{k}\delta(k-k') t_{\hat{k}\hat{k}'}(k) 
\eea
and factoring $(2\pi)^3\frac{\delta(k-k')}{k^2}$, 
and replacing $f=-\frac{\mu}{2\pi} t$, then
\bea
\hat{S}_{\hat{k}\hat{k}'}=\delta_{\hat{k}\hat{k}'}
                        +i\frac{k}{2\pi} f_{\hat{k}\hat{k}'}(k)
\eea
}
여기서, $S_{\alpha\beta}$ 와 $f_{\alpha\beta}$는 
radial wave function의 normalization까지 모두 포함한 것이다.
\bea
S^J_{\alpha\beta}=\delta_{\alpha\beta}-4i\mu q\sum_{\gamma}
\la {\cal Y}_\alpha, j_{\alpha}|V|{\cal Y}_{\gamma},\Psi_{\gamma\beta}\ra
=\delta_{\alpha\beta}-4i\mu q {\rm t}^J_{\alpha\beta}
\eea
와 같이 정의 할 수도 있다. 이 때, 
${\rm t}^J_{\alpha\beta}\equiv \sum_{\gamma}
\la {\cal Y}_\alpha, j_{\alpha}|V|{\cal Y}_{\gamma},\Psi_{\gamma\beta}\ra$ 이고,
$t^J_{\alpha\beta}=\la p,\alpha|V|p,\beta\ra^{(+)}$ 는 다르다.

\end{itemize}

\begin{itemize}
\item partial wave expansion: denoting $\alpha$ as
'angular-spin' quantum numbers and let us separate pure angular part
and radial part for convenience,
\bea
|\vk\ra&=&\sum_\alpha |k, \alpha\ra  
                \la \alpha|\hat{k}\ra
,\quad 
|\hat{k}\ra = \sum_{\alpha}|\alpha\ra \la \alpha|\hat{k}\ra,\no
\la \hat{k}|\hat{k}'\ra&=&\delta^{(2)}(\hat{k}-\hat{k}')
 =\sum_{\alpha\beta}\delta_{\alpha\beta} 
  Y_{\alpha}(\hat{k})Y_{\beta}^*(\hat{k}')
\eea
where $|k,\alpha\ra$ contains 'radial' wave function
and spherical harmonics 
thus have dimension,
while $|\alpha\ra$ only contains 'angular' part
and dimensionless.
Note that the $|k,\alpha\ra$ in this definition 
have to include $i^\alpha$ factor from
$\la \vx|k,\alpha\ra= i^\alpha f_\alpha(k,x) Y_\alpha(\hat{x})$.


\item partial wave expansion of scattering amplitude
\bea
f(\vk,\vk')&=&-(2\pi)^2\mu \la \vk|V|\vk'\ra^{(+)}\no
           &=&-(2\pi)^2\mu \sum_{\alpha\beta} 
        \la k,\alpha| V| k,\beta\ra^{(+)}
        Y_{\alpha}(\hat{k})Y_{\beta}^*(\hat{k}')   
\eea
Thus, if we define $S_{\alpha\beta}$, $f_{\alpha\beta}$ and
$T_{\alpha\beta}$ as
\bea
\la \hat{k}|
[\hat{S}\mbox{ or }\hat{T}]   |\hat{k}'\ra&\equiv&\sum_{\alpha\beta}
      Y_{\alpha}(\hat{k})Y_{\beta}^*(\hat{k}') 
      [S\mbox{ or } T]_{\alpha\beta}(k),\no
f(\vk,\vk')&\equiv&\sum_{\alpha\beta}
      Y_{\alpha}(\hat{k})Y_{\beta}^*(\hat{k}') f_{\alpha\beta}(k),\no
S_{\alpha\beta}(k)
&=&\delta_{\alpha\beta}-(2\pi)i \mu k 
        \la k,\alpha| V| k,\beta\ra^{(+)},\no
&=&\delta_{\alpha\beta} -(2\pi)i \mu k T_{\alpha\beta}(k),\no
&=&\delta_{\alpha\beta}+ i\frac{k}{2\pi} f_{\alpha\beta}(k)             
\eea

\item Special case : angular momentum conserved case.
 $\la pl|V|pl'\ra^{(+)}=t(p)\delta_{ll'}$

Usual definition of $f_l(E)$ is
\bea
f(E,\theta)\equiv\sum_{l}(2l+1)f_l(E)P_l(\cos\theta)
\eea
Comparing this with previous expansion,
\bea
f(\vk,\vk'=k \hat{z}) &=&\sum_{lm, l' m'} 
         Y_{lm}(\hat{k})Y^*_{l'm'}(\hat{k}')
         f_{l l'}(k)\no
    &=&\sum_{ll'} \delta_{ll'} f_{ll}(k)
           \frac{2l+1}{4\pi}P_l(\cos\theta)
\eea
Thus, $f_l(E)=\frac{f_{l,l}(k)}{4\pi}$ and $S_{ll'}(k)=\delta_{ll'}s_l(k)$,
\bea
s_l(k)=1+2ik f_l(k), \quad f_l(k)=\frac{s_l(k)-1}{2ik}
\eea

By using 
\bea
\int d\Omega P_l(\cos\theta)P_{l'}(\cos\theta)=\frac{4\pi}{2l+1}\delta_{ll'}
\eea
We get
\bea
\sigma_{tot}=\frac{4\pi}{k^2}\sum_l (2l+1)\sin^2\delta_l
\eea
\end{itemize}
\subsection{Integral representation of scattering amplitudes}
When angular momentum is conserved, we can write
\bea
\la \vx|\vp\ra^{(+)}&=&
\sum_{l' m'}\la x l' m' | pl'm'\ra^{(+)} \la\hat{x}|xl'm'\ra
                               \la p l'm' |\hat{p}\ra \no
&=&\sum_{l 'm '}\la x l' m' | pl'm'\ra^{(+)} i^l Y_{l' m'}(\hat{x})
                               Y^*_{l' m'}(\hat{p})   
\eea
With
\bea
\la x l' m' | pl'm'\ra^{(+)}\to \sqrt{\frac{2}{\pi}}j_l(pr)
\mbox{ free limit} 
\eea
But, if angular momentum is not conserved, then
we have to consider something like
\bea
\la \vx|\vp\ra^{(+)}
=\sum_{l'm', lm}\la x l' m'|p l m\ra^{(+)}
                 \la \hat{x}|x l' m'\ra \la p l m|\hat{p}\ra 
\eea
where
\bea
\la x l'm'|p l m\ra^{(+)} 
&=& \sqrt{\frac{2}{\pi}}\psi_{l'l}(x,p)
\no
&\to& \sqrt{\frac{2}{\pi}}j_l(pr)\delta_{l' l} 
 \mbox{ without interaction} \no
&\to& \sqrt{\frac{2}{\pi}}
      \frac{1}{2}[\delta_{l'l}h_{l'}^{(-)}(px)
                 +S_{l'l} h_{l'}^{(+)}(px)]
 \mbox{ asymptotically}                      
\eea
where, $h^{(\pm)}_l(px)\equiv j_l(px)\pm i y_l(px)$
and 
\bea
S_{l' l}(q)=\delta_{l'l}-4i\mu q\sum_{l''}
         \la j_{l'} , l'| V| l'', \psi_{l'',l}\ra  
\eea
OR we can write,
\bea
S_{\alpha\beta}(k)
&=&\delta_{\alpha\beta}-(2\pi)i \mu k 
        \la k,\alpha| V| k,\beta\ra^{(+)},\no
&=&\delta_{\alpha,\beta}-4 i \mu k \sum_{\beta'}
       \la j_{\alpha}(k),\alpha|V|\psi^{(+)}_{\beta'\beta},\beta'\ra
\eea
such that,
\bea
|k,\alpha\ra&=&\sqrt{\frac{2}{\pi}}|j_{\alpha}(k),\alpha\ra \no
|k,\beta\ra^{(+)}&=&\sqrt{\frac{2}{\pi}}
                    \sum_{\beta'}|\psi_{\beta'\beta},\beta'\ra
\eea
We may define partial-wave projected T-matrix element by
\bea
T^J_{\alpha',\alpha}&\equiv&\sum_{\beta} 
    \la j_{\alpha'},\alpha'|V|\beta,\Psi^{(+)}_{\beta\alpha}\ra\no
S_{\alpha\beta}(k)&=&\delta_{\alpha\beta}-4i\mu k T^J_{\alpha\beta}    
\eea

$\bullet$ total cross section in partial wave:
Total cross section can be written in terms of S-matrix as
\footnote{
I am not sure about $i^{-l_\alpha+l_\beta}$ factor.
Is it correct? It seems to be....
}
\bea
\sigma_{tot}=\frac{4\pi}{k}{\rm Im}\sum_{\alpha\beta}i^{-l_\alpha+l_\beta}\sqrt{(2l_\alpha+1)(2l_\beta+1)}
\left(\frac{S_{\alpha\beta}-\delta_{\alpha\beta}}{2i k}\right)
\eea
{\bf Note} when we use above equation,
we have to be careful because $\alpha,\beta$ represent
all possible quantum numbers including polarizations.
Thus, unpolarized total cross section requires average 
over polarization. 
For example, 
$\frac{1}{(2j_1+1)(2j_2+1)}\sum_{m_1 m_2} M_{m_1m_2,m_1m_2}$.
If two particle were spin $\frac{1}{2}$, it will be
$\frac{1}{4}[M^{J=0}+3 M^{J=1}]$ and if two particle were
spin $1$ and $\frac{1}{2}$,
$\frac{1}{6}[2 M^{J=\frac{1}{2}}+4 M^{J=\frac{3}{2}}]$.
Thus, n-d scattering case,
\bea
\sigma_{tot}=\frac{2\pi}{k^2}{\mathcal Im}(\frac{1}{3}[S^{J=1/2}_{00}-1]
+\frac{2}{3}[S^{J=3/2}_{00}-1]).
\eea 



\section{Cross section}
\subsection{Differential cross section, total cross section and
 Optical theorem}
\begin{itemize}
\item Asymptotic wave \footnote{
For long range scattering case, the asymptotic wave should be
distorted. Thus, we need to include logarithmic phase factor
in asymptotic form. In case of long range force,
total cross section becomes infinite, because
total number of particle incident are infinite for
plane waves and they all have to scatter.
}
\bea
\psi\to e^{ik_i z}+f(\theta)\frac{e^{ik_f r}}{r}
\eea
gives flux $j_{inc}=\frac{k_i}{\mu}$, 
$j_{scatt}=\frac{k_f}{\mu}\frac{|f(\theta)|^2}{r^2}$
into area $\Delta A=r^2 d\Omega$.\footnote{
This simply corresponds to $j=v|A e^{ikx}|^2$.
}. Here we consider the case of in-elastic scattering.
In-elastic scattering case, one of the particle changes
its internal structure(or excited), and thus $k_f\neq k_i$.

\item The differential cross section is defined as
\bea
\frac{dN}{dt}=j_i n\Delta\Omega\sigma(\theta,\phi)
\eea
with $\sigma(\theta,\phi)$ is the number of particles scattered 
per unit time per unit scattering center and per unit incident flux
into a unit solid angle. units are usually, $mb/sr$. $10\ mb=1\ fm^2$

If we only consider $n=1$ scattering center,
\bea
d\sigma
=\frac{\mbox{(\# of particles scattered)/sec}}
    {\mbox{(\# of particles incident)/area/sec}}
=\frac{|j_{scatt}| r^2 d \Omega}{|j_{inc}|}    
\eea

\item differential cross section and scattering amplitude:
Thus,
differential cross section becomes 
\footnote{
이 관계는 normalization convention 과 무관하며
단지, differntial cross section과 potential 또는 t-matrix와의
관계는 normlaization convention 에 따라 식이 달라진다.}\bea
\frac{d\sigma}{d\Omega_k}=|f(\vk,\vk')|^2
\eea

In general including in-elastic scattering, $v_i\neq v_f$, asymptotic wave
\bea
\psi^{asym}=A[e^{ikz}+f(\theta,\phi)\frac{e^{ik_f r}}{r}],
\eea
implies
\bea
\sigma(\theta,\phi)=\frac{v_f}{v_i}|f(\theta,\phi)|^2.
\eea
Sometimes the flux factor $v_f/v_i$ can be absorbed to 
the definition of scattering amplitudes, $\tilde{f}=\sqrt{v_f/v_i} f$.

\item 위 계산에서 scattering cross section을 계산할 때, scattered wave function part만을 고려하였지만,
      실제로 입자를 발견할 확율은 전체 wave function을 이용해야 구해야 한다. 
      This is usually not a problem as long as the
      incident beam is localized enough. 
      We can neglect interference between incident wave and scattered wave. 하지만, 
      forward scattering 의 경우에는 이야기가 달라진다. 이 영역에서는 incident beam과 scattered beam
      이 함께 존재하므로, interference를 반드시 고려해야한다. 다르게 말하면, forward 영역에서 
      입자를 발견되는 입자의 수는 incident beam flux보다 작아야하므로, 이 곳에서는 desturctive 
      interference가 일어나야 한다. 이 것이 Optical theorem 과 관련된다.  
      
\item Optical throem: 
Let us derive optical theorem, from the definition,
$V|\psi\ra^{(+)}=T|\vk\ra$,
\bea
{\rm Im}\la \vk|V|\psi\ra^+={\rm Im}\left[
\left({}^+\la\psi|-{}^+\la\psi|V\frac{1}{E-H_0-i\epsilon}\right)V|\psi\ra
\right]
\eea
에서 imaginary part는 $i\epsilon$ 으로 부터 온다. 따라서,
\bea
\frac{1}{E-H_0-i\epsilon}=\frac{P}{E-H_0}+i\pi\delta(E-H_0)
\eea 
를 이용하면,
\bea
{\rm Im}\la \vk|V|\psi\ra^+
&=&-\pi{}\la\psi^{(+)}|V\delta(E-H_0)V|\psi\ra^{(+)},\no
{\rm Im}\la \vk|T|\vk\ra
&=&-\pi\la\psi|V\delta(E-H_0)V|\psi\ra
   =-\pi\la\vk|T^\dagger\delta(E-H_0)T|\vk\ra
\eea
plane wave의 complete set을 넣으면,
$\la \vk|\vk'\ra=\delta^{(3)}(\vk-\vk')$ convention

\bea
{\rm Im}\la \vk|T|\vk\ra
&=&-\pi\la\vk|T^\dagger\delta(E-H_0)T|\vk\ra\no
&=&-\pi\int d^3k'\la \vk|T^\dagger|\vk'\ra 
                     \la \vk'|T|\vk\ra
                  \delta(E-\frac{\hbar^2 k'^2}{2\mu})\no 
 &=&-\pi\int d\Omega' 
    \frac{\mu k}{\hbar^2}|\la \vk'|T|\vk\ra|^2
\eea
이 되고, 이것을 scattering amplitude의 식으로 바꾸면,
$f=-(2\pi)^2\mu t$,
\bea
{\rm Im}\la \vk|f|\vk\ra
=\frac{k}{4\pi\hbar^2}\int d\Omega'
 |\la \vk'|f|\vk\ra|^2 
= \frac{k}{4\pi\hbar^2} \sigma_{tot}
\eea
따라서, 잘 알려진 optical theorem 이 만족된다.
\bea
\boxed{
{\rm Im}f(\theta=0)=\frac{k}{4\pi}\sigma_{tot}}
\eea
주의, 만약 다른 normalization convention을 사용하는 경우
$f=-\mu/(2\pi) t$ 이고, 또한 completeness relation changes
as $\int d^3 k/(2\pi)^3 |\vk\ra\la \vk|=1$ 이 된다.
따라서, optical relation 식은 convention independent이다.

\item Incoming wave and outgoing wave:
Asymptotically we may write
\bea 
j_l(kr)\to \frac{1}{2ikr}[ e^{i(kr-l\pi/2)}-e^{-i(kr-l\pi/2)}]   
          = i^{-l}\frac{1}{2ikr}[ e^{ikr}-e^{-i(kr-l\pi)} ], \quad \mbox{using } i^l=e^{i l\pi /2}
\eea 
Thus, plane wave is a sum of spherically outgoing wave $e^{ikr}/r$
and spherically incoming wave $-e^{-i(kr-l\pi)}/r$. 
From asymptotic form of wave function in potential,
\bea 
\psi^{(+)}(\vx)
\to \frac{1}{(2\pi)^{3/2}}\sum_l (2l+1) \frac{1}{2ik}\left[ 
       \left(1+2ik f_l(k) \right) \frac{e^{ikr}}{r}- \frac{e^{-i(kr-l\pi) }}{r}\right]
       P_l(\cos\theta)
\eea 
where, $f_l$ is defined through
\bea 
f(\vk',\vk)=\sum_l (2l+1) f_l P_l(\cos\theta )
\eea 
The potential scattering changes
outgoing wave as
\bea
1\to 1+2ik f_l(k)=s_l(k).
\eea 
the incoming wave is completely unaffected.

\end{itemize}
%===============NN Potential
\chapter{NN potential}
%==============================================================
\chapter{{NN Phase shifts}}
%==============================================================
\chapter{Numerical methods}
%========================APPENDIX: SUPPLEMENTS=======================================================================================
\chapter{Supplements}

\subsection{Common special functions}
\subsubsection{Coulomb functions}
\subsubsection{Spherical Bessel functions}
\subsubsection{Spherical Harmonics and Legendre Polynomial}

\end{document}
