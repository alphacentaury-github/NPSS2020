\documentclass[11pt]{article}

\usepackage{kotex} % korean tex

\usepackage[utf8]{inputenc} % set input encoding (not needed with XeLaTeX)

\usepackage{geometry} % to change the page dimensions
\geometry{letterpaper} % or letterpaper (US) or a5paper or....
% \geometry{margins=2in} % for example, change the margins to 2 inches all round
% \geometry{landscape} % set up the page for landscape
%   read geometry.pdf for detailed page layout information

\usepackage{graphicx} % support the \includegraphics command and options

% \usepackage[parfill]{parskip} % Activate to begin paragraphs with an empty line rather than an indent

\usepackage{booktabs} % for much better looking tables
\usepackage{array} % for better arrays (eg matrices) in maths
\usepackage{paralist} % very flexible & customisable lists (eg. enumerate/itemize, etc.)
\usepackage{verbatim} % adds environment for commenting out blocks of text & for better verbatim
\usepackage{subfig} % make it possible to include more than one captioned figure/table in a single float
% These packages are all incorporated in the memoir class to one degree or another...

%%% HEADERS & FOOTERS
\usepackage{fancyhdr} % This should be set AFTER setting up the page geometry
\pagestyle{fancy} % options: empty , plain , fancy
\renewcommand{\headrulewidth}{0pt} % customise the layout...
\lhead{}\chead{}\rhead{}
\lfoot{}\cfoot{\thepage}\rfoot{}

%%% SECTION TITLE APPEARANCE
\usepackage{sectsty}
\allsectionsfont{\sffamily\mdseries\upshape} % (See the fntguide.pdf for font help)
% (This matches ConTeXt defaults)

%%% ToC (table of contents) APPEARANCE
\usepackage[nottoc,notlof,notlot]{tocbibind} % Put the bibliography in the ToC
\usepackage[titles,subfigure]{tocloft} % Alter the style of the Table of Contents
\renewcommand{\cftsecfont}{\rmfamily\mdseries\upshape}
\renewcommand{\cftsecpagefont}{\rmfamily\mdseries\upshape} % No bold!

\usepackage{amsmath}
\usepackage{amssymb}
\usepackage{epsfig}
\usepackage{color}
\parindent 10pt\textheight 9in\topmargin -0.4in\textwidth 6in
\oddsidemargin .25in\evensidemargin 0in
\def\bm{\boldsymbol}
\newcommand{\bea}{\begin{eqnarray}}
\newcommand{\eea}{\end{eqnarray}}
\newcommand{\be}{\begin{eqnarray}}
\newcommand{\ee}{\end{eqnarray}}
\newcommand{\no}{\nonumber \\}
\newcommand{\nnb}{\nonumber}
\newcommand{\etal}{{\it et al.}~}
\newcommand{\eg}{{\it e.g.}}
\newcommand{\ie}{{\it i.e.}}
\newcommand{\sll}[1]{#1\hspace{-0.5em}/}
\newcommand{\del}{\partial}
 
\newcommand{\vh}{{\bm h}}
\newcommand{\vp}{{\bm p}}
\newcommand{\vq}{{\bm q}}
\newcommand{\vk}{{\bm k}}
\newcommand{\vl}{{\bm l}}
\newcommand{\vx}{{\bm x}}
\newcommand{\vy}{{\bm y}}
\newcommand{\vv}{{\bm v}}
\newcommand{\vrho}{{\bm \rho}}
\newcommand{\vr}{{\bm r}}
\newcommand{\vR}{{\bm R}}
\newcommand{\la}{\langle}
\newcommand{\ra}{\rangle}

\newcommand{\threejsymbol}[6]{\left(\begin{tabular}{ccc} {$#1$}&{$#2$}&{$#3$}\\
                             {$#4$}&{$#5$}&{$#6$}\end{tabular}\right)}
\newcommand{\sixjsymbol}[6]{\left\{\begin{tabular}{ccc} {$#1$}&{$#2$}&{$#3$}\\
                             {$#4$}&{$#5$}&{$#6$} \end{tabular}\right\}}
\newcommand{\ninejsymbol}[9]{\left\{\begin{tabular}{ccc}
                             {$#1$}&{$#2$}&{$#3$}\\
                             {$#4$}&{$#5$}&{$#6$}\\
                             {$#7$}&{$#8$}&{$#9$}\end{tabular}\right\}}

%%% The "real" document content comes below...

\title{Scattering Theory Note: Three body}
\author{Young-Ho Song}
\date{\today}
%\date{} % Activate to display a given date or no date (if empty),
         % otherwise the current date is printed 

\begin{document}
\maketitle

Faddeev equation을 공부해보자. 

\section{Jacobi coordinates }
3-body에서 center of mass motion을 제외한 relative motion을 기술하기 위해서는 
Jacobi coordinate를 도입하는 것이 편하다. In many case, we assume {\color{blue}equal mass} particles.

It is very important to fix our notation for particular meaning.

Let us first define ket state $|\vx_i \vy_i\ra_j$. This vector means
in configuration 'j', the distance vector between 
$i$ pairs is $\vx$ and i-particles distance from $i$-pair is $\vy$.
The configuration implies not just 3-particles, but interactions and 
thus channels(like two-particle makes a bound state or three-particle bound state or 
three particle break up..).

Thus, Jacobi coordinates and three particle coordinates are related 
as in configuration space, with center of mass ${\bm R}=(\vx_1+\vx_2+\vx_3)/3$
\bea
  \left(\begin{array}{c} \vx_1=\vr_2-\vr_3 
      \\ \vy_1=\vr_1-\frac{1}{2} (\vr_2+\vr_3) 
  \end{array}\right)
  ,\quad
  \left(\begin{array}{c} \vx_2=\vr_3-\vr_1 
      \\ \vy_2=\vr_2-\frac{1}{2} (\vr_3+\vr_1) 
  \end{array}\right)
  ,\quad
  \left(\begin{array}{c} \vx_3=\vr_1-\vr_2 
      \\ \vy_3=\vr_3-\frac{1}{2} (\vr_1+\vr_2) 
  \end{array}\right)
\eea
But, they are related with each other( the same configuration can be described in terms of 
each $(\vx_i,\vy_i)$ pairs. )
\bea
  \left(\begin{array}{c}
  \vx_2=-\frac{1}{2}\vx_1-\vy_1\\
  \vy_2=\frac{3}{4}\vx_1-\frac{1}{2}\vy_1
  \end{array}\right),\quad
  \left(\begin{array}{c}
  \vx_3=-\frac{1}{2}\vx_1+\vy_1\\ 
  \vy_3=-\frac{3}{4}\vx_1-\frac{1}{2}\vy_1 
  \end{array}\right)
\eea
In momentum space, with ${\bm K}=(\vk_1+\vk_2+\vk_3)$
\bea
  \left(\begin{array}{c} \vp_1=\frac{1}{2}(\vk_2-\vk_3) 
      \\ \vq_1=\frac{2}{3}[\vk_1-\frac{1}{2} (\vk_2+\vk_3)] 
  \end{array}\right)
  ,\quad
   \left(\begin{array}{c} \vp_2=\frac{1}{2}(\vk_3-\vk_1) 
      \\ \vq_2=\frac{2}{3}[\vk_2-\frac{1}{2} (\vk_3+\vk_1)] 
  \end{array}\right)
  ,\quad
   \left(\begin{array}{c} \vp_3=\frac{1}{2}(\vk_1-\vk_2) 
      \\ \vq_3=\frac{2}{3}[\vk_3-\frac{1}{2} (\vk_1+\vk_2)] 
  \end{array}\right)
\eea
Also, each Jacobi momentums are convertible,
\bea
  \left(\begin{array}{c}
  \vp_2=-\frac{1}{2}\vp_1-\frac{3}{4}\vq_1\\
  \vq_2=\vp_1-\frac{1}{2}\vq_1
  \end{array}\right),\quad
  \left(\begin{array}{c}
  \vp_3=-\frac{1}{2}\vp_1+\frac{3}{4}\vq_1\\ 
  \vq_3=-\vp_1-\frac{1}{2}\vq_1 
  \end{array}\right)
\eea

All state vectors are normalized as
\bea
& &\int d\vk_1 d \vk_2 d\vk_3|\vk_1 \vk_2\vk_3\ra\la \vk_1\vk_2\vk_3|=1 \no
& &\la \vk'_1\vk'_2\vk'_3|\vk_1\vk_2\vk_3\ra
=\delta^{(3)}(\vk'_1-\vk_1)\delta^{(3)}(\vk'_2-\vk_2)
 \delta^{(3)}(\vk'_3-\vk_3)\no
& &\int d\vp_\alpha d\vq_\alpha d{\bm K}
  |\vp_\alpha \vq_\alpha {\bm K}\ra\la \vp_\alpha \vq_\alpha {\bm K}|=1\no
& &\la \vp'_\alpha \vq'_\alpha {\bm K}'|\vp_\alpha \vq_\alpha {\bm K}\ra=\delta^{(3)}(\vp'_\alpha-\vp_\alpha) \delta^{(3)}(\vq'_\alpha-\vq_\alpha)\delta^{(3)}({\bm K}'_\alpha-{\bm K}_\alpha) 
\eea

Asymptotic channel states can be written for 
channel $\alpha$ as
\bea
|\phi_{\vq_\alpha}\ra=|\varphi_\alpha\ra|\vq_\alpha\ra
\eea
and break up channel as\footnote{
It is noted by $(+)$ because in special case, we have to consider 
pair interaction even in break up channel. }
\bea
|\phi_{\alpha}\ra^{(+)}=|\vp_\alpha\ra^{(+)}|\vq_\alpha\ra
\eea

------------------------- very confusing arguments following--------------------


{\color{blue} Three identical boson} particle의 경우에는
\footnote{We can imagine three particles in space. And name them,
$(123)$ or $(231)$ or $(312)$. This do not change actual distance 
between them. Only changes names of them. 
} 
\bea
& &|\vx \vy\ra=|\vx_1 \vy_1\ra_1=|\vx_2\vy_2\ra_2=|\vx_3 \vy_3\ra_3 \no
& &\Psi(\vx_1,\vy_1)={}_1\la \vx_1 \vy_1|\Psi\ra 
 ={}_2\la \vx_2 \vy_2|\Psi\ra={}_3\la \vx_3 \vy_3|\Psi\ra
\eea 
$|\Psi\ra$는 permutation에 대해 symmetric 해야한다. 
\bea
\Psi(\vx_1 \vy_1)&=&{}_1\la \vx_1\vy_1|\Psi\ra={}_1\la \vx_1\vy_1|P_{12}P_{23}|\Psi\ra\no 
   &=&{}_1\la \vx_1\vy_1|P_{12}P_{23}|\Psi\ra
\eea 

\bea 
& &|\vx_1 \vy_1\ra_2=|\vx_2 \vy_2\ra_3=|\vx_3 \vy_3\ra_1 \no
& &|\vx_1 \vy_1\ra_3=|\vx_2 \vy_2\ra_1=|\vx_3 \vy_3\ra_2
\eea
where, $|\vx\vy\ra$ only states that distance 
between one-pair is $\vx$ and other distance is $\vy$
without specifying particle index.


Permutation operator defined $P^+=P_{12}P_{23}$, $P^{-}=P_{23} P_{12}$
change these ket vectors. 
In above convention\footnote{{\bf NOTE:} It is very confusing
notation and seems contradicting by looking
$|\vx_2 \vy_2\ra_2 =|\vx_1 \vy_1\ra_2$. I am not sure how to 
resolve this.
}
we have
$$P^+|\vx\vy\ra_i=|\vx\vy\ra_{i+1} (?)$$,
where $i+1$ is considered in a cyclic order.
Thus
$$P^+|\vx\vy\ra_1=|\vx\vy\ra_2=|\vx_2\vy_2\ra_2=
|\vx_1 \vy_1\ra_1 (?)$$
and $\vx_1^{(1)}=\vx_2^{(2)}$, $\vy_1^{(1)}=\vy_2^{(2)}$.
 
According to Glockle,\footnote{In fact, I am not sure
this convention or relation is correct or not.}
\bea
& &P^+|\vx_i \vy_i\ra_{j} =|\vx_i \vy_i\ra_{j+1},\quad 
  \mbox{in a cyclic notation for $j+1$},\no
& &P^-|\vx_i \vy_i\ra_{j} =|\vx_i \vy_i\ra_{j-1},\quad
 \mbox{in a cyclic notation for $j-1$}
\eea
And we have relation $(P^+)^\dagger=P^-$ which make it possible to act permutation on bra state.

( For example, 
$P^+|\vx_1\vy_1\ra_2=|\vx_1\vy_1\ra_3=|\vx_2\vy_2\ra_1$)

We can define full Faddeev state vector $|\Psi\ra$
as sum of Faddeev components $|\psi\ra_1+|\psi\ra_2+|\psi\ra_3$.
This full Faddeev wave function is invariant under permutation
so that
\bea
\Psi(\vx_1 \vy_1)=\Psi(\vx_2 \vy_2)=\Psi(\vx_3 \vy_3)
={}_1\la \vx_1 \vy_1|\Psi\ra
\eea

We can think permutation acts to Faddeev component as
 $|\psi\ra_2=P^+|\psi\ra_1$, 
$|\psi\ra_3=P^-|\psi\ra_1$.
\footnote{This definition of permutation operator is the same as above defined permutation operator?}

And write projection of Faddeev component on position space as
\bea
{}_i\la \vx \vy|\psi\ra_i=\psi_i(\vx,\vy)
\eea
We will use wave function form when bra and ket are in the same
configuration.


Then first use permutation acting on bra vector
and acts on ket
\bea
{}_1\la \vx_1 \vy_1|\psi\ra_1={}_2\la \vx_2 \vy_2|\psi\ra_1
       ={}_3\la \vx_2 \vy_2|P^+|\psi\ra_1
       ={}_3\la \vx_2 \vy_2|\psi\ra_2
       ={}_2\la \vx_1 \vy_1|\psi\ra_2      
\eea
In terms of wave function, we get
\bea
\psi_1(\vx_1,\vy_1)=\psi_2(\vx_1,\vy_1)
\eea
And similar method gives also $\psi_1(\vx_1,\vy_1)=\psi_3(\vx_1,\vy_1)$.
This means that Faddeev wave function have the same functional form.

Some book use notation, 
\bea
\psi(\vx_2,\vy_2)=P^+ \psi(\vx_1,\vy_1)
\eea
This corresponds to the relation in the notation here,
\bea
{}_2\la \vx_2\vy_2|\psi\ra_2
={}_1\la \vx_1 \vy_1|P^+|\psi\ra_1
\eea

\footnote{
{\bf However}, I am confused in the equation 
\bea
{}_1\la\vx_1\vy_1|\psi_1\ra={}_2\la \vx_1\vy_1|\psi_2\ra
\eea
because 
$\vx_1$ and $\vy_1$ in left-hand side 
and right-hand side seems not the same one. Thus
How can we get the conclusion from following equation?
\bea
\psi_1(\vx_1^{(1)},\vy_1^{(1)})=\psi_2(\vx_1^{(2)},\vy_1^{(2)})
\eea


}
We can also have
\bea
{}_1\la \vx_1 \vy_1| P^+|\psi\ra_1
={}_3\la \vx_1 \vy_1|\psi\ra_1={}_1\la \vx_2 \vy_2|\psi\ra_1
={}_2\la \vx_2 \vy_2|\psi\ra_2=\psi_2(\vx_2,\vy_2).
\eea

To reduce confusion, let us use every matrix or element wave 
function in terms of configuration '1', Faddeev component 
and coordinates.

\subsection{Jacobi coordinates for unequal mass particles}
For particles with unequal mass $m_{1,2,3}$ and coordinate $\vr_1{1,2,3}$
or momentum $\vp_{1,2,3}$, Jacobi coordinate becomes,


-----------------------------------------------

\section{Problem in L-S equation for scattering with more than two particles} 
3개의 입자로 이루어진 시스템을 생각하자. CM motion을 제외하면, 
시스템 전체의 Hamiltonian은 Jacobi coordinate로 다음과 같이 나타낼 수 있다. 
\bea 
H=H_0+V_1+V_2+V_3, 
\eea 
여기서, $V_i=V_{jk}$ 는 i 를 제외한 나머지 입자 사이의 interaction을 나타낸다. 
새로운 notation $V^i=V_{ij}+V_{ik}$를 도입하면, 
\bea 
H=H_0+V_\alpha+V^\alpha=H_\alpha+V^\alpha 
\eea  
라고 쓸 수 있다. (여기서, $H_0$는 free 3-body term으로 $\frac{p_l^2}{2\mu_l}+\frac{q_l^2}{2M_l}$
으로 나타낼 수 있다. 

3개의 입자가 만들수 있는 asymptotic configuration은 
\bea 
Bd&:&(123),\quad  E_{bd}<0\no  
1 &:&1,(23),\quad E_1=\epsilon_1+\frac{q_1^2}{2M_1},\quad |\phi_1\ra =|\varphi_1\ra|\vq_1\ra \no 
2 &:&2,(31),\quad E_2=\epsilon_2+\frac{q_2^2}{2M_2},\quad |\phi_2\ra =|\varphi_2\ra|\vq_2\ra\no 
3 &:&3,(12),\quad E_3=\epsilon_3+\frac{q_3^2}{2M_3},\quad |\phi_3\ra =|\varphi_3\ra|\vq_3\ra\no 
0 &:& 1,2,3,\quad E_0=\frac{p_l^2}{2\mu_l}+\frac{q_l^2}{2M_l},
  \quad |\phi_0\ra=|\vp_l\ra |\vq_l\ra  
\eea 
으로 쓸 수 있다. (여기서, $H_i|\varphi_i\ra=\epsilon_i|\varphi_i\ra$는 실제로는 excited state일 수 있다. $\epsilon_i^{(n)}$ )
우리는 산란에 의해서, 예를 들어 특정 boundary condition의 "free"-state $|\phi_1\ra$ 가 
scattering state $|\Psi_1^{(+)}\ra$를 unique 하게 결정하기를 원한다. 
($|\Psi_1^{(+)}\ra$는 다른 channel(2,3)로의 outgoing state를 포함한다.)
Formal theory of scattering, introduced M\"{o}ller channel operator such that
\bea 
& &\lim_{t\to -\infty}||\Psi^{(+)}_\alpha(t)-\phi_\alpha(t)||=0, \no 
& &|\psi^{(+)}_\alpha(0)\ra=\lim_{t\to -\infty} e^{iHt}e^{-iH_\alpha t}|\phi_{\alpha}(0)\ra,\no 
& &\Omega_\alpha^{(+)}=\lim_{t\to -\infty} e^{iHt}e^{-iH_\alpha t}
\eea 
This can be re-written in $\epsilon$ limit,
\bea 
|\Psi_{q_\alpha}^{(+)}\ra=\lim_{\epsilon\to 0}\frac{i\epsilon}{E_{q_\alpha}+i\epsilon-H}|\phi_{q_\alpha}\ra, 
\eea 
By defining channel resolvent operators, we get identity
\bea 
G_\alpha(z)&=&\frac{1}{z-H_\alpha},\no 
G(z)&=&G_\alpha(z)+G_\alpha(z)V^\alpha G(z) \no 
    &=&G_\alpha(z)+G(z)V^\alpha G_\alpha(z)
\eea 
Unlike two-body case, we have several ways to write the identity with $\alpha$.

Applying this identity, we get, LS like equations with different $\alpha$ and $\beta$, 
\bea 
|\Psi_{q_\alpha}^{(+)}\ra = \lim_{\epsilon\to 0} \frac{i\epsilon}{E_\alpha+i\epsilon-H_\beta}|\phi_{\alpha}\ra 
 +\lim_{\epsilon\to 0}\frac{1}{E_\alpha+i\epsilon-H_\beta}V^\beta|\Psi^{(+)}_{\alpha}\ra 
\eea 

From the Lippmann's identity, (증명은 생략)
\bea 
\lim_{\epsilon\to 0} \frac{i\epsilon}{E_\alpha+i\epsilon-H_\beta}|\phi_{\alpha}\ra
 =\delta_{\alpha\beta}|\phi_{\alpha}\ra, 
\eea 
We will have three relations for $\alpha=1$,
\bea 
\label{eq:triadLS}
|\Psi_1^{(+)}\ra&=& |\phi_1\ra + \lim_{\epsilon\to 0}\frac{1}{E_1+i\epsilon-H_1}V^1|\Psi^{(+)}_{1}\ra ,\no 
|\Psi_1^{(+)}\ra&=&  \lim_{\epsilon\to 0}\frac{1}{E_1+i\epsilon-H_2}V^2|\Psi^{(+)}_{1}\ra ,\no 
|\Psi_1^{(+)}\ra&=&  \lim_{\epsilon\to 0}\frac{1}{E_1+i\epsilon-H_3}V^3|\Psi^{(+)}_{1}\ra.
\eea 
이 중에서 첫번째 식이 Lippmann-Schwinger equation이 된다. 
그러나 문제는 첫번째 식인 LS equation을 푸는 것으로는 $|\Psi^{(+)}_1\ra$의 unique solution 을 
얻을 수 없다는 것이다. 예를 들어,
위 식에서 $\alpha=2$인 경우에대해서도 비슷한 식을 얻는다.
\bea 
|\Psi_2^{(+)}\ra&=& |\phi_2\ra + \lim_{\epsilon\to 0}\frac{1}{E_2+i\epsilon-H_2}V^2|\Psi^{(+)}_{2}\ra ,\no 
|\Psi_2^{(+)}\ra&=&  \lim_{\epsilon\to 0}\frac{1}{E_2+i\epsilon-H_1}V^1|\Psi^{(+)}_{2}\ra ,\no 
|\Psi_2^{(+)}\ra&=&  \lim_{\epsilon\to 0}\frac{1}{E_2+i\epsilon-H_3}V^3|\Psi^{(+)}_{2}\ra.
\eea 
이로부터 $|\Psi_2^{(+)}\ra$ 도 $|\Psi_1^{(+)}\ra$ 에 대한 LS equation의 homogeneous part
에 대한 solution이 됨을 알 수 있다. 즉, $|\Psi_1^{(+)}-N \Psi_2^{(+)}\ra$ 도
LS equation의 해가 됨을 알 수 있다. 따라서, LS equation 만으로는 $|\Psi_1^{(+)}\ra$ 를 
unique하게 결정하지 못한다. 

반면, Eq\eqref{eq:triadLS}의 첫번째 식뿐 아니라 나머지 식까지 모두 고려하면 (즉 triad LS equations), 
$|\Psi_1^{(+)}$를 unique하게 결정할 수 있다. (triad LS 를 모두 고려하면,  
위의 예에서 $|\Psi_1^{(+)}-N \Psi_2^{(+)}\ra$가 해를 만족하지 않음을 알 수 있다.)
따라서, 3-body 의 경우에는 LS equation 만을 푸는 것이 아니라, triad LS equations을 
풀어야 한다. 

이것은 동등하게, Faddeeve equation또는 AGS equation을 푸는 것으로 바꾸어 생각할 수 있다. 
Resolvent 로부터 우리는 다음과 같은 identity를 얻을 수도 있다.
\bea 
G(z)=G_0(z)+G_0(z)VG(z)=G_0(z)+G(z)VG_0(z),\quad G_0(z)=\frac{1}{z-H_0}
\eea 
Lippmann identity를 적용하면,
\bea 
|\Psi^{(+)}_\alpha\ra =G_0 V|\Psi^{(+)}_\alpha\ra 
               = \sum_{i=1}^3 G_0 V_i|\Psi^{(+)}_\alpha\ra
               =\sum_{i=1}^3|\psi_{\alpha,i}\ra 
\eea  
여기서,  Fadeeve component를 정의하면,
\bea 
|\psi_{\alpha,i}\ra \equiv G_0 V_i|\Psi^{(+)}_\alpha\ra ,
\quad 
\eea 
Then, LS equation을 적용하면, (Use $G_0 V_\alpha G_\alpha=G_\alpha V_\alpha G_0$
and $G_0V_\beta|\phi_{\beta}\ra=|\phi_{\beta}\ra$),
\bea 
|\psi_{\alpha,\alpha}\ra&=&  G_0 V_\alpha (|\phi_{\alpha}\ra +G_\alpha V^\alpha 
|\Psi_{\alpha}^{(+)}\ra )\no 
 &=& G_0 V_\alpha |\phi_{\alpha}\ra +G_\alpha V_\alpha (G_0V_\beta+G_0 V_\gamma)|\Psi_{\alpha}^{(+)}\ra \no  
 &=& |\phi_{\alpha}\ra +G_\alpha V_\alpha (|\psi_{\alpha,\beta}\ra +|\psi_{\alpha,\gamma}\ra),
 \eea
 Thus, we get Faddeeve equation, 
\bea   
|\psi_{\alpha,\alpha}\ra=&|\phi_{\alpha}\ra + &G_\alpha V_\alpha (|\psi_{\alpha,\beta}\ra +|\psi_{\alpha,\gamma}\ra),\no 
|\psi_{\alpha,\beta}\ra =&  &G_\beta V_\beta( |\psi_{\alpha,\gamma}\ra +|\psi_{\alpha,\alpha}\ra),
 \no 
|\psi_{\alpha,\gamma}\ra =&  &G_\gamma V_\gamma( |\psi_{\alpha,\alpha}\ra +|\psi_{\alpha,\beta}\ra), 
\eea 

만약 식을 3-body space에서의 transition amplitude 에 관한 식으로 바꾸어 쓰면, AGS
(Alt-Grassberger-Sandhas) equation이 얻어진다. 
\bea 
\la \phi_{\beta}|U_{\beta\alpha}|\phi_{\alpha}\ra 
\equiv \la \phi_{\alpha} |V^\beta |\Psi_\alpha^{(+)}\ra 
\eea 

From the triad equation, $|\Psi_1^{(+)}\ra=G_2V^2|\Psi_1^{(+)}\ra=G_2 U_{21}|\phi_1\ra$,
\bea 
U_{11}|\phi_1\ra &=& V^1|\Psi_1^{(+)}\ra =(V_2+V_3)|\Psi_1^{(+)}\ra
                  = V_2 U_{21}|\phi_1\ra +V_3 U_{31}|\phi_1\ra,\no 
U_{21}|\phi_1\ra&=& V^2|\Psi_1^{(+)}\ra=(V_3+V_1)|\Psi_1^{(+)}\ra
                =V_3G_3 U_{31}|\phi_1\ra+V_1|\phi_1\ra+V_1G_1 U_{11}|\phi_1\ra,\no 
U_{31}|\phi_1\ra&=&V^3|\Psi_1^{(+)}\ra=(V_1+V_2)|\Psi_1^{(+)}\ra
                =V_1|\phi_1\ra+V_1G_1 U_{11}|\phi_1\ra
                +V_2G_2 U_{21}|\phi_1\ra
\eea 

Thus, we get AGS equation, from $V_1|\phi_1\ra=(E_1-H_0)|\phi_1\ra=G_0^{-1}|\phi_1\ra$,
\bea 
U_{\beta\alpha}=\bar{\delta}_{\alpha\beta} G_0^{-1} +
  \sum_{\gamma} \bar{\delta}_{\gamma\beta} V_\gamma G_\gamma U_{\gamma\alpha},
  \quad  \bar{\delta}_{\alpha\beta}=1-\delta_{\alpha\beta} 
\eea 
Be careful that the energy in the resolvent are given as $E_1$( $G_i=1/(E_1+i\epsilon-H_i)$)
for half on-shell U-matrix. 

When all particles are identical, we would have relations for bosons and fermions,
\bea 
P_{ij}|\Psi\ra =\pm |\Psi\ra, 
\eea 
from the definition of Faddeeve components
\bea 
|\psi_2\ra&=&G_0 V_2|\Psi\ra =G_0 V_2 (P_{12}P_{23})^\dagger P_{12}P_{23}|\Psi\ra 
             =P_{12}P_{23} G_0 V_1|\Psi\ra \no 
          &=& P_{12}P_{23}|\psi_1\ra , \no 
|\psi_3\ra&=&P_{23}P_{12}|\psi_1\ra   
\eea 
Thus,\footnote{We can verify that this satisfies the 
condition for both bosons and fermions, $P_{ij}|\Psi\ra =\pm |\Psi\ra$} 
\bea 
|\Psi\ra =(1+P^++P^{-})|\psi_1\ra 
\eea 


\section{Derivation of Faddeev equation: Skip}
3-body system의 wave function or state $|\Psi\ra$ would satisfy Schrodinger equation,
\bea
i\frac{\del}{\del t} |\Psi\ra=(H_0+V)|\Psi\ra
\eea
Energy eigenstate 라고 생긱하면, LS equation 의 kernel은 
\bea
K(z)=G(z)V=G_0(z)V+G_0(z)VG_0(z)V+\dots
\eea
으로 마지막 식은 Born series 전개로 생각할 수 있다. 그러나, 이러한 전개는 
비 실용적이며(Born series가 diverge 할 수도 있다.),
 Faddeev의 의하면, unique solution 을 주지도 못한다. Three body system의 경우 
kernel $G_0(z)V$는 $\delta$-function을 momentum space에서 가지게 되어 singular
하게 된다.

2-body 의 경우는 T-matrix가 바로 scattering amplitude 및 cross section과 연관이 있지만, 
3-body 에서의  T-matrix는 scattering cross section과 직접 연결되지 못한다.
\bea
T(z)=V+VG(z)V
\eea
여기서, 3-body T-matrix, $V=V_1+V_2+V_3$, $G(z)=1/(z-H)$.  이식을 LS equation 형식으로 바꾸면,
\bea
T(z)=V+VG_0(z)T(z)
\eea
그러나, 이 경우도 여전히 singular한 문제가 있다. 만약 bound-state가 있는 경우는 Born-series는 
올바른 bound-state pole을 주지 못하므로, re-sum을 해야한다.(즉 non-perturvative 하다.)

Faddeev는 $T(z)$를 분해했다.
\bea
T(z)&=&T^{(1)}(z)+T^{(2)}(z)+T^{(3)}(z),\no 
T^{(k)}_{ij}(z)&=&V_{ij}+V_{ij}G_0(z)T_{ij}(z)
\eea
이 식을 다시 rearrange하면, 
\bea
[1-V_{ij}G_0](z)T^{(k)}(z)=V_{ij}+V_{ij}G_0(z)[T^{(i)}(z)+T^{(j)}(z)]
\eea 
이 되고 , $T_{ij}(z)=V_{ij}+V_{ij}G_0(z) T_{ij}(z)$ 를 이용하면(즉,
$T_{ij}$는 $V_{ij}$ interaction 만이 포함된다. 또한 $T^{(k)}(z)$와 
$T_{ij}(z)$ 는 다르다. $T_{ij}(z)$는 two-body interaction 만을  포함하는 
3-body 에서의 two-body sector 인반면에 , $T^{(k)}(z)$ 는 disconneted diagram
이 포함되지 않는  3-body matrix 이다.
,
\bea
T^{(k)}(z)=T_{ij}(z)+T_{ij}(z)G_0(z)[T^{(i)}(z)+T^{(j)}(z)]
\eea
가 얻어진다.


정리하면,
In terms of Faddeev component of wave function, (anti-)symmetrized wave function
can be represented by
\bea
|\Psi\ra_{S,A}=|\Psi\ra_1+|\Psi\ra_2+|\Psi\ra_3
\eea
where $\Psi$ represent all physical quantum numbers. This implies that
all three $|\Psi\ra_i$ have the same quantum numbers and only have 
different naming for particles. And each $|\Psi\ra_{i;jk}$
is already (anti-)symmetrized for $jk$ exchange.

$\bullet$ {\bf Bound state} 의 경우:
Bound state problem 의 경우는 
\bea
|\psi^{(k)}\ra=G_0(z)T_{ij}(z)[|\psi^{(i)}\ra+|\psi^{(j)}\ra]
\eea
의 homogeneous eq.을 만족시킨다.

$\bullet$ {\bf Scattering}의 경우, 만약 particle 1 scattering on a bound state of 2 and 3,
\bea
|\psi^{(1)}\ra&=&\Phi_1+G_0(z)T_{23}(z)[|\psi^{(2)}\ra+|\psi^{(3)}\ra],\no 
|\psi^{(2)}\ra&=&G_0(z)T_{31}(z)[|\psi^{(3)}\ra+|\psi^{(1)}\ra],\no
|\psi^{(3)}\ra&=&G_0(z)T_{12}(z)[|\psi^{(1)}\ra+|\psi^{(2)}\ra]
\eea
with
\bea
T^{(k)}(z)&=&T_{ij}(z)+T_{ij}(z) G_0(z)[T^{(i)}(z)+T^{(j)}(z)],\no
T_{ij}(z)&=&V_{ij}+V_{ij}G_0(z)T_{ij}(z)
\eea
where $z=E\pm i\epsilon$ and $G_0(z)=(z-H_0)^{-1}$. 
특히 two-body t-matrix in three body space는
$$G_\alpha V_\alpha=G_0 t_\alpha$$
를 만족시킨다. 따라서, two-body potential 대신 two-body t-matrix
를 이용하여 식을 쓸 수도 있다.

 $\bullet$ {\bf  Schrodinger equation form: } bound state나
 scattering state나 모두
\bea
(E-H_0-V_{23})|\psi^{(1)}\ra&=&V_{23}[|\psi^{(2)}\ra+|\psi^{(3)}\ra],\no 
(E-H_0-V_{31})|\psi^{(2)}\ra&=&V_{31}[|\psi^{(3)}\ra+|\psi^{1)}\ra],\no 
(E-H_0-V_{12})|\psi^{(3)}\ra&=&V_{12}[|\psi^{(1)}\ra+|\psi^{(2)}\ra]
\eea
또는
\bea
(E-H_0-V_{i})|\psi^{(i)}\ra&=&V_{i}[|\psi^{(j)}\ra+|\psi^{(k)}\ra]
\eea
의 형식이 된다.




\section{Explicit form of Faddeev equation in momentum space
and configuration space: None partial wave form}
Faddeev eq.은  momentum space, configuration space 에서
LS form and Schrodinger form으로 쓸 수 있다. 또한, potential 대신, two-body t-matrix를
이용하여 식을 쓸 수도 있다.
또한
wave function에 대한 식을 푸는 대신 scattering amplitude에 대한 
식을 풀 수도 있다. 

어떤 경우나 integral equation 형식으로 쓰건, integro differential equation 으로 쓰건, 위 식들을 projection 시키기만 하면
얻어진다. partial wave state로의 projection은 
다음 section 에서  살펴보도록 하자. 

$\bullet$ {\bf Momentum space LS form:} 
momentum space에서는 t-matrix를 이용한 식이 편리하다.
\bea
|\psi_i\ra=|\phi_i\ra\delta_{i,1}+G_0 t^{(i)}
           ( |\psi_j\ra+|\psi_k\ra)
\eea
where, I set the asymptotic state is '1' configuration.
Then, projection to momentum space gives
\bea
\la \vp\vq|\psi_i\ra=\la \vp \vq|\psi_i\ra\delta_{i,1}
       +\int d\vp' d\vq' \la \vp\vq| G_0 t^{(i)}|\vp'\vq'\ra
        (\la \vp'\vq'|\psi_j\ra+\la\vp'\vq'|\psi_k\ra)     
\eea
여기서, 
\bea
\la \vp\vq|G_0(E) t^{(i)}(E)|\vp'\vq'\ra
&=&\frac{1}{E-E_{\vp\vq}+i\epsilon}
   \la \vp\vq|t^{(i)}|\vp'\vq'\ra \no
\la \vp\vq|t^{(i)}(E)|\vp'\vq'\ra
&=&\la \vq\vq|V_i|\vp'\vq'\ra
  +\la \vp\vq|V G_0 t^{(i)}(E)|\vp'\vq'\ra\no
&=&\delta^{(3)}(\vq-\vq')\la \vp|V_i|\vp'\ra
   +\int d^3 p''\la \vp|V|\vp''\ra
    \frac{1}{E-E_{\vp''\vq''}+i\epsilon}
     \la \vp'' \vq'|t^{(i)}(E)|\vp'\vq'\ra  
\eea
로 쓸 수 있고, 만약 two-body t-matrix in 2-body system 을 아는 경우에는
해를
\bea
{}_i\la \vp \vq|t^{(i)}(E)|\vp\vq\ra_i=\delta^{(3)}(\vq-\vq')
                                 \la \vp|\hat{t}(E-E_{q'})|\vp'\ra
\eea
로 쓸 수 있다.\footnote{예를 들어 하나의 covention 에서 $E_{pq}=\frac{p^2}{m}+\frac{3}{4m}q^2$이다.}
따라서, Faddeev equation은 
\bea
\la \vq\vq|\psi_i\ra=\la\vp\vq|\phi_i\ra\delta_{i,1}
   +\int d^3\vp' d^3 p'' d^3 q''
    \frac{1}{E-E_{p'q}+i\epsilon}\la \vp|\hat{t}(E-E_{q})|\vp'\ra
          \la \vp'\vq|P|\vp''\vq''\ra\la\vp''\vq''|\psi_i\ra                        
\eea
Bound state 의 경우는 위식의 free term을 없애면 된다.

$\bullet${\bf Schrodinger form in configuration space:}
Projecting on configuration space gives, 
\bea
(E-H_0)\la \vx\vy|\psi_i\ra=
\int d^3 \vx' d^3 \vy' d^3 \vx'' d^3 \vy''
\la \vx\vy|V_i|\vx'\vy'\ra
                            \la \vx'\vy'|(1+P)|\vx''\vy''\ra
                            \la \vx''\vy''|\psi_i\ra
\eea
If potential is local, we can write 
\bea
(E-H_0)\la \vx\vy|\psi_i\ra=
\int d^3 \vx' d^3 \vy' V_i(\vx)
                      \la \vx\vy|(1+P)|\vx'\vy'\ra
                      \la \vx'\vy'|\psi_i\ra
\eea

\section{Partial wave decomposition of Faddeev component}

Let us write a complete set,
\bea
\sum_\alpha \int dx x^2 \int dy y^2 |\alpha x y\ra\la \alpha x y|=1
\eea
and define,
\bea
\la \vx'\vy'|\alpha x y\ra
=\frac{\delta(x'-x)}{x'x}\frac{\delta(y'-y)}{y'y}
 {\cal Y}_{\alpha}(\hat{x}'\hat{y}')
\eea
where, ${\cal Y}_\alpha(\hat{x}\hat{y})$ can be thought as
generalized bipolar spherical Harmonics.
\footnote{
Bipolar spherical harmonics is
\bea
{\cal Y}_{l_1 l_2}^{LM}(\hat{x}\hat{y})\equiv[Y_{l_1}(\hat{x})\otimes Y_{l_2}(\hat{y)}]_{LM}
\equiv\sum_{m_1,m_2}\la L M|l_1 m_1,l_2 m_2\ra 
  Y_{l_1m_1}(\hat{x})Y_{l_2m_2}(\hat{y})
\eea
} 

Thus, configuration space representation of state,
$|\psi\ra=\sum_{\alpha}\int dx x^2 dy y^2
|\alpha x y\ra \la \alpha xy|\psi\ra$,
\bea
\la \vx \vy|\psi\ra=\sum_\alpha \la \vx\vy|\alpha x y\ra 
                    \la \alpha x y|\psi\ra
\eea
and write
\bea
\la \alpha x y|\psi\ra &=&\frac{F_\alpha(x,y)}{xy} \no
\la \vx\vy|\alpha x y\ra&=&{\cal Y}_\alpha(\hat{x}\hat{y})
=\la \alpha \hat{x}\hat{y}|\alpha\ra
\eea
Thus, one component of Faddeev Wave function can be written
\bea
\psi^{(k)}(\vx,\vy)&=&\sum_\alpha \frac{F^{(k)}_\alpha(x,y)}{x y}{\cal Y}_\alpha(\hat{x}\hat{y})
\eea
From now, let us assume that radial function $F_\alpha(x,y)$ is known.
\footnote{
In fact, implicitly there is another index $|\vp,\alpha_i\ra$ for
incoming states before scattering.
Thus, we may further decompose
\bea
\psi^{(k)}(\vx,\vy)&=&\sum_{\alpha',\alpha} 
           \frac{F^{(k)}_{\alpha',\alpha}(x,y)}{x y}
           {\cal Y}_\alpha(\hat{x}\hat{y})
           \la \alpha,\hat{p}|\vp,\alpha_i\ra
\eea
 
}
\subsection{Momentum space in LS form}
\bea
\la \alpha p q|\psi_i\ra
&=&\la \alpha pq|\phi_i\ra\delta_{i1}
  +\la \alpha p q| G_0 t^{(i)}P|\psi_i\ra 
\eea
Using 
\bea
\la \alpha p q|G_0(E)|\alpha' p' q'\ra
&=&\delta_{\alpha\alpha'}\frac{\delta(p-p')}{pp'}\frac{\delta(q-q')}{qq'}\frac{1}{E-\frac{p^2}{m}-\frac{3}{4m}q^2+i\epsilon} \no
{}_i\la \alpha p q|t^{(i)}(E)|\alpha' p' q'\ra_i
&=&\frac{\delta(q-q')}{qq'}
   \delta_{\alpha_y \alpha'_y}
   \hat{t}_{\alpha_x,\alpha'_x}(p p',E-\frac{3}{4m}q^2)
\eea
Thus,
\bea
\la \alpha p q|\psi_i\ra
&=&\la \alpha pq|\phi_i\ra\delta_{i1}
  +\frac{1}{E-\frac{p^2}{m}-\frac{3}{4m}q^2+i\epsilon}
  \sum_{\alpha'}\int dp' p^{'2} \delta_{\alpha_y\alpha'_y}
     \hat{t}_{\alpha_x,\alpha'_x}(p p',E-\frac{3}{4m}q^2)\no
  & &\qquad\qquad\times 
  \sum_{\alpha''}\int dp'' p^{''2}\int dq'' q^{''2}    
  \la \alpha' p' q'| P|\alpha'' p'' q''\ra
  \la\alpha'' p'' q''|\psi_i\ra 
\eea
\subsection{Configuration space in Schrodinger form}
Projecting $\la \alpha xy|$,
\bea
\la \alpha xy|E-H_0-V_i|\psi^i\ra=\la \alpha xy|V_i P|\psi_i\ra \no
\eea
If we let 
$\la xy\alpha |\psi^i\ra\equiv \frac{F^{(i)}_\alpha(x,y)}{xy}$ 
and if potential does not contain any derivatives,
\bea
\la xy\alpha|H_0+V_i|\psi^i\ra = 
\sum_{\alpha'}
\frac{1}{xy}[-\Delta^{(\alpha)}(x,y)\delta_{\alpha\alpha'}
             +\hat{V}_{\alpha\alpha'}(x) ] F^{(i)}_{\alpha'}(x,y)
\eea
with
\footnote{
For two-body case, Schrodinger equation for full wave
function is written as
\bea
T_l(x)\equiv \frac{-1}{2\mu}\frac{1}{x^2}\frac{d}{dx}(x^2\frac{d}{dx})+\frac{1}{2\mu}\frac{l(l+1)}{x^2} 
\eea
And
\bea
[T_{l'}(x)-E]\Psi^J_{l's ls}(x)
+\sum_{l''}{\cal V}^J_{l'sl''s}(x)\Psi^J_{l''s ls}(x)=0
\eea
with
\bea
{\cal V}^J_{l'sl''s}(x)\equiv
\la {\cal Y}^{JM}_{l's}|V(x)|{\cal Y}^{JM}_{ls}\ra
\eea
Using $\Psi^J_{l''s ls}(x)=\frac{u^J_{l's ls}(x)}{x}$,
above equation can be rewritten as
\bea
[\tilde{T}_{l'}(x)-E]u^J_{l's ls}(x)
+\sum_{l''}{\cal V}^J_{l'sl''s}(x)u ^J_{l''s ls}(x)=0
\eea
with
\bea
\tilde{T}_{l'}(x)=-\frac{1}{2\mu}\frac{d^2}{dx^2}
\eea
Be careful that if there was a derivative in the potential
the last term should written as 
$x \sum_{l''}{\cal V}^J_{l'sl''s}(x)\frac{u ^J_{l''s ls}(x)}{x}$
}
\bea
\la \alpha x y|H_0|\alpha' x' y'\ra&=&\delta_{\alpha\alpha'}
                                 \frac{\delta(x-x')}{xx'}
                                 \frac{\delta(y-y')}{yy'}
                                 (-\Delta^{(\alpha)}(x,y)) \no
\la \alpha xy|V_i|\alpha' x' y'\ra&=&
       \hat{V}_{\alpha\alpha'}(x)
       \frac{\delta(x-x')}{xx'}\frac{\delta(y-y')}{yy'} 
       \quad \mbox{ if local potential}                 
\eea
Thus,
\bea
\frac{1}{xy}\sum_{\alpha'}
[(E+\Delta^{(\alpha)}(x,y))\delta_{\alpha\alpha'}
 -\hat{V}_{\alpha\alpha'}(x)]F^{(i)}_{\alpha'}(x,y)
=\sum_{\alpha'} \hat{V}_{\alpha\alpha'}(x)
 \sum_{\alpha''}\int dx' x'^2 dy' y'^2 
   \la \alpha' x y|P|\alpha'' x' y'\ra 
   \frac{F^{(i)}_{\alpha''}(x',y')}{x' y'}  
\eea
If using the $h$ function in the following section,
we can write
\bea
\sum_{\alpha'}
[(E+\Delta^{(\alpha)}(x,y))\delta_{\alpha\alpha'}
 -\hat{V}_{\alpha\alpha'}(x)]F^{(i)}_{\alpha'}(x,y)
=2\sum_{\alpha'} \hat{V}_{\alpha\alpha'}(x)
 \sum_{\alpha''}\int dz\hat{h}_{\alpha'\alpha''}(x,y,z)  
  F^{(i)}_{\alpha''}(x',y')  
\eea
\section{Permutation}
The representation of permutation operator is in the heart of three body calculation.
Permutation operator gives
\bea
{}_1\la x y\alpha| P_{12}P_{23}+P_{13}P_{23}|x' y' \alpha'\ra_1
={}_1\la xy \alpha|x' y'\alpha'\ra_2+{}_1\la xy \alpha|x' y'\alpha'\ra_3
\eea
where, the meaning of above amplitude is that
overlap matrix element betwen the state with
relative position is $x, y$ and quantum numbers $\alpha$ 
in $1(23)$ configuration and 
the state with relative position $x',y'$ and quantum numbers
$\alpha'$ in $2(31)$ or $3(12)$ configuration. 

By using $P_{13}P_{23}=P_{23}P_{12}P_{23}P_{23}$ and 
\bea
{}_1\la x y \alpha|x' y' \alpha'\ra_3
={}_1\la x y \alpha|P_{13}P_{23}|x' y' \alpha'\ra_1
=(-1)^{l+s+t}(-1)^{l'+s'+t'}{}_1\la x y\alpha|P_{12}P_{23}|x'y'\alpha'\ra_1
={}_1\la x y \alpha|x' y'\alpha'\ra_2
\eea
where we use $(-1)^{l+s+t}=-1$ from the anti-symmetrization of $(23)$ pair
in $|x y\alpha\ra_1$. Thus, we only have to 
consider ${}_1\la x y \alpha|x' y'\alpha'\ra_2$.

\subsection{Explicit form of permutation operator}
There are different representation of permutation
operator. Here, let us use $h_{\alpha\alpha'}$ form.
Assume that
\bea
\left(\begin{array}{c} \vx' \\ \vy'\end{array}\right)
=\left(\begin{array}{cc} c_1 & s_1 \\ s_2 &c_2\end{array}\right)
 \left(\begin{array}{c} \vx \\ \vy\end{array}\right)
\eea
Thus,
\bea
{}_1\la \vx_1 \vy_1|P^{\pm}|\vx'_1\vy'_1\ra_1
=\delta^{(3)}(\vx'_1-c_1\vx_1-s_1\vy_1)
 \delta^{(3)}(\vy'_1-s_2\vx_1-c_2\vy_1)
\eea

Thus finally we have
\bea 
& &{}_1\la x y (l_x l_y)L M| x' y'(l'_xl'_y)L M \ra_2
={}_1\la x y (l_x l_y)L M|P^+| x' y'(l'_xl'_y)L M \ra_1
\no
&\equiv&\int_{-1}^1 dz \delta(x'-|x'_1|)\delta(y'-|y'_1|) 
    \hat{h}^{LM}_{l_xl_y,l'_x l'_y}(x,y,z)\frac{1}{ x y x' y'}
\eea
where,
\bea
|x'_1|&=&\sqrt{(c_1 x)^2+(s_1 y)^2+2 c_1 s_1 x y z} \no
|y'_1|&=&\sqrt{(s_2 x)^2+(c_2 y)^2+2 s_2 c_2 x y z}
\eea
with
\bea
\hat{h}^{LM}_{l_xl_y,l'_xl'_y}(x,y,z)
&=&\sum_k \frac{\hat{k}\sqrt{\hat{l}_x\hat{l}_y\hat{l}'_x\hat{l}'_y}}{2}
        P_k(z)
 \sum_{l_1+l_2=l'_x}\sum_{l_3+l_4=l'_y}
 \sqrt{\frac{(2l'_x+1)!(2l'_y+1)!}{(2l_1)!(2l_2)!(2l_3)!(2l_4)!}}
 \no
& &\times\frac{x^{l_1+l_3+1}y^{l_2+l_4+1}}{(x')^{l'_x+1}(y')^{l'_y+1}}
 c_1^{l_1}c_2^{l_4}s_1^{l_2}s_2^{l_3}
 \sum_{l_5}\hat{l}_5\threejsymbol{k}{l_x}{l_5}{0}{0}{0}
  \threejsymbol{l_1}{l_3}{l_5}{0}{0}{0}
  \no & &\times
  \sum_{l_6}\hat{l}_6(-1)^{l_6+l_x+L}
  \threejsymbol{k}{l_y}{l_6}{0}{0}{0}
  \threejsymbol{l_2}{l_4}{l_6}{0}{0}{0}
  \sixjsymbol{l_x}{l_y}{L}{l_6}{l_5}{k}
  \no & &\times
  \ninejsymbol{l_1}{l_3}{l_5}{l_2}{l_4}{l_6}{l'_x}{l'_y}{L}          
\eea
form is useful to obtain full wave function
in a form
\footnote{
We may think
\bea
(xy)\la \alpha xy|P^+|\psi\ra_1
=\sum_{\alpha'}\int_{-1}^{+1} du \hat{h}_{\alpha\alpha'}(x,y,u)F_{\alpha'}(x',y')
\eea
}
\bea
(x y){}_1\la \alpha x y|(1+P)|\psi\ra_1
&=&(x y){}_1\la \alpha x y|\psi\ra_1+2(xy){}_1\la \alpha x y|P^+|\psi\ra_1 \no
&=&
F_\alpha^1(x,y)
+2\sum_{\alpha'}\int_{-1}^1 dz \hat{h}_{\alpha\alpha'}(x,y,z)
 F^1_{\alpha'}(x',y')
\eea

\bea
\int \int d\hat{x} d\hat{y} {\cal Y}^*_{\alpha}(\hat{x}\hat{y})
     \frac{xy}{x'y'}F_{\alpha'}(x', y')
      {\cal Y}_{\alpha'}(\hat{x}',\hat{y}')
 =\int^{+1}_{-1} du \hat{h}_{\alpha\alpha'}(x,y,u)F_{\alpha'}(x',y')     
\eea

\section{Numerical realization of the Faddeev equation}
The first thing to do is to make the equation discrete.
Let us use spline interpolation.
\bea
F_\alpha(x,y)=\sum_{m_x,m_y} C^\alpha_{m_x,m_y} 
             S^x_{m_x}(x) S^y_{m_y}(y)
\eea  

We can choose collocation point's such that the same $m_x$
and $m_y$ index can be used also for collocation points.
Then, we can represent wave function at collocation
points by indexes $n=(\alpha,m_x,m_y)$, $n'=(\alpha',m'_x,m'_y)$.
This makes,
\bea
F_{\alpha}(x_{m_x},y_{m_x})\to F_{n}
      =\sum_{n'}[B]_{nn'} C_{n'} 
\eea 
with
\bea
[B]_{nn'}
=\delta_{\alpha\alpha'}S^x_{m'_x}(x_{m_x})S^y_{m'_y}(y_{m_y})
\eea

$\bullet$ {\bf identity:}
Then, we can convert
\bea
E\delta_{\alpha\alpha'} F_{\alpha'}(x_{m_x},y_{m_x})
\to E[B]_{nn'}C_{n'}
\eea
$\bullet$ {\bf potential:}
\bea
V_{\alpha,\alpha'}F_{\alpha'}(x_{m_x},y_{m_x})
\to [V]_{nn'} C_{n'}=([\tilde{V}][B])_{nn'}C_{n'}
\eea
where,
\bea
[V]_{nn'}&=&V_{\alpha\alpha'}(x_{m_x})
          S^x_{m'_x}(x_{m_x})S^y_{m'_y}(y_{m_y})
          =([\tilde{V}][B])_{nn'}\no
\left[\tilde{V}\right]_{nn'}&=&
          V_{\alpha\alpha'}(x_{m_x})
          \delta_{m_x,m'_x}\delta_{m_y,m'_y}
\eea
$\bullet$ {\bf kinetic term:}
\bea
\Delta^{\alpha}(x,y)\delta_{\alpha\alpha'}F_{\alpha}(x,y)
\to [\Delta]_{nn'} C_{n'}
\eea
where,
\bea
[\Delta]_{nn'}&=&\delta_{\alpha\alpha'}
                 \frac{1}{m}
             \left[S^{''x}_{m'_x}(x_{m_x})S^y_{m'_y}(x_{m_y})
            -\frac{l_x(l_x+1)}{x_{m_x}^2}
              S^{x}_{m'_x}(x_{m_x})S^y_{m'_y}(y_{m_y})\right.
              \no & &\qquad \left.
            +S^{x}_{m'_x}(x_{m_x})S^{''y}_{m'_y}(x_{m_y})
            -\frac{l_y(l_y+1)}{y_{m_y}^2}
              S^{x}_{m'_x}(x_{m_x})S^y_{m'_y}(y_{m_y}) 
             \right]     
\eea
$\bullet$ {\bf Integral operators}
For integration, we can use Gaussian quadrature
\bea
\int dz F(z)\simeq \sum_{i} w_i F(z_i)
\eea
Then, integral part of the equation become
\bea
& &2\sum_{\tilde{\alpha}}\hat{V}_{\alpha\tilde{\alpha}}(x)
\sum_{\alpha'}
\int dz h_{\tilde{\alpha}\alpha'}(x,y,z)F_{\alpha'}(x',y')\no
&\to& 2\sum_{\tilde{\alpha}}\hat{V}_{\alpha\tilde{\alpha}}(x)
\sum_{\alpha'}
\sum_{i_z} w_{i_z} h_{\tilde{\alpha}\alpha'}(x,y,z_i)
                   F_{\alpha'}(x',y')\no
&\to&2\sum_{\tilde{\alpha}}\hat{V}_{\alpha\tilde{\alpha}}(x)
\sum_{\alpha'}
\sum_{i_z} w_{i_z} h_{\tilde{\alpha}\alpha'}(x,y,z_i)
            \sum_{n''} \delta_{\alpha'\alpha''}
            S_{m''_x}(x')S_{m''_y}(y')C_{n''}   \no
&\to&2\sum_{n''}
\sum_{\tilde{\alpha}}\hat{V}_{\alpha\tilde{\alpha}}(x)
\sum_{i_z} w_{i_z} h_{\tilde{\alpha}\alpha''}(x,y,z_i)             
            S_{m''_x}(x')S_{m''_y}(y')C_{n''}   \no            
&=&2\sum_{n''}[H]_{n n''}C_{n''}    
\eea
where, $[H]_{n n''}$ can be written as
\bea
[H]_{nn''}=\sum_{\tilde{\alpha}}\hat{V}_{\alpha\tilde{\alpha}}(x_{m_x})
\sum_{i_z} w_{i_z} h_{\tilde{\alpha}\alpha''}(x_{m_x},y_{m_y},z_i) 
            S_{m''_x}(x';m_x,m_y,z_i)S_{m''_y}(y';m_x,m_y,z_i)
\eea
for given $n=(\alpha,m_x,m_y)$, $n''=(\alpha'',m''_x,m''_y)$.
and $x'$ and $y'$ are function of $m_x,m_y,z_i$.
We can separate $[H]_{n n'}=\sum_{\tilde{n}}
[\tilde{V}]_{n\tilde{n}}[\tilde{H}]_{\tilde{n}n'}$
with
\bea
[\tilde{H}]_{\tilde{n}n''}
=\sum_{i_z}w_{i_z} 
 h_{\tilde{\alpha}\alpha''}(x_{\tilde{m}_x},y_{\tilde{m}_y},z_i)
 S^x_{m''_x}(x^h(m_x,m_y,z_i))
 S^y_{m''_y}(y^h(m_x,m_y,z_i))
\eea
$\bullet$ {\bf Faddeev equation for bound state:} In abstract matrix notation we have
\bea
(E[B]+[\Delta]-[V])[C]=2[H][C]
\to [A][C]=([V]-[\Delta]+2[H])[C]=E[B][C]
\eea
Though the whole equation is to be $k(N+1)$ dimension 
for each coordinate, bound state vanishes at large separation,
thus, in actual calculation we only need $kN$ dimension.

$\bullet$ {\bf scattering state:}
Scattering problem is different in the fact that the wave function
does not vanish at large separation.
Thus, if we want to express the equation in $kN$ dimension,
there is a term coming from boundary condition.
Let us say the wave function as
\bea
F_\alpha(x,y)&=&\sum_{m'_x=0}^{k (N_x+1)-1}
              \sum_{m'_y=0}^{k (N_y+1)-1}  
              C_{\alpha,m'_x,m'_y} S_{m'_x}(x)S_{m'_y}(y)\no
             &=& \sum_{m'_x=1}^{k N_x}
              \sum_{m'_y=1}^{k N_y}  
              C_{\alpha,m'_x,m'_y} S_{m'_x}(x)S_{m'_y}(y)
             + \sum_{m'_x=1}^{k (N_x+1)-1}
              \sum_{m'_y=k N_y+1}^{k (N_y+1)-1}  
              C_{\alpha,m'_x,m'_y} S_{m'_x}(x)S_{m'_y}(y)
\eea
If $C_{\alpha, m'_x,m'_y}$ outside of the boundary 
is known by the boundary condition,  then
acting $\hat{L}$ makes the first term as
the same form as Faddeev equation of bound state
and remaining terms as additional terms.

For example, if $k=2$ and boundary condition was
$C_{m_x,m_y=2N_y}=1$. Then , by renaming last two spine 
function and its coefficient as $2N_y+1\leftrightarrow 2N_y$
makes,
\bea
F_\alpha(x,y)&=& \sum_{m'_x=1}^{2 N_x}
              \sum_{m'_y=1}^{2 N_y}  
              C_{\alpha,m'_x,m'_y} S_{m'_x}(x)S_{m'_y}(y)
             +\sum_{m'_x=1}^{2 N_x} 
              C_{\alpha,m'_x} C^y_{\alpha,m'_y=2N_y+1} 
              S_{m'_x}(x)S_{2N_y+1}(y)
\eea
where $C_\alpha,m_x$ is a spline coefficients 
of asymptotic 2-body bound state wave function
and the last spline index for $y$ is actually
for $S_{2N_y}(y)$. Then applying $\hat{L}$ gives
usual Faddeev equation of form $[A]C$ and
additional boundary condition term
\bea
[b]_n=\sum_{m'_x=1}^{2 N_x} 
              C_{\alpha,m'_x} C^y_{\alpha,m'_y=2N_y+1} 
              \hat{L} S_{m'_x}(x_{m_x})S_{2N_y+1}(y_{m_y})
\eea
For example, explicit form of $[b]$ term gives
\bea
b_n&=&\sum_{\alpha' j'_x}C_{\alpha'j'_x}
    \left[(E-V_{\alpha\alpha'}(x_{j_x}))
    S^x_{j'_x}(x_{j_x})S^y_{2N_y+1}(y_{j_y})\right.\no & &
    -\frac{\hbar^2}{m}\delta_{\alpha\alpha'}
    \left( S''_{j'_x}(x_{j_x})S_{2N_y+1}(y_{j_y})
           -\frac{l_x(l_x+1)}{x_{j_x}^2}
           S_{j'_x}(x_{j_x})S_{2N_y+1}(y_{j_y})
         \right.\no & &\left.  
         + S_{j'_x}(x_{j_x})S''_{2N_y+1}(y_{j_y})
           -\frac{l_y(l_y+1)}{y_{j_y}^2}
           S_{j'_x}(x_{j_x})S_{2N_y+1}(y_{j_y})   
    \right)\no & &
    +V_{\alpha\alpha'}(x_{j_x})
    \sum_{i_u=1}^{N_u}\dots
\eea

Thus, equation looks like, $[A] c=b$.
\bea
([V]-[\Delta]-E[B]+[H])[C]=[b]
\eea

\section{Linear Algebra in numerical method}
The problem is now to solve linear algebra problem
of either $[A]c=E[B]c$ or $[A]c=b$. However, because of the
large dimension of matrix $[A]$, we cannot use direct
inversion of matrix $[A]$. We need some iterative 
algorithm to solve this problem.

\subsection{Bound state problem}
According to Rimas, there is two iterative algorithm to solve
$[A]x=E[B]x$ type equation. Both are some variation of
Lanczo's algorithm using eq. of form $[B]^{-1}[A]x=E x$.

(1) Inverse iteration method using equation of form
\bea
(E_0-[B]^{-1}[A])\psi=(E_0-E)\psi
\eea
(2) Power method using equation of form
\bea
([V]-[\Delta]+[H])\psi=E[B]\psi\to 
(E_0[B]-[V]+[\Delta])^{-1}[H]\psi=\lambda(E_0)\psi
\eea
\subsection{Scattering problem}
For $Ax=b$ type equation can be solves by

(1) GMRES algorithm

(2) BICGSTAB algorithm

These method requires good preconditioning $M$ such that
\bea
M^{-1}Ax=M^{-1}b,\quad\mbox{ or }, A M^{-1}x=M^{-1}b
\eea

\subsection{TENSOR inversion trick}
Thus, we need either $[B]^{-1}[A]$ or
$(E_0[B]-[V]+[\Delta])^{-1}[H]$ for bound state problem and
also they an be used as a preconditioning for scattering problem.
Because of the block diagonal property of matrix,
we may use $[L][U]$ factorization for 
each block diagonal matrices to obtain inverse of a matrix.
However, it is not effective. Thus, we need to use
tensor inversion trick
which uses the tensor structure of matrices.

Matrix $[M]=(E_0[B]-[V]+[\Delta])$ can be grouped as follows:
\bea
[M]_{n,n'}&=&\delta_{\alpha,\alpha'}
     \left( \left[(E_0-V_{\alpha,\alpha'}
     -\frac{\hbar^2}{m}\frac{l_x(l_x+1)}{x_{j_x}^2}) 
     S_{j'_x}(x_{j_x})
     +\frac{\hbar^2}{m}S''_{j'_x}(x_{j_x})\right] 
     S^y_{j'_y}(y_{j_y}) \right.\no & &\left.
     +S^x_{j'_x}(x_{j_x})
     \left[-\frac{\hbar^2}{m}\frac{l_y(l_y+1)}{y_{j_y}^2})
     S_{j'_y}(y_{j_y})
     +\frac{\hbar^2}{m}S''_{j'_y}(y_{j_y})\right]
     \right)
\eea
{\bf 이 식은 central potential 인 경우에만 성립한다.} 
$[V]$는 partial wave 에 대해서,
일반적으로 diagonal 이 아니므로, 그 경우에는
완전한 tensor decomposition 은 불가능하다. 

Introducing matrices
\begin{align}
&[1^\alpha]_{\alpha,\alpha'} 
&=&\delta_{\alpha,\alpha'} 
&\mbox{(of dimension $N_\alpha\times N_\alpha$)} \no
&[\delta^a_{\alpha_0}]_{\alpha,\alpha'}
&=&\delta_{\alpha_0,\alpha}\delta_{a,\alpha'}
&\mbox{(of dimension $N_\alpha\times N_\alpha$)}\no
& & & &\mbox{$N_\alpha$ distinct matrices} \no
&[N^X]_{j_x,j'_x}
&=& S^x_{j'_x}(x_{j_x})
&\mbox{(of dimension $k_x N_x \times k_x N_x$)}\no
&[N^Y]_{j_y,j'_y}
&=& S^y_{j'_y}(y_{j_y})
&\mbox{(of dimension $k_y N_y \times k_y N_y$)}\no
&[D^X_{\alpha_0}]_{j_x,j'_x}
&=& \left( E_0-V_{\alpha_0,\alpha_0}(x_{j_x})
          -\frac{\hbar^2}{m}\frac{l_x(l_x+1)}{x_{j_x}^2}
          \right) S_{j'_x}(x_{j_x})
          +\frac{\hbar^2}{m}S''_{j'_x}(x_{j_x})
&\mbox{(of dimension $k_x N_x\times k_y N_y$)}\no
& & & &\mbox{$N_\alpha$ distinct matrices} \no 
&[D^Y]_{j_x,j'_x}
&=& -\frac{\hbar^2}{m}\frac{l_x(l_x+1)}{x_{j_x}^2}
          S_{j'_x}(x_{j_x})
          +\frac{\hbar^2}{m}S''_{j'_x}(x_{j_x})
&\mbox{(of dimension $k_x N_x\times k_y N_y$)}
\end{align}
In terms of these matrices, which decomposes
partial wave state, x-space and y-space,
$[B]$ becomes,
\bea
\left[B\right]_{nn'}=[[1^\alpha]_{\alpha\alpha'}\otimes
           [N^X]_{j_x,j'_x}\otimes 
           [N^Y]_{j_y,j'_y}]
\eea
and its inversion
\bea
[B]^{-1}=[[1^\alpha]\otimes[N^X]^{-1}\otimes [N^Y]^{-1}]
\eea
where, $[N^X]^{-1}$, $[N^Y]^{-1}$ can be obtained by
direct methods.
We can have
\bea
\delta_{\alpha,\alpha'}
\left[(E_0-V_{\alpha,\alpha'}
     -\frac{\hbar^2}{m}\frac{l_x(l_x+1)}{x_{j_x}^2}) 
     S_{j'_x}(x_{j_x})
     +\frac{\hbar^2}{m}S''_{j'_x}(x_{j_x})\right]
\to \sum_{\alpha_0}[\delta_{\alpha_0}]_{\alpha\alpha'}
            \otimes [D^X_{\alpha_0}]_{j_x,j'_x}  
\eea
Then,
\bea
[M]=\left[\sum_{\alpha_0}
    [\delta_{\alpha_0}]_{\alpha\alpha'}\otimes
    [D_{\alpha_0}^X]_{j_x j'_x}\otimes [N^Y]_{j_y j'_y}\right]
    +\left[[1]_{\alpha\alpha'}\otimes[N^X]_{j_x j'_x}
     \otimes [D^Y]_{j_y j'_y}\right]
\eea
Matrix inversion of $[M]$ can be done
if we factor out interpolation matrix $[B]$
and introduce diagonal matrices by similarity transformation,
\bea
\left[N^X\right]^{-1}[D^X_{\alpha_0}]
&=&[U^X_{\alpha_0}][d^X_{\alpha_0}][U^X_{\alpha_0}]^{-1}\no
\left[N^Y\right]^{-1}[D^Y]
&=&[U^Y][d^Y][U^Y]^{-1}
\eea
\bea
[M]&=&[B]\left[\sum_{\alpha_0}[1^\alpha]\otimes[U^X_\alpha]^{-1}
             \otimes [U^Y]^{-1}\right]    
    \left(\left[\sum_{\alpha_0}
    [\delta_{\alpha_0}]\otimes
    [d_{\alpha_0}^X]\otimes [1^Y]\right]
    +\left[[1]\otimes[1^X]
     \otimes [d^Y]\right] \right)\no & &
     \left[\sum_{\alpha_0}[1^\alpha]\otimes[U^X_\alpha]
             \otimes [U^Y]\right] \no
\left[M\right]^{-1}&=&\left[\sum_{\alpha_0}[1^\alpha]\otimes[U^X_\alpha]
             \otimes [U^Y]\right]^{-1}
           \left(\left[\sum_{\alpha_0}
    [\delta_{\alpha_0}]\otimes
    [d_{\alpha_0}^X]\otimes [1^Y]\right]
    +\left[[1]\otimes[1^X]
     \otimes [d^Y]\right] \right)^{-1}  \no & &
     \left[\sum_{\alpha_0}[1^\alpha]\otimes[U^X_\alpha]^{-1}
             \otimes [U^Y]^{-1}\right]^{-1}
     [B]^{-1}        
\eea

\section{Krylov space technique: Iterative Orthonormal Vectors}
Suppose we change the bound state equation into a coupled equations as
\bea 
K(-E_b)|\psi\ra=|\psi\ra  
\eea 
This equation holds only when energy is at the correct value.
In most case, we don't know $E_b$ and try to find the $E_b$ by 
trial of various $E$. Then, for a given trial energy $E$, 
the trial equation becomes
\bea 
K(E)|\psi\ra=\lambda(E)|\psi\ra,  
\eea 
so that the $\lambda(E)$ eigenvalues will be different from 1 in most case. 
Only at the correct energy, $\lambda$ becomes 1. 
Problem is that even for the trial equation is too large in dimension. 
(For example, 3-body problem will have dimension 
$$N=N_\alpha\times N_p\times N_q ,$$
where $N_\alpha$ is total number of partial waves, $N_p$ and $N_q$ are
number of mesh in momentum $p$ and $q$)
Thus, we need a different way to solve trial equation to obtain 
eigenvalue $\lambda$ and find a value of E such that largest eigenvalue becomes 1.

One idea is to use some basis functions up to ${\cal N}\ll N$ such that 
the eigenvalue problem can be solved by using standard method. 
How can find such basis functions? 

Suppose we can write states in terms of orthonormal basis functions,
\bea 
|\psi\ra =\sum_{i=0}^{\cal N} c_i |\bar{\varphi}_i\ra 
\eea 
Then, the original problem changes into  ${\cal N}$-dimensional matrix equation ,
\bea 
\sum_{j=0}^{\cal N} B_{ij}(E) c_j=\lambda(E) c_i,
\eea 
where
\bea 
B_{ij}=\la \bar{\varphi}_i|K|\bar{\varphi}_j\ra 
\eea 

If most of eigenvalues are smaller than 1, then the state $K^n|\phi_0\ra $
which is a combination of all basis states, but only the dominant state contribution
will be important as increasing $n$. Thus, if we make a basis functions
in terms of $K^n|\phi_0\ra$, only first small n basis functions will be enough
to describe the largest eigenvalue states.

(a) start from arbitrary normalized state vector $|\bar{\varphi}_0\ra$. 
    generate state $|{\varphi}_1\ra$ by,
    \bea 
    |{\varphi}_1\ra=K|\bar{\varphi}_0\ra
    \eea 
(b) After Orthogonalization and normalization, we get $|\tilde{\varphi}_1\ra$ 
  and $|\bar{\varphi}_1\ra  $
  \bea 
  |\tilde{\varphi}_1\ra&=&|\varphi_1\ra -|\bar{\varphi}_0\ra \la \bar{\varphi}_0|\varphi_1\ra,\no  
  |\bar{\varphi}_1\ra&=&\frac{|\tilde{\varphi}_1\ra}{||\tilde{\varphi}_1||}
  \eea 
  Note that $|\tilde{\varphi}_1\ra$ and $|\bar{\varphi}_1\ra$ are orthogonal to $|\bar{\varphi}_0\ra$.
(c) repeat (a) and (b). Then, in each time, we generate
  \bea 
  |\varphi_{i+1}\ra &=& K|\bar{\varphi}_{i}\ra,\no 
  |\tilde{\varphi}_{i+1}\ra&=&|\varphi_{i+1}\ra
                             -\sum_{n=0}^{i} |\bar{\varphi}_n\ra \la \bar{\varphi}_n|\varphi_{i+1}\ra,\no 
  |\bar{\varphi}_{i+1}\ra&=&\frac{|\tilde{\varphi}_{i+1}\ra}{||\tilde{\varphi}_{i+1}||}.
  \eea   
  Note that $|\tilde{\varphi}_{i+1}\ra$ and $|\bar{\varphi}_{i+1}\ra$  are
  orthogonal to all $|\bar{\varphi}_{n}\ra $ with $n\le i$.
  And we can express state vector
  \bea 
  |\varphi_{i+1}\ra=\sum_{n=0}^{i+1} |\bar{\varphi}_n\ra \la \bar{\varphi}_n|\varphi_{i+1}\ra,
  \quad \la \bar{\varphi}_{i+1}|\varphi_{i+1}\ra=||\tilde{\varphi}_{i+1}||. 
  \eea 
(d) The matrix element $B_{ij}$ can be computed as
  \bea 
  B_{ij}&=&\la \bar{\varphi}_i|K|\bar{\varphi}_j\ra=\la \bar{\varphi}_i|{\varphi}_{j+1}\ra \no 
        &=& 0\quad \mbox{for } i> j+1 \no 
        &=&  ||\tilde{\varphi}_{j+1}||\quad \mbox{for } i= j+1 \no 
        &=& \la \bar{\varphi}_i|\varphi_{j+1}\ra\quad \mbox{for } i< j+1 \no 
  \eea 
  
  If kernel K is a hermitian, we may get
  \bea 
  B_{ij}&=&\la \bar{\varphi}_i|K|\bar{\varphi}_j\ra=\la{\varphi}_{i+1}|\bar{\varphi}_{j}\ra \no 
        &=& 0 \quad \mbox{ for } j > i+1  
  \eea 
  In this case, the matrix $B$ will be block-diagonal(?) form.
  
(e) Now use the standard numerical methods to solve the matrix eigenvalue equation,
  \bea 
  B c=\lambda c
  \eea 
  since ${\cal N}\ll N$.
  
In practice, the ${\cal N}$ is not known a priori. Thus, we need to check convergence
of the eigenvalue by increasing ${\cal N}$ values. (i.e. compare the eigenvalue 
$\lambda^{(n)}$ and $\lambda^{(n+1)}$ by increasing $n$ until no eigenvalue change occurs.) 
In fact, we also need to test convergence of $\lambda$ to 1 by changing trial energy $E$.

This is similar to the Lanczos method for
\bea 
A x= B x, \to A x= \lambda B x \to A^{-1} B x=\frac{1}{\lambda} x
 \to H x=\Lambda x
\eea 
where $\Lambda=1/\lambda\to 1$ for correct energy. (Also, $\Lambda=1$ will be
the largest eigenvalue for the equation $H x=\Lambda x$. 

In case of scattering problem, the matrix equation of form $A\cdot x=b$ have to be
solved. 

For example, 2-body T-matrix equation basically looks like
\bea 
t(k,k_0;E)=V(k,k_0)+\int K(k,k';E)t(k',k_0;E)
\eea 
For given half-on-shell momentum $k_0$, we can expand in terms of orthonormal basis functions
(similarily for potential and kernel),
\footnote{We may consider, 
$$\la k|W\ra= \la k|\hat{t}(E)|k_0\ra = \sum_{i=0}^{\cal N} c_i \bar{\varphi}_i(k) $$
} 
\bea 
|W\ra &=& \hat{t}(E)|\phi_0\ra =\sum_{i=0}^{\cal N} c_i |\bar{\varphi}_i\ra 
\eea 
\bea 
\la \bar{\varphi}_i|W\ra = c_i 
 = \la \bar{\varphi}_i|V(E)|\phi_0\ra 
   +\sum_{j=0}^{\cal N} \la \bar{\varphi}_i|K(E)|\bar{\varphi}_j\ra 
   c_j 
\eea 
We may set starting vector $|\tilde{\varphi}_0\ra=V(E)|\phi_0\ra$ and define normalized vector
\footnote{
In practice, we would need partial wave expression for each state vectors. 
For a case $|\phi_0\ra$ is a $|^1S_0;k_0\ra$ state case, its partial wave 
expression will be
$\alpha, k_i|\phi_0\ra =\delta_{\alpha, ^1S_0}\delta_{k_i,k_0}$. And,
$\la \alpha,k_i|\tilde{\phi}_0\ra = \sum_{\beta}\sum_j w_{\beta,j} \la \alpha k_i|V|\beta k_j \ra
   \la \beta k_j|\phi_0\ra $
} 
\bea 
|\bar{\varphi}_0\ra =\frac{|\tilde{\varphi}_0\ra}{||\tilde{\varphi}_0||}.
\eea 
Then, 
\bea 
c_i&=& \la \bar{\varphi}_i|\bar{\varphi}_0\ra ||\tilde{\varphi}_0||
    +\sum_{j=0}^{\cal N} B_{ij} c_j \no 
   &=& \delta_{i0} ||\tilde{\varphi}_0||+ \sum_{j=0}^{\cal N} B_{ij} c_j
\eea 
We get the matrix equation ,
\bea 
& &\sum_{j=0}^{\cal N} A_{ij} c_j = f_i ,
\eea 
where
\bea  
& & A_{ij}=\delta_{ij}-B_{ij},\no 
& & f_i=\delta_{i0} ||\tilde{\varphi}_0||.
\eea 
The procedure to obtain orthonormal vectors are the same as previous case,

(a) get $|\bar{\varphi}_0\ra $

(b) get $|\varphi_1\ra$,$|\tilde{\varphi}_1\ra$,$|\bar{\varphi}_1\ra$,
   \bea 
   |\varphi_1\ra&=&K|\bar{\varphi}_0\ra \no 
   |\tilde{\varphi}_1\ra &=&|\varphi_1\ra-|\bar{\varphi}_0\ra\la \bar{\varphi}_0|\varphi_1\ra
    =|\bar{\varphi}_1\ra  ||\tilde{\varphi}_1||
   \eea   

(c) repeat to get $|\varphi_{i+1}\ra$,$|\tilde{\varphi}_{i+1}\ra$,$|\bar{\varphi}_{i+1}\ra$,
   \bea 
      |\varphi_{i+1}\ra&=&K|\bar{\varphi}_{i}\ra \no 
      |\tilde{\varphi}_{i+1}\ra
      &=&|\varphi_{i+1}\ra-\sum_{n=0}^{i} |\bar{\varphi}_{n}\ra\la \bar{\varphi}_n|\varphi_{i+1}\ra
      \no 
      |\bar{\varphi}_{i+1}\ra&=&\frac{|\tilde{\varphi}_{i+1}\ra}{ ||\tilde{\varphi}_{i+1}|| }\no 
   \eea 

(d) solve $\sum_{j=0}^{\cal N} A_{ij} c_j = f_i$ for $c_{j}$

(e) Convergence test as increasing ${\cal N}$:
   first check $|c_{\cal N}|$ is small enough. 
    And then, test for the original integral equation, 
    $$\epsilon=\max(|\la k_{i}|W\ra -\la k_{i}|\tilde{\varphi}_0\ra 
                                     -\la k_{i}|K|W\ra |), \quad \mbox{ for } i=1,\dots, N
    $$
   momentum quadrature may be chosen such that $k_{N}=k_0$ on-shell momentum. 
   
 {\bf Note:} Notation $|k_i\ra$ in fact includes all momentum mesh points and partial wave indices. For example, 


\subsection{Another explanation}
General eigenvalue problem,
\bea 
A\cdot v = \lambda \cdot v, 
\eea 
with $A$ is a $n\times n$ matrix. Eigen vector $v$ can be rewritten in the basis
of the power-iterated subspace, $\{q, Aq, A^2q, A^3 q,\cdots  \}$. Then
one could use a finite number of m vectors to orthogonally transform the matrix A into 
a Hessenberg matrix $H_m$, from which the eigenvalues and vectors are calculated faster than a 
general matrix. (Hessenberg matrix have matrix elements zero for upper or lower part.)
After orthogonalization, basis is given by $p_1,\cdots,p_m$ and applied to the general matrix A,
we get the $m\times m$ Hessenberg matrix,
\begin{equation} 
H_m =\left(\begin{array}{cccc} 
            h_{1,1} & h_{1,2} & \dots & h_{1,m} \\
            h_{2,1} & h_{2,2} & \dots & h_{2,m} \\
            0       & h_{3,2} & \dots & h_{3,m} \\
            0       &  0      & \dots & h_{4,m} \\
                    &         & \dots &         \\
                    &         & \dots & h_{m,m} \\ 
           \end{array} \right) 
\end{equation} 
Now the equation becomes something like $H_{m} \cdot p = \lambda p$ and can be solved 
fast. 
\section{RGM and GCM}
Reference: Microscopic Cluster Models by P. Descouvemont and M. Dufour.

Cluster Model은 핵을 일종의 클러스터들의 합으로 생각하여 문제를 단순화 시킨다. 당연히, 이때 사용되는 
클러스터들 간의 힘은 유효핵력이 된다. miscroscopic 방법은 클러스터의 내부 구조까지 고려하여, anti-symmetrization을 
생각한다.  단, 같은 핵이라고 해도, 핵의 상태에 따라 서로 다른 클러스터링으로 취급하는 것이 좋을 수 있다. 

Resonating group method(RGM) wave function for two-cluster 는  
\bea 
\Psi={\cal A} \phi_1 \phi_2 g({\bm \rho})
\eea 
where, ${\cal A}$ is an anti-symmetrization operator including the internal degree of freedom
in clusters and $g({\bm \rho})$ is unknown relative wave function which should be obtained by 
cluster Schrodinger equation. 

Basically, Generator coordinate method(GCM)은 RGM과 같지만, relative wave function을 
Gaussian basis function으로 expand 하는 것이다. 일반적으로는 total wave function에 여러가지 종류의 
clustering 이 포함된다.

RGM, GCM의 microscopic 계산에는 계산의 편리함때문에 보통   Volkov, Minesotta 타입의 단순화된 핵력을 이용한다. 

\subsection{RGM equation}
Let us consider two cluster system. 
Coordinates of clusters are defined as
\bea 
\vR_{cm,1}=\frac{1}{A_1}\sum_{i=1}^{A_1} \vr_i,\quad 
 \vR_{cm,2}=\frac{1}{A_2}\sum_{i=A_1+1}^{A} \vr_i
\eea 
\bea 
\vR_{cm}=\frac{1}{A}(A_1\vR_{cm,1}+A_2\vR_{cm,2}),
\quad \vrho=\vR_{cm,2}-\vR_{cm,1}.
\eea 
각각의 클러스터의 wave function은 $\phi_1(\xi_{1i})$ 와 
$\phi_2(\xi_{2i})$ 로 나타내진다. 
\bea 
\xi_{1,i}=\vr_i-\vR_{cm,1}, \quad 
\xi_{2,i}=\vr_i-\vR_{cm,2}.
\eea 
But, note that not all $\xi_{i}$ are independent.
\bea 
\sum_{i=1}^{A_1}\xi_{1i}=\sum_{i=A_i+1}^{A}\xi_{2i}=0.
\eea 
RGM wavf function,
\bea 
\Psi(\xi_{1i},\xi_{2i},\vrho)
={\cal A}\phi_1(\xi_{1i})\phi_2(\xi_{2i})g(\vrho)
\eea 
where, anti-symmetrization operator is 
with Permutation operator $P_{p}$,
\bea 
{\cal A}=\sum_{p=1}^{A!}\epsilon_p P_{p}
\eea    
Note that ${\cal A}$ is not a projector,
\bea 
{\cal A}^2=A!{\cal A}
\eea 

Re-writing,
\bea 
\Psi={\cal A}\phi_1\phi_2 g(\vrho)
    =\int d\vr {\cal A}\phi_1\phi_2 \delta(\vrho-\vr)g(\vr)  
\eea 
The Schr\"odinger equation in CM coordinate,(CM motion is removed) 
\bea 
& & (H - E_{T})\Psi =0
 \to \int d\vrho' \left[ {\cal H}(\vrho,\vrho')
  -E_T {\cal N}(\vrho,\vrho') \right]g(\vrho') \no 
& & {\cal N}(\vrho,\vrho')
    =\la \phi_1 \phi_2 \delta(\vrho-\vr)| 1| {\cal A}   \phi_1 \phi_2 \delta(\vrho'-\vr)\ra ,\no 
& & {\cal H}(\vrho,\vrho')
=\la \phi_1 \phi_2 \delta(\vrho-\vr)| H | {\cal A}   \phi_1 \phi_2 \delta(\vrho'-\vr)\ra .  
\eea 
where integration over internal coordinates $\xi$ 
and relative coordinates $\vr$ are implied. 
One can simplify the equation by writing,
\bea 
{\cal A}&=& 1+{\cal A}' ,\no 
H&=& H_1+H_2+H_{rel}.
\eea 
여기서, ${\cal A}'$은 다른 클러스터 사이에서의 exchange terms 만,
$H_i$는 internal Hamiltonians of the clusters,
\bea 
H_{rel}&=&-\frac{\hbar^2}{2\mu}\Delta_\rho
   +\sum_{i=1}^{A_1}\sum_{j=A_1+1}^{A} v_{ij},
   \quad \mu=\mu_0 m_N, \quad \mu_0=\frac{A_1A_2}{A_1+A_2},\no 
E_i&=&   \la \phi_i|H_i|\phi_i\ra,\no 
E_{rel}&=&E_T-E_1-E_2.
\eea 
(각 cluster wavr function의 antisymmetrization은 ?
이미 wave function에 고려되어 있기 때문에 쓰지 않는다인가?)
Now non-local Kernels are
\bea 
{\cal N}(\vrho,\vrho')
&=&\delta(\vrho-\vrho')+{\cal N}_E(\vrho,\vrho'),\no 
{\cal N}_E(\vrho,\vrho')
&=& \la \phi_1\phi_2\delta(\vrho-\vr)|{\cal A}'\phi_1\phi_2\delta(\vrho'-\vr)\ra, \no 
{\cal H}(\vrho,\vrho')
&=& (E_1+E_2)\delta(\vrho-\vrho') \no & &
+\la \phi_1\phi_2 \delta(\vrho-\vr)|H'|\phi_1\phi_2 \delta(\vrho'-\vr)\ra 
+\la \phi_1\phi_2 \delta(\vrho-\vr)|H'|{\cal A}'\phi_1\phi_2 \delta(\vrho'-\vr)\ra \no 
&=&
\left(-\frac{\hbar^2}{2\mu}\Delta_\rho+ V_D(\vrho) 
   +E_1+E_2\right) \delta(\vrho-\vrho')
   +{\cal H}_{E}(\vrho,\vrho'),
\eea 
\bea 
V_D(\vrho)=\la \phi_1 \phi_2|\sum_{i=1}^{A_1}\sum_{j=A_1+1}^{A}
                 v_{ij}|\phi_1 \phi_2\ra 
\eea 
Thus RGM equation is 
\bea 
\left(-\frac{\hbar^2}{2\mu}\Delta_\rho+ V_D(\vrho) \right)g(\vrho)
+\int K(\vrho,\vrho')g(\vrho') d\vrho'
= E g(\vrho),
\eea 
with 
\bea 
K(\vrho,\vrho')={\cal H}_{E}(\vrho,\vrho')-E_T {\cal N}_{E}(\vrho,\vrho').
\eea 
따라서, 일단 Kernel이 준비되면, RGM equation을 통해 
$g(\vrho)$ 를 얻을 수 있다. 
RGM에서 어려운 부분은 이러한 overlap과 Hamiltonian
을 준비하는 것이다.

\subsubsection{Example, $\alpha+n$ system}
Suppose $\alpha$ and neutron system,
\bea 
\phi_1&=&\Phi_\alpha(\xi_1,\xi_2,\xi_3)|n_1\downarrow n_1\uparrow
  p_1 \downarrow p_1 \uparrow\ra ,\no 
\phi_2&=&|n_2 \downarrow  \ra, 
\eea 
여기서, $n_1,p_1$ 은 1st cluster 에 속한다는 것을 나타낸다. 
($\phi_{1,2}$는 이미 antisymmetrization이 이루어져 있다고 볼 수 있다.)
Simple approximation,
\bea 
\Phi_\alpha(\xi_1,\xi_2,\xi_3)
 =\frac{1}{N}\exp(-\nu \sum_{i=1}^4 \xi_i^2)
\eea  
with normalization 
\bea 
\la \Phi_\alpha|\Phi_\alpha\ra 
=1=\frac{1}{N^2}(\frac{\pi^3}{32\nu^3})^{3/2}.
\eea 
The only exchange operator $P_{15}$ 
contribute to the antisymmetrization. 
(Because external neutron is spin down. 허어??)

Note that in CM coordinate,
\bea 
\vr_{i=1,2,3,4}=-\frac{1}{5}\vrho+\xi_i,\quad 
\vr_5=\frac{4}{5}\vrho.  
\eea 
From
\bea 
P_{15}\xi_1&=& P_{15}
  \left(\vr_1-\frac{1}{4}\sum_{i=1}^4\vr_i \right) 
 =\vr_5-\frac{1}{4}(\vr_5+\vr_2+\vr_3+vr_4)
 =\frac{3}{4}\vrho+\frac{1}{4}\xi_1,\no 
 P_{15}\xi_2&=& -\frac{1}{4}\vrho+\frac{1}{4}\xi_1+\xi_2,\no 
 P_{15}\xi_3&=& -\frac{1}{4}\vrho+\frac{1}{4}\xi_1+\xi_3,\no 
 P_{15}\vrho&=&-\frac{1}{4}\vrho+\frac{5}{4}\xi_1,
\eea 
we get 
\bea 
P_{15}\Phi_\alpha\delta(\vrho-\vr)
=\Phi_\alpha\exp\left[-\frac{4\nu}{5}(r^2-(P_{15}\vrho)^2)   \right]\delta(P_{15}\vrho-\vr)
\eea 
By using 
\bea 
\int |\Phi_\alpha(\xi_1,\xi_2,\xi3)|^2d\xi_2 d\xi_3=
\left(\frac{\sqrt{3}\pi}{6\nu}\right)^3 \exp(-\frac{8\nu}{3}\xi_1^2),
\eea 

\bea 
{\cal N}_{E}(\vrho,\vrho')
&=&-\int d\vr d\xi_1 d\xi_2 d\xi_3 \Phi_\alpha(\xi_1,\xi_2,\xi_3)
\delta(\vrho-\vr) P_{15}\Phi_\alpha(\xi_1,\xi_2,\xi_3)
\delta(\vrho'-\vr) \no 
&=&-(\frac{4}{5})^3(\frac{8\nu}{3\pi})^{3/2}
 \exp\left[ -\frac{4\nu}{75}(17\rho^2+17\rho^{'2}+16\vrho\cdot\vrho')
 \right].  
\eea 

\subsubsection{Generator Coordinate Method}
Expand the radial function $g(\vrho)$ over Gaussian
functions centered at different locations,
called the {\bf generator coordinates}.

Let us define one-center Slater determinant,
\bea 
\Phi_1(\vr_1\cdots \vr_{A_1}; {\bm S}_1)
=\frac{1}{\sqrt{A_1!}} \det \{\hat{\varphi}({\bm S}_1) \dots
  \hat{\varphi}({\bm S}_1) \},
\eea 
where 
\bea 
\hat{\varphi}_i({\bm S})&=&\varphi_i(\vr,{\bm S})|m_{s_i}\ra 
  |m_{t_i}\ra,\no 
 \varphi_i(\vr,{\bm S})
 &=& \varphi_{0s}(\vr,{\bm S})=(\pi b^2)^{-3/4}
  \exp(- \frac{(\vr-{\bm S})^2}{2b^2} ) \mbox{ for s-orbital},\no 
 &=& \varphi_{1p\mu}(\vr,{\bm S})
    =\frac{\sqrt{2}}{b}(\vr_\mu-{\bm S}_\mu)\varphi_{0s}(\vr,{\bm S})  \mbox{ for p-orbital}
\eea 
By using translation invariant determinant, $\phi_1$
\bea 
\phi_1(\xi_{1i}) ,\quad \xi_{1i}=\vr_i-{\vR}_{cm,1}.
\eea 
We may express, one-center Slater determinant as
\footnote{
One can simply show this by 
\bea 
 \sum_{i}\frac{(\vr_i-{\bm S})^2}{2b^2}
=\sum_{i}\frac{(\xi_{1,i}+{\bm S}-\vR_{cm,1})^2}{2b^2}
=\left(\sum_{i}\frac{\xi_{1,i}^2}{2b^2} \right) 
 +\frac{A_1}{2b^2} ({\bm S}-\vR_{cm,1})^2
\eea 
} 
\bea 
\Phi(\vr_{i};{\bm S} )
=\exp\left(-\frac{A_1}{2b^2}(\vR_{cm,1}-{\bm S})^2\right) \phi_1(\xi_{1i}) 
\eea 

We may express full $A-$body Slater determinant 
by using two kinds of basis functions with different center,
$\varphi_i(\vr,{\bm S}_1)$ and $\varphi_i(\vr,{\bm S}_2)$.
We may set the origin 
along the $\vR={\bm S}_2-{\bm S}_1$.  
In other words, we can choose coordinates as
${\bm S}_1=-\lambda \vR$, ${\bm S}_2=(1-\lambda) \vR$.
However, note that ${\bm S}_i$ does not necessarily the 
same as the center of mass of each cluster $\vR_i$. 

Define two-cluster Slater determinant as
\bea 
\Phi(\vr_{i};\vR)
&=&\frac{1}{\sqrt{A!}}\det \{ 
  \hat{\varphi}(\vr_1,-\lambda\vR)\dots 
  \hat{\varphi}(\vr_{A_1},-\lambda\vR) 
  \dots \hat{\varphi}(\vr_{A},(1-\lambda)\vR) \} \no 
 &=&\frac{1}{\sqrt{N_0}}{\cal A} \Phi_1(\vr_i; -\lambda \vR)
   \Phi_2(\vr_i; (1-\lambda) \vR), \quad N_0=\frac{A!}{A_1!A_2!}
\eea 
This can be rewritten as 
\bea 
\Phi(\vr_{i};\vR)
=\frac{1}{\sqrt{N_0}}\Phi_{cm}(\vR_{cm};\vR)
 {\cal A} \phi_1(\xi_{1i})\phi_2(\xi_{2i})
 \Gamma(\rho,\vR)
\eea 
where\footnote{
This can be shown from
\bea 
& &A_1(\vR_{cm,1}-(-\lambda \vR))^2+A_2(\vR_{cm,2}-(1-\lambda) \vR)^2 \no 
&=& A_1(\vR_{cm,1}+\vR(\lambda-\frac{A_2}{A})+\frac{A_2}{A}(\vR-\vrho)  )^2
+A_2(\vR_{cm,1}+\vR(\lambda-\frac{A_2}{A})-\frac{A_1}{A}(\vR-\vrho) )^2
\eea 
} 
\bea 
\Phi_{cm}(\vR_{cm};\vR)
&=& (\frac{A}{\pi b^2})^{3/4}\exp\left( 
    -\frac{A}{2b^2}[\vR_{cm}+\vR(\lambda-\frac{A_2}{A})]^2 \right) ,
    \no 
\Gamma(\rho,\vR)&=& (\frac{\mu_0}{\pi b^2})^{3/4}
 \exp(-\frac{\mu_0}{2b^2}(\rho-\vR)^2).    
\eea 
This expression makes the c.m. and radial coordinates uncoupled. In other words, the two-cluster Slater determinant 
can be treated separately for the c.m. and radial part
though the wave function itself is not translation invariant. 
Then, two-cluster Slater determinant is 
\bea 
\Phi(\vr_i;\vR)=\Phi_{cm}\bar{\Phi}(\xi;\vR),
\eea 
where $\bar{\Phi}(\xi,\vR)$ is independent of c.m. coordinate. 
Note that $\Phi(\vr_i;\vR)$ is good for numerical
calculation because it is a Slater determinant.
The overlap of Slater determinant is 
\bea 
\la \Phi(\vR)|\Phi(\vR')\ra 
&=&\la \Phi_{cm}|\Phi_{cm}\ra 
 \la \bar{\Phi}(\vR)|\bar{\Phi}(\vR)\ra ,\no 
\la \Phi_{cm}|\Phi_{cm}\ra
&=& \exp\left( -\frac{A(\lambda-A_2/A)^2}{4b^2}(\vR-\vR')^2\right) . 	
\eea 
In other words, one can obtain overlap between
basis functions $\bar{\Phi}(\vR) $ .

We may consider $\frac{1}{\Phi_{cm}}\Phi(\vr_i;\vR)$
as a kind of basis function for the radial function 
of RGM.  In other words,
\bea 
\Psi =\Phi_{cm}^{-1}\int f(\vR)\Phi(\vR) d\vR 
\eea 
where $\Psi$ is invariant under translation
and generator function is to be determined from Hamiltonian. 

중요: Actually, the difference between $\Psi$ and $\Phi$ is the main
message from GCM. While $\Psi$ is the RGM wave function which describes internal 
structure of a nuclei, it is difficult to use because of the complication 
from anti-symmetrization operator. On the other hand, $\Phi$ is easier to manipulate and
make calculation, while it contains additional center of mass motion. 
By using GCM, one can separate internal degrees of freedom and the center of mass
from Slater determinant. 

This can be expressed as 
\bea 
\Psi={\cal A}\phi_1 \phi_2 g(\rho),
\eea 
with 
\bea 
g(\rho)=\int f(\vR)\Gamma(\rho,\vR) d\vR. 
\eea 

Generally speaking A-body Slater determinant wave function made from s.p. wave functions are not translation invariant. 
However, by using the GCM basis with c.m. correction, 
we may express,
\bea 
\Phi(\vr_i)\simeq \sum_{\vR_n} f(\vR_n) \Phi_{cm}(\vR_n) \bar{\Phi}(\vR_n) 
\eea 
where $\bar{\Phi}(\vR_n) $ is translation invariant.

The generator function $f(\vR)$ can be obtained by solving 
\bea 
\int [H(\vR,\vR')-E_T N(\vR,\vR')]f(\vR')d\vR'=0,
\eea 
where GCM kernels are
\bea 
N(\vR,\vR')&=&\la \bar{\Phi}(\vR)|\bar{\Phi}(\vR')\ra ,\no 
H(\vR,\vR')&=&\la \bar{\Phi}(\vR)|H|\bar{\Phi}(\vR')\ra.
\eea 
Note that one can use the Slater determinant wave functions
to compute the matrix elements with cm correction.
For example, overlap kernel can be computed as
$\la \bar{\Phi}(\vR)|\bar{\Phi}(\vR')\ra/\la \Phi_{cm}|\Phi_{cm}\ra $.
Thus, GCM simplifies the calculation of Kernels
and by discretization 
\bea 
g(\rho)\simeq \sum_{n}  f(\vR_n)\Gamma(\rho,\vR_n)
\eea 
the equation for generator function becomes a diagonalization
of a matrix. 
\end{document}
